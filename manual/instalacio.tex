% Time Log
%

%%%%%%%%%%%%%%%%%%%%%%%%%%%%%%%%%%%%%%%%%%%%%%%%%%%%%%%%%%%%%%%%%%%%%%
\chapter{Instalaci� i execuci� de l'eina}
%%%%%%%%%%%%%%%%%%%%%%%%%%%%%%%%%%%%%%%%%%%%%%%%%%%%%%%%%%%%%%%%%%%%%%

%%%%%%%%%%%%%%%%%%%%%%%%%%%%%%%%%%%%%%%%%%%%%%%%%%%%%%%%%%%%%%%%%%%%%%
\section{Contingut de l'arxiu comprimit}
%%%%%%%%%%%%%%%%%%%%%%%%%%%%%%%%%%%%%%%%%%%%%%%%%%%%%%%%%%%%%%%%%%%%%%

% TODO: Comentar millor el contingut del zip

A l'arxiu comprimit s'inclouen els fonts, la documentaci�, l'executable
per a MS-DOS i els fitxers de configuraci�.

Per a l'execuci� del programa, nom�s s�n necessaris l'executable i
els tres arixius de configuraci� {\tt bioscena.ini}, {\tt agents.ini}
i {\tt opcodes.ini}.
Han d'estar tots quatre al mateix directori en el moment de l'execuci�.

�s possible que no estigui disponible l'executable per la plataforma
escollida. Es pot generar tot seguint les instruccions de compilaci�
de l'apartat \label{sec:Compilacio}.

{\bf L'executable necessita una consola de text de 80x50 i seq��ncies de 
terminal ANSI.} Els seg�ents apartats expliquen com fer-ho a diferents
sistemes.

Si, al sistema dest�, no �s possible treballar amb un terminal 80x50
es poden canviar f�cilment el codi font les coordenades dels objectes
gr�fics.

% TODO: Aixo fins que posi quelcom per configurar-ho

%%%%%%%%%%%%%%%%%%%%%%%%%%%%%%%%%%%%%%%%%%%%%%%%%%%%%%%%%%%%%%%%%%%%%%
\section{Terminal ANSI de 50 linies sota Windows 95}
%%%%%%%%%%%%%%%%%%%%%%%%%%%%%%%%%%%%%%%%%%%%%%%%%%%%%%%%%%%%%%%%%%%%%%

Per suportar les seq��ncies de control ANSI, cal haver iniciat l'ordinador
o una sessi� MSDOS amb la l�nia 
\begin{verbatim}
DEVICE=C:\WINDOWS\COMMAND\ANSI.SYS
\end{verbatim}
dintre del config.sys.

Per posar la pantalla a 80x50, si �s un executable de DOS (DJGPP), 
cal crear un acces directe i, a les seves propietats, al separador 
'Pantalla' especificar 50 linies.

% TODO: Insertar grafic configurant 80X50 Windows95

Si �s un executable Win32 per a consola (compilat amb Microsoft Visual C++), 
automaticament es posa a 80x50.

%%%%%%%%%%%%%%%%%%%%%%%%%%%%%%%%%%%%%%%%%%%%%%%%%%%%%%%%%%%%%%%%%%%%%%
\section{Terminal ANSI de 50 linies sota Windows-NT}
%%%%%%%%%%%%%%%%%%%%%%%%%%%%%%%%%%%%%%%%%%%%%%%%%%%%%%%%%%%%%%%%%%%%%%

Per fer servir les seq��ncies ANSI amb l'executable MS-DOS, cal 
seguir les seg�ents passes:
\begin{itemize}
\item Si ja existeix l'arxiu {\tt config.nt} al comprimit, cal 
copiar-lo al directori del executable, i saltar-se els dos seg�ents 
passos.
\item Copiem l'arxiu \verb"c:\WINNT\SYSTEM32\config.nt" al directori de l'executable.
\item L'editem i afegim la l�nia:
\begin{verbatim}
DEVICE=$WINNT$\SYSTEM32\ANSI.SYS
\end{verbatim}
\item Obrim les propietats de l'executable DOS
\item Premem el bot� 'Avanzada' del separador 'Programa'
\item A la capsa pel {\tt config.nt} posem l'encaminament del personalitzat.
\end{itemize}

Per canviar el nombre de linies de la pantalla, si ho fem al mateix lloc
que cal fer-ho a Windows 95, no en fa cas. Cal seguir les seg�ents passes:
\begin{itemize}
\item Executar el programa
\item Accedir a l'opcio propietats del menu de sistema de la finestra.
Surtir� un di�leg de propietats diferent que el de Windows 95. 
\item Cal canviar a una lletra suficient petita, si no, ignorar� la resta de canvis. 
\item Augmentar el tamany del buffer de sortida i de l'�rea de pantalla fins a 50 l�nies com a m�nim.
\item En aceptar, posar que guardi les opcions per a proximes execucions.
\end{itemize}
Si tot va b�, la proxima vegada que ho executem, sortir� b�.

Un executable Win32 es posar� automaticament a 80x50, per�, no s'ha 
provat encara com fer funcionar l'ANSI.SYS amb un executable Win32 sota 
Windows NT.

% TODO: Insertar grafic configurant 80X50 WindowsNT

%%%%%%%%%%%%%%%%%%%%%%%%%%%%%%%%%%%%%%%%%%%%%%%%%%%%%%%%%%%%%%%%%%%%%%
\section{Terminal ANSI de 50 linies sota Linux}
%%%%%%%%%%%%%%%%%%%%%%%%%%%%%%%%%%%%%%%%%%%%%%%%%%%%%%%%%%%%%%%%%%%%%%

A linux ja hi ha ANSI per defecte tant a terminals en mode text com a X-terms.

Per obtindre un terminal 80x50 aqu� hi han unes solucions.

\begin{tabular}{lp{5in}}
Xterms & Simplement, cal augmentar el tamany de la finestra fins que hi
capiguen 80x50 linies. \\
Terminals ordinaris & Cal especificar al programa d'arranc {\tt lilo} el mode gr�fic que fan servir els terminals. Consulta el manual.
\end{tabular}

%%%%%%%%%%%%%%%%%%%%%%%%%%%%%%%%%%%%%%%%%%%%%%%%%%%%%%%%%%%%%%%%%%%%%%
