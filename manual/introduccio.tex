
%%%%%%%%%%%%%%%%%%%%%%%%%%%%%%%%%%%%%%%%%%%%%%%%%%%%%%%%%%%%%%%%%%%%%%
\chapter{Introducci�}
%%%%%%%%%%%%%%%%%%%%%%%%%%%%%%%%%%%%%%%%%%%%%%%%%%%%%%%%%%%%%%%%%%%%%%

%%%%%%%%%%%%%%%%%%%%%%%%%%%%%%%%%%%%%%%%%%%%%%%%%%%%%%%%%%%%%%%%%%%%%%
\section{Sobre Bioscena}
%%%%%%%%%%%%%%%%%%%%%%%%%%%%%%%%%%%%%%%%%%%%%%%%%%%%%%%%%%%%%%%%%%%%%%

{\tt Bioscena} �s una aplicaci� que permet simular entorns bi�logics
evolutius, �s a dir, biosistemes on viuen organismes que evolucionen
al llarg de generacions competint pels recursos.

L'eina ha estat desenvolupada al Departament d'Inform�tica d'Enginyeria 
la Salle amb l'objectiu de fer-la servir de base per a futurs
desenvolupaments de sistemes similars de Vida Artificial.
Per aquest motiu, ha estat dissenyada pensant sobretot en la 
flexibilitat i en la facilitat d'ampliaci� futura.

El sistema en s� permet configurar molts par�metres i presentar
als organismes escenaris molt diversos.

A m�s, els m�duls del programa estan acoblats molt d�bilment, el 
que permet crear i intercanviar nous m�duls amb la mateixa funci� 
que un d'existent.

Es podria reprogramar, per exemple, la geometria i composici� del medi, 
el funcionament intern dels organismes...

No es volia lligar els futurs sistemes basats en aquest a una
plataforma en concret i el resultat ha sigut una aplicaci� molt 
portable.
L'eina ha estat provada en els sistemes operatius MS-DOS, 
MS-Windows, Windows-NT, Linux i Sun-OS.


%%%%%%%%%%%%%%%%%%%%%%%%%%%%%%%%%%%%%%%%%%%%%%%%%%%%%%%%%%%%%%%%%%%%%%
\section{Objectius del manual}
%%%%%%%%%%%%%%%%%%%%%%%%%%%%%%%%%%%%%%%%%%%%%%%%%%%%%%%%%%%%%%%%%%%%%%

El present manual cobreix diferents nivells d'�s i coneixement sobre l'eina.

Per un costat, estan els usuaris que simplement volen probar l'eina.
Els primers cap�tols descriuen les indicacions b�siques d'instalaci�, 
i operaci�.

Si es vol experimentar amb l'eina, �s necessari un coneixement una 
mica m�s profund del model conceptual del sistema amb la finalitat de
reproduir, mitjan�ant la configuraci�, les condicions del sistema 
simulat i interpretar correctament els resultats obtinguts.

Si el que es vol �s ampliar o modificar l'eina, per a cada m�dul, 
s'explica la seva implementaci� i es donen indicacions molt detallades 
de com fer-ne modificacions que s'integrin en el disseny modular
del sistema.

El present manual es basa en la mem�ria del projecte de final de 
carrera de David Garc�a Garz�n anomenat {\em Bioscena: Simulaci� 
d'un sistema biol�gic evolutiu amb interacci� entre els individuus}.
La finalitat dels dos textos es diferent. La mem�ria del projecte
explicava el que s'havia fet, com i perqu�.
Aquest text ha esta reestructurat, actualitzat i adaptat per que
serveixi de guia als usuaris i futurs mantenidors del sistema.

%%%%%%%%%%%%%%%%%%%%%%%%%%%%%%%%%%%%%%%%%%%%%%%%%%%%%%%%%%%%%%%%%%%%%%
%\section{Estructura del manual}
%%%%%%%%%%%%%%%%%%%%%%%%%%%%%%%%%%%%%%%%%%%%%%%%%%%%%%%%%%%%%%%%%%%%%%



