%%%%%%%%%%%%%%%%%%%%%%%%%%%%%%%%%%%%%%%%%%%%%%%%%%%%%%%%%%%%%%%%%%%%%
%
%

%%%%%%%%%%%%%%%%%%%%%%%%%%%%%%%%%%%%%%%%%%%%%%%%%%%%%%%%%%%%%%%%%%%%%
\section{Visi� global del disseny}
%%%%%%%%%%%%%%%%%%%%%%%%%%%%%%%%%%%%%%%%%%%%%%%%%%%%%%%%%%%%%%%%%%%%%

%%%%%%%%%%%%%%%%%%%%%%%%%%%%%%%%%%%%%%%%%%%%%%%%%%%%%%%%%%%%%%%%%%%%%
\subsection{Criteris de disseny i metodologia}
%%%%%%%%%%%%%%%%%%%%%%%%%%%%%%%%%%%%%%%%%%%%%%%%%%%%%%%%%%%%%%%%%%%%%

%%%%%%%%%%%%%%%%%%%%%%%%%%%%%%%%%%%%%%%%%%%%%%%%%%%%%%%%%%%%%%%%%%%%%
\subsection{Disseny modular}
%%%%%%%%%%%%%%%%%%%%%%%%%%%%%%%%%%%%%%%%%%%%%%%%%%%%%%%%%%%%%%%%%%%%%
\label{sec:moduls}

L'aplicatiu que es vol dissenyar consta de diversos elements
principals, cadascun, amb funcions determinades dintre del sistema.

% TODO: Esquemeta del disseny modular global

\begin{description}
\item[Biosistema:] �s l'objecte coordinador de la resta
d'elements. Les seves funcions s�n:
\begin{enumerate}
\item Demanar instruccions als organismes.
\item Controlar quantes operacions pot executar cada organisme
d'una tacada.
\item Fer les operacions de modificaci� i consulta sobre
la resta d'elements del sistema, necess�ries per executar les
instruccions prove�des per la comunitat d'organismes.
\item Demanar a la comunitat d'organismes per un nou organisme
quan l'organisme actual ja ha executat les seves instruccions.
\item Accionar els agents configuradors que varien el medi
al llarg del temps segons uns parametres.
\end{enumerate}

\item[Topologia:] Determina la geometria del medi on viuen els
organismes. Les seves funcions s�n:
\begin{enumerate}
\item Associar un identificador a cada posici� dins del substrat
\item Establir interconexions entre les parceles de substrat
\item Determinar moviments, direccions, camins... i tota l'operativa
que t� a veure amb la geometria (topologia) del medi segons aquestes
interconexions.
\item Proporcionar l'acc�s, mitjan�ant l'identificador de posici�, a
les propietats del medi en aquesta posici�.
\end{enumerate}

\item[Substrat:] Determina les propietats del medi en una posici�
donada. Les seves funcions s�n:
\begin{enumerate}
\item Determinar si la posici� l'ocupa un organisme i, en cas afirmatiu,
quin �s l'organisme ocupant.
\item Contenir els nutrients lliures al medi.
%TODO: Posar altres funcions del Substrat si apareixen
\end{enumerate}

\item[Agents Configuradors:] Determinen l'evoluci� de certs
par�metres (posici�, composici�, probabilitat, estacionalitat...) que
intervenen  en les propietats dels elements del sistema al llarg del
temps. Les seves aplicacions s�n:
\begin{enumerate}
\item Afegeixir o eliminar nutrients lliures dins del medi.
\item Modificar els par�metres del substrat.
\item Generar expont�neament organismes.
%% TODO: Posar altres funcions dels Agents si apareixen
%% TODO: Considerar les que hi ha perque t'has lluit
\end{enumerate}

\item[Comunitat:] Representa al conjunt d'organismes que viuen al
bi�top. La comunitat compleix amb les seg�ents funcions.
\begin{enumerate}
\item Associar un identificador a cada organisme dintre de la comunitat
\item Afegir-ne o extreure'n organismes.
\item Controlar la informaci� referent a l'organisme que el relaciona
amb el seu entorn, com ara, la posici�, el grup reproductiu al que
pertany... (Informaci� externa de l'organisme)
\item Proporcionar l'acc�s, mitjan�ant l'identificador d'organisme,
tant a la informaci� externa com al propi organisme.
%TODO: Posar altres funcions de la Comunitat si apareixen
\end{enumerate}

\item[Organismes:] Representen als individuus que viuen al
biosistema. Contenen la informaci� gen�tica i les estructures
internes que els fan anar.
\begin{enumerate}
\item Oferir instruccions al biosistema del que volen fer.
\item Proporcionar al biosistema acc�s al fenotip.
\item Proporcionar al biosistema operacions per modificar el seu estat intern.
\item Generar organismes nous.
%TODO: Posar altres funcions del Organismes si apareixen
\end{enumerate}

\item[Taxonomista:] Reuneix un conjunt d'eines que permeten fer un
an�lisis de l'evoluci� d'un grup reproductiu (poblaci�) i de les
interaccions amb els altres grups. Aquest seguiment requereix que el
taxonomista estigui intimament lligat al funcionament del biosistema.
\begin{enumerate}
\item Detectar l'aparici� de comportaments sexuals.
\item Mantenir informaci� hist�rica sobre l'aparici� de grups
reproductius.
\item Mantenir un cens per edats de la poblaci� de cada grup
reproductiu i a cada edat.
\item Mantenir un llistat sobre la dieta de cada grup reproductiu.
\item Mantenir un llistat dels agressors de cada grup reproductiu.
%TODO: Posar altres funcions del Organismes si apareixen
\end{enumerate}

\end{description}

%%%%%%%%%%%%%%%%%%%%%%%%%%%%%%%%%%%%%%%%%%%%%%%%%%%%%%%%%%%%%%%%%%%%%
\newpage
