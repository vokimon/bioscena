% Time Log
%

%%%%%%%%%%%%%%%%%%%%%%%%%%%%%%%%%%%%%%%%%%%%%%%%%%%%%%%%%%%%%%%%%%%%%%
\chapter{Compilaci�
\label{sec:Compilacio}}
%%%%%%%%%%%%%%%%%%%%%%%%%%%%%%%%%%%%%%%%%%%%%%%%%%%%%%%%%%%%%%%%%%%%%%

Aquest cap�tol explica com compilar el codi font per generar un 
executable. �s possible que a la distribuci� ja vingui un executable
per a la seva plataforma i no calgui generar-ho.

%%%%%%%%%%%%%%%%%%%%%%%%%%%%%%%%%%%%%%%%%%%%%%%%%%%%%%%%%%%%%%%%%%%%%%
\section{Generaci� de n�meros de compilaci�}
%%%%%%%%%%%%%%%%%%%%%%%%%%%%%%%%%%%%%%%%%%%%%%%%%%%%%%%%%%%%%%%%%%%%%%

Per generar n�meros de compilaci�, cal el programa {\tt buildnum},
el codi font del qual est� disponible a la pagina 
\url{http://www.salleurl.edu/~is04069/Codders}.

Si no es vol fer servir aquest programa, cal treure les crides que 
s'hi fan al {\tt makefile} o als fitxers de projecte.

%%%%%%%%%%%%%%%%%%%%%%%%%%%%%%%%%%%%%%%%%%%%%%%%%%%%%%%%%%%%%%%%%%%%%%
\section{GCC/DJGPP}
%%%%%%%%%%%%%%%%%%%%%%%%%%%%%%%%%%%%%%%%%%%%%%%%%%%%%%%%%%%%%%%%%%%%%%

Aquest compilador i, sobretot, la seva versi� per a DOS, el DJGPP, ha 
sigut el m�s testejat. De cara a compilar cal una versi� superior a la 
2.7 i, per fer-ho sense problemes cal la 2.95.
Cal assegurar-se que estan instalades les llibreries est�ndars de C++.

Per aquest compilador, es proveeix un fitxer {\tt makefile}. 
El procediment �s el seg�ent:
\begin{enumerate}
\item Adaptar les seg�ents variables del makefile als noms
dels executables al sistema on es compila:
\begin{description}
\item [CC] Nom del compilador de C. ({\tt gcc}, {\tt cc}, {\tt egcs}...)
\item [CPPC] Nom del compilador C++. ({\tt c++}, {\tt cxx}...)
\item [RM] Comanda per esborrar arxius
\item [EXEC] Nom del executable generat
\end{description}
\item Si cal, comenta les crides a 'buildnum' que es fan als makefiles
\item Si es vol optimitzar per una maquina en concret, cal afegir '-march=ix86' a la variable CFLAGS del makefile; on x={3,4,5,6}
\item Executar la comanda 'make'
\end{enumerate}

El makefile t� en compte de forma autom�tica qualsevol arxiu {\tt .CPP} 
i {\tt .C} que s'afegeixi al directori. Si es modifiquen les
depend�ncies cal executar la comanda {\tt make dep} per actualitzar
el fitxer {\tt .depend}. Si aquest fitxer no existeix, cal crear-ho
manualment ({\tt echo >.depend}) per que el makefile no doni
errors i despr�s executar {\tt make dep}.

%%%%%%%%%%%%%%%%%%%%%%%%%%%%%%%%%%%%%%%%%%%%%%%%%%%%%%%%%%%%%%%%%%%%%%
\subsection{Problemes}
%%%%%%%%%%%%%%%%%%%%%%%%%%%%%%%%%%%%%%%%%%%%%%%%%%%%%%%%%%%%%%%%%%%%%%

\begin{itemize}
\item Generals:
\begin{itemize}
\item Si es fa servir una versi� del compilador o de les llibreries 
de C++ inferior a la 2.8, d�na problemes a una l�nia de la funci�
{\tt CCariotip::ocupaCromosomes} del fitxer {\tt Cariotip.cpp}.
A sota de la mateixa l�nia hi ha un fragment de codi substitutori
per a la l�nia del error, amb indicacions de com arreglar-ho.
\end{itemize}
\item Sistemes Unix:
\begin{itemize}
\item Els arxius de text del comprimit estan en format MS-DOS. Per aix�, cal 
treure els salts de carro adicionals que impideixen la compilaci�.
La forma m�s r�pida de fer-ho �s fer servir l'opci� {\tt -a} quan es
descomprimeix el comprimit amb la utilitat {\tt unzip}.
\end{itemize}
\end{itemize}

%%%%%%%%%%%%%%%%%%%%%%%%%%%%%%%%%%%%%%%%%%%%%%%%%%%%%%%%%%%%%%%%%%%%%%
\section{Microsoft Visual C++}
%%%%%%%%%%%%%%%%%%%%%%%%%%%%%%%%%%%%%%%%%%%%%%%%%%%%%%%%%%%%%%%%%%%%%%

Dins de l'arxiu comprimit hi ha un projecte per Microsoft Visual C++
anomenat {\tt Bioscena.dsw}.

%%%%%%%%%%%%%%%%%%%%%%%%%%%%%%%%%%%%%%%%%%%%%%%%%%%%%%%%%%%%%%%%%%%%%%
\subsection{Problemes amb versions antigues}
%%%%%%%%%%%%%%%%%%%%%%%%%%%%%%%%%%%%%%%%%%%%%%%%%%%%%%%%%%%%%%%%%%%%%%

Amb versions antigues del compilador es donen un parell de problemes:

\begin{itemize}
\item
Hi ha problemes amb la biblioteca est�ndard C++ que proporciona 
l'entorn degut a conflictes entre la implementaci� dels templates que 
fa el compilador i la forma de fer-los servir en la biblioteca, 
implementada per HP.
Es pot compilar l'eina si es modifiquen els fonts de la biblioteca, 
ficant entre comentaris les funcions que donen errors.
\item
No suporten massa b� punters a funcions membres.
El codi que fa servir punters a funcions membres, tot i que compila, 
genera un assembler no massa correcte i d�na errors d'execuci�.
\end{itemize}

%%%%%%%%%%%%%%%%%%%%%%%%%%%%%%%%%%%%%%%%%%%%%%%%%%%%%%%%%%%%%%%%%%%%%%
\section{Compilant amb altres compiladors}
%%%%%%%%%%%%%%%%%%%%%%%%%%%%%%%%%%%%%%%%%%%%%%%%%%%%%%%%%%%%%%%%%%%%%%

Per compilar amb altres compiladors, es requereix com a m�nim un 
compilador que suporti templates i punters a funcions membres i que
tingui implementada de forma suficientment amplia la llibreria 
est�ndar C++ segons s'especifica en l'est�ndard ISO/ANSI \cite{ISOCPP}. 
Concretament es fan servir les cap�aleres est�ndard 
{\tt <algorithm>}, {\tt <functional>}, 
{\tt <list>}, {\tt <vector>}, {\tt <map>}, {\tt <deque>}, {\tt <string>}, 
{\tt <iostream>}, {\tt <fstream>}, {\tt <iomanip>} i {\tt <strstream>}.


%%%%%%%%%%%%%%%%%%%%%%%%%%%%%%%%%%%%%%%%%%%%%%%%%%%%%%%%%%%%%%%%%%%%%%
