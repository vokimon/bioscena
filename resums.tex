
%%%%%%%%%%%%%%%%%%%%%%%%%%%%%%%%%%%%%%%%%%%%%%%%%%%%%%%%%%%%%%%%%%%%%%
\section{Abstract}
%%%%%%%%%%%%%%%%%%%%%%%%%%%%%%%%%%%%%%%%%%%%%%%%%%%%%%%%%%%%%%%%%%%%%%

Aquest projecte planteja la simulaci� d'un sistema biol�gic
natural evolutiu amb interacci� entre els organismes dins
d'un medi de variabilitat controlada.
Es tracta de reproduir comportaments naturals o l�gics en
individuus mitjan�ant el proc�s evolutiu.
Es vol estudiar si el fet d'acostar un algorisme gen�tic al
proc�s real natural tenint m�s en compte aspectes biol�gics i
naturals dona una major adaptabilitat a un medi variable.

La part pr�ctica consisteix en la implementaci� d'un prototip que
permeti a l'usuari veure les relacions que s'estableixen entre els
individuus al llarg del proc�s evolutiu.

\newpage

%%%%%%%%%%%%%%%%%%%%%%%%%%%%%%%%%%%%%%%%%%%%%%%%%%%%%%%%%%%%%%%%%%%%%%
\section{Resum}
%%%%%%%%%%%%%%%%%%%%%%%%%%%%%%%%%%%%%%%%%%%%%%%%%%%%%%%%%%%%%%%%%%%%%%

L'objectiu del present treball de fi de carrera �s implementar una
eina d'estudi ampliable, d'aplicaci� als camps de la biologia i la
vida artificial que permeti a un usuari configurar, analitzar e
intervindre en la din�mica d'un sistema biol�gic evolutiu simulat amb
interacci� entre organismes i entre cada organisme i el medi
variable.

Per aix�, primer s'estudiaran els processos naturals (evolutius,
etol�gics i ecol�gics), i es triaran aquells que siguin m�s
susceptibles de �sser implementats i que afegeixin m�s realisme i
generalitat al model. Hi haur� processos que s'implementaran
directament al model, d'altres que s'espera que emergeixin. Tot i
aix�, cal que considerem tamb� els processos emergents de cara a fer
el model per fer possible que els organismes els adoptin i detectar
quan i com ho fan.

L'aplicaci� proveir� eines de configuraci� i intervenci� perque
l'usuari pugui controlar la forma en que varia el medi i, aix�, poder
contrastar-ho amb els resultats obtinguts.

Tamb� s'implementaran eines d'an�lisis per tal de que es puguin
detectar els fen�mens que en l'estudi previ dels processos naturals
es considerin interesants.

El sistema ha de ser prou flexible per permetre l'experimentaci� amb
configuracions prou variades. A m�s, cal donar a l'usuari programador
l'espai necessari per modificar algun aspecte concret del model o
ampliar les opcions donades, tot modificant el codi font.
