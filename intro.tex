
\chapter{Introducci�}

\section{Introducci� al projecte}

\section{Objectius}



L'objectiu del present treball de fi de carrera �s implementar una
eina educativa d'aplicaci� als camps de la biologia i la vida artificial
que permeti a un usuari configurar, analitzar e intervindre en la 
din�mica d'un sistema biol�gic evolutiu simulat amb interacci�
entre organismes i entre cada organisme i el medi variable.

Les eines d'analisis divideixen la comunitat en poblacions, i analitzen les
diferents interaccions que s'hi produeixen.
Les eines de configuraci� i intervenci� serveixen per controlar la forma
en que canvia aquest medi, de forma que, els resultats obtinguts amb les
eines d'analisis siguin confrontables i els usuaris puguin extreure
conclusions.

Els organismes de la simulaci� han de fer front a un medi variable, amb
canvis ca�tics o peri�dics que es poden produir freq�entment al llarg de
la vida d'un organisme o de forma progressiva en el decurs de diverses
generacions. Amb la finalitat de que la comunitat tingui capacitat de
reaccionar davant de tots aquests canvis, s'incorporen, als mecanismes
evolutius que intervenen en la simulaci�, algunes caracter�stiques que
es donen en els entorns evolutius naturals i que, cl�ssicament, no
s'incorporen en els entorns evolutius computacionals

[Potser aixo ja no es objectiu sino implementacio?] 
Les caracter�stiques naturals introdu�des a l'entorn evolutiu s�n:

- Mecanismes d'expressi� g�nica (transcripci�, maduraci� i
traducci�): utils per implementar les altres caracteristiques

- Regulaci� sobre l'expressi� g�nica: Control dels gens que es
transcriuen segons la presencia o no d'alguns factors. Ajuda a
adaptar-se, sense la utilitzacio d'un sistema cognitiu, a canvis en el
medi tan esporadics que el proc�s evolutiu no els soporti.

- Promotor/teminador, zones no codificadores i longitud de cromosoma
variable: Permet solucions obertes no parametritzades (Nombre i longitud
variable pels gens)

- Parametres del proces evolutiu codificats parcialment al genotip: Ens
permetra tenir uns parametres optimitzats per a cada situacio concreta.

- Genotips no haploides i al�lels (Encara he de considerar si
implementar-ho): Mantenen la variabilitat genetica de la descendencia
augmentant aixi la capacitat de canvi i adaptaci�.

- Cariotip multicromos�mic (Encara ho he de considerar, segurament no ho
posi)

\section{Contingut de la mem�ria}

