
%%%%%%%%%%%%%%%%%%%%%%%%%%%%%%%%%%%%%%%%%%%%%%%%%%%%%%%%%%%%%%%%%%%%%%
\section{Abstract}
%%%%%%%%%%%%%%%%%%%%%%%%%%%%%%%%%%%%%%%%%%%%%%%%%%%%%%%%%%%%%%%%%%%%%%

Aquest projecte planteja la simulaci� d'un sistema biol�gic
natural evolutiu amb interacci� entre els organismes dins
d'un medi de variabilitat controlada.

Es tracta de reproduir comportaments naturals o l�gics 
fent evolucionar individus no cognitius.
Es vol estudiar si el fet d'acostar aquests processos evolutius
al proc�s real que es d�na a la natura, tenint en compte m�s 
aspectes biol�gics, d�na una major adaptabilitat a un medi variable.

La part pr�ctica consisteix en la implementaci� d'un prototip que
permeti a l'usuari veure el comportament dels organismes y les 
relacions que s'estableixen entre ells al llarg del proc�s evolutiu.

\newpage

%%%%%%%%%%%%%%%%%%%%%%%%%%%%%%%%%%%%%%%%%%%%%%%%%%%%%%%%%%%%%%%%%%%%%%
\section{Resum}
%%%%%%%%%%%%%%%%%%%%%%%%%%%%%%%%%%%%%%%%%%%%%%%%%%%%%%%%%%%%%%%%%%%%%%

L'objectiu del present treball de fi de carrera �s implementar una
eina ampliable d'experimentaci� pels camps de la biologia i la
vida artificial. Aquesta eina simular� un sistema biol�gic evolutiu 
amb interacci� entre organismes i entre cada organisme i el medi. 
Ha de permetre a un usuari configurar el sistema, intervenir en la 
seva din�mica i oferir eines d'an�lisi per obtenir conclusions.

Tot i que s'intentar� fer un sistema obert que pugui, en el futur,
adaptar-se a molts tipus de sistemes, en aquest primer prototip hem
implementat els organismes amb un sistema de control que simula
els mecanismes de control sobre l'expressi� g�nica que es donen a la 
natura.

Per aix�, primer s'estudiaran els processos naturals (evolutius,
etol�gics i ecol�gics) prou interessants per introduir-los en el 
sistema. Es triaran dos tipus de processos: aquells que, 
d'implementar-los, afegirien realisme al model, i, aquells que 
s'expera observar en el comportament del biosistema de cara a fer
una primera an�lisi.

L'aplicaci� proveir� eines de configuraci� i intervenci� perqu�
l'usuari pugui controlar la forma en qu� varia el medi i, aix�, poder
contrastar-ho amb els resultats obtinguts.

Tamb� s'implementaran eines d'an�lisi per tal de que es puguin
detectar els fen�mens que es considerin interessants en l'estudi previ 
dels processos naturals.

El sistema ha de ser prou flexible per permetre l'experimentaci� amb
configuracions prou variades. A m�s, cal donar a l'usuari programador
l'espai necessari per modificar algun aspecte concret del model o
ampliar les opcions donades, tot modificant el codi font.

