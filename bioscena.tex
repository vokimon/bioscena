%----------------------------------------------------------
%  Aixo es una plantilla basica de latex on es vol tenir
%   el format principal d'un document i totes aquelles coses
%   que normalment utilitzo en latex.

%  Data inici creacio: 19-11-97
%----------------------------------------------------------
\documentclass[11pt,titlepage]{report}
\usepackage[latin1]{inputenc}
\usepackage[catalan]{babel}
\usepackage{layout} % Per visualitzar marges
\usepackage{graphicx}
\usepackage{makeidx}
\usepackage{verbatim}
\usepackage{amssymb}
\usepackage{hyperref} % Per PDFTEX
%\usepackage[dvips]{hyperref} % Per DVIPS
\def\boldindex#1{\textbf{\hyperpage{#1}}}

\hypersetup{backref,hyperindex,colorlinks}

\include{psfig}

\makeindex
\newlength{\defbaselineskip}
\setlength{\defbaselineskip}{\baselineskip}
\newcommand{\setlinespacing}[1]%
           {\setlength{\baselineskip}{#1 \defbaselineskip}}
\newcommand{\doublespacing}{\setlength{\baselineskip}%
                           {2.0 \defbaselineskip}}
\newcommand{\singlespacing}{\setlength{\baselineskip}{\defbaselineskip}}

\oddsidemargin 0 cm
\evensidemargin 0 cm
\topmargin 0 cm
\voffset 0 mm
\headsep 1 cm
\textheight 8 in
\textwidth 6.4 in
\footskip 2 cm
\parskip 2 mm
\columnsep 1 cm

\hyphenation{}

\def\baselinestretch{1}
\setlinespacing{1.5}

\makeindex

\begin{document}

%\DeclareGraphicsExtensions{.eps,.jpg,.gif,.png,.bmp}

%-----------------------------------------------------
%    Titol del document
%-----------------------------------------------------
\title{Bioscena\\Simulaci� d'un sistema biol�gic evolutiu amb interacci� entre els individuus}
\author{
David Garcia\\
\href{mailto:vokimon@jet.es}{vokimon@jet.es}\\
%\includegraphics{04069.jpg}
}
\date{\today}

\maketitle
\newpage

%-----------------------------------------------------
%   Index
%-----------------------------------------------------
\tableofcontents
\newpage
\listoffigures
\newpage
\listoftables
\newpage
% La linia de sota serveix per visualitzar l'efecte dels parametres
% de pagina. Cal descomentar (o afegir) la linia \usepackage{layout}
% a la capcalera
%\layout*

\setlinespacing{1.66}

%%%%%%%%%%%%%%%%%%%%%%%%%%%%%%%%%%%%%%%%%%%%%%%%%%%%%%%%%%%%%%%%%%%%%%
\section{Abstract}
%%%%%%%%%%%%%%%%%%%%%%%%%%%%%%%%%%%%%%%%%%%%%%%%%%%%%%%%%%%%%%%%%%%%%%

Aquest projecte planteja la simulaci� d'un sistema biol�gic
natural evolutiu amb interacci� entre els organismes dins
d'un medi de variabilitat controlada.
Es tracta de reproduir comportaments naturals o l�gics en
individuus mitjan�ant el proc�s evolutiu.
Es vol estudiar si el fet d'acostar un algorisme gen�tic al
proc�s real natural tenint m�s en compte aspectes biol�gics i
naturals dona una major adaptabilitat a un medi variable.

La part pr�ctica consisteix en la implementaci� d'un prototip que
permeti a l'usuari veure les relacions que s'estableixen entre els
individuus al llarg del proc�s evolutiu.

\newpage

%%%%%%%%%%%%%%%%%%%%%%%%%%%%%%%%%%%%%%%%%%%%%%%%%%%%%%%%%%%%%%%%%%%%%%
\section{Resum}
%%%%%%%%%%%%%%%%%%%%%%%%%%%%%%%%%%%%%%%%%%%%%%%%%%%%%%%%%%%%%%%%%%%%%%

L'objectiu del present treball de fi de carrera �s implementar una
eina d'estudi ampliable, d'aplicaci� als camps de la biologia i la
vida artificial que permeti a un usuari configurar, analitzar e
intervindre en la din�mica d'un sistema biol�gic evolutiu simulat amb
interacci� entre organismes i entre cada organisme i el medi
variable.

Per aix�, primer s'estudiaran els processos naturals (evolutius,
etol�gics i ecol�gics), i es triaran aquells que siguin m�s
susceptibles de �sser implementats i que afegeixin m�s realisme i
generalitat al model. Hi haur� processos que s'implementaran
directament al model, d'altres que s'espera que emergeixin. Tot i
aix�, cal que considerem tamb� els processos emergents de cara a fer
el model per fer possible que els organismes els adoptin i detectar
quan i com ho fan.

L'aplicaci� proveir� eines de configuraci� i intervenci� perque
l'usuari pugui controlar la forma en que varia el medi i, aix�, poder
contrastar-ho amb els resultats obtinguts.

Tamb� s'implementaran eines d'an�lisis per tal de que es puguin
detectar els fen�mens que en l'estudi previ dels processos naturals
es considerin interesants.

El sistema ha de ser prou flexible per permetre l'experimentaci� amb
configuracions prou variades. A m�s, cal donar a l'usuari programador
l'espai necessari per modificar algun aspecte concret del model o
ampliar les opcions donades, tot modificant el codi font.

%\part{Introducci�}

\chapter{Introducci� al projecte}

\section{�mbit del projecte}

Aquest projecte �s dins de l'�mbit de la vida artificial
(Artificial Life), disciplina que recull coneixements
d'inform�tica i biologia per recrear fen�mens biol�gics en un
entorn artificial.

Aquesta disciplina va sorgir de cara a la biologia com a camp de
proves alternatiu a la vida real, per�, les aplicacions van
extenent-se, per exemple, aportant noves perspectives a l'analisis,
simulaci� i predicci� de sistemes complexos no biol�gics o a
algorismes per la resoluci� de problemes als sistemes inform�tics.

Les aplicacions de la vida artificial tenen un seguit de
caracter�stiques m�s o menys comunes. Les principals s�n:

\begin{description}
\item[M�tode sint�tic:]
En comptes d'analitzar la vida, sintetitzem artificialment sistemes
amb un comportament similar partint de premises que han sigut 
obtingudes de l'analisis dels sistemes reals.

\item[Construccio Botton-Up:]
Es parteix de unitats petites, definint les interaccions locals, en
comptes de partir del comportament global desitjat i anar perfilant
com han de ser els components. El comportament global del sistema �s
un comportament que no estava expl�citament dissenyat.

\item[Emerg�ncia:]
�s el fet de que apareixin aquests comportaments globals no dissenyats
expl�citament a partir de l'entramat complex d'interacions simples.

\item[No lligat als sistemes reals:]
Donat el seu caracter sint�tic la vida no es limita a la vida
coneguda sin� que prova de extreure propietats generals per a
qualsevol forma de vida possible.

\item[Paral�lelisme impl�cit:]
La complexitat dels sistemes vius �s deguda, en part, a que els
diferents processos es donen en paral�lel. Per fer aix� en els
sistemes de vida artificial, tot i que no sempre sigui possible
dedicar un processador a cada proc�s, s� que caldria fer servir
t�cniques de temps compartit, que, macrosc�picament, els diferents
processos donin la impressi� d'executar-se paral�lelament.
\end{description}

% TODO: Cap �mbit m�s pel projecte? De banda de l'Alife


\section{Antecedents}

Dels treballs que s'han fet sobre vida artificial, cal destacar,
com a precedents, per la seva relaci� amb aquest projecte, els
seg�ents:

\subsection{Latent Energy Environment (LEE)}

El projecte que aquesta mem�ria presenta parteix de les experi�ncies
que Richard K. Beleg i Filippo Menezer van fer sobre el model Latent
Energy Environments (LEE)\index{LEE!Latent Energy Environments}.
%% TODO: Referencies de LEE


LEE �s un model que es basa en l'evoluci� de sistemes cognitius
\index{sistemes cognitius} (xarxes neuronals) que interaccionen amb
un medi qu�mic de complexitat controlada. La complexitat del medi es
controla amb la complexitat d'un model metab�lic: Per obtenir energia
�til, a l'organisme no li basta amb introduir els nutrients dins seu,
sin� que li cal juntar-los amb altres nutrients que produiran
mitjan�ant una reacci�, un balan� energ�tic, positiu o negatiu, i uns
productes segons s'indica a una taula de reaccions. 

Bioscena partir� d'aquesta idea de l'entorn metab�lic de complexitat 
controlada. La figura \ref{fig:modelMetabolicLEE}
mostra esquem�ticament el sistema metab�lic implementat a LEE. La
comparativa es pot fer amb la figura \ref{fig:modelMetabolic} que
representa el model metab�lic implementat finalment en aquest
projecte.

\begin{figure}[h]
\centering
\begin{tabular}{@{\extracolsep{.1 cm}}c@{\extracolsep{.1 cm}}c@{\extracolsep{.1 cm}}c@{\extracolsep{-.1 cm}}c@{\extracolsep{.1 cm}}c@{\extracolsep{.1 cm}}c@{\extracolsep{.1 cm}}c@{\extracolsep{.1 cm}}}
    \begin{tabular}{|c|}
            \hline Medi \\\hline
    \end{tabular}
    &
    \begin{tabular}{c}
        \\ $\rightarrow$
        \\ ingesti�
    \end{tabular}
    &
    \begin{tabular}{|c|}
    \hline Pap\\ \hline
    \end{tabular}
    &
    \begin{tabular}{c}
        \\ $\rightarrow$
        \\ reactius
    \end{tabular}
    &
    \begin{tabular}{|c|}
    \hline Reaccions\\ metab�liques\\ \hline
    \end{tabular}
    &
    \begin{tabular}{cc}
        \begin{tabular}{c}
            energia \\
            $\leftrightarrow$ \\
        \end{tabular}
        &
        \begin{tabular}{|c|}
            \hline Energia\\ �til\\ \hline
        \end{tabular}
        \\ \\
        \begin{tabular}{c}
            \\ $\rightarrow$
            \\ productes
        \end{tabular}
        &
        \begin{tabular}{|c|}
            \hline Medi \\\hline
        \end{tabular}
    \end{tabular}
\end{tabular}
\caption{Model metab�lic a LEE} \label{fig:modelMetabolicLEE}
\end{figure}

%% TODO: Refer�ncia al Plasticity de LEE
La plasticitat\index{plasticitat}, segons Beleg i Menezer
\cite{LEEPlasticity1994}, 
�s la capacitat que t� l'organisme de modificar el seu fenotip durant la
seva vida. Als LEE aquesta plasticitat els hi dona la xarxa neuronal,
un sistema cognitiu. Per�, trobem a la natura, que la resta
d'organismes que no estan provistos d'un sistema cognitiu
d'aprenentatge tamb� presenten una plasticitat equivalent que els hi
permet adaptar-se a tot un seguit de canvis.

El que es vol comprovar com a primera experi�ncia amb el sistema �s 
si els sistemes control sobre l'expressi� g�nica que es donen a 
natura poden ocupar el paper de les xarxes neuronals en el sentit de 
donar plasticitat als organismes no cognitius. La funci� d'aquest 
control sobre l'expressi� g�nica �s decidir, en cada moment, quin 
gens s'expressen i quins no, tot tenint en compte factors del medi i 
de l'estat intern.

A la natura, aquest control sobre l'expressi� g�nica, prov� d'una
certa complexitat en els mecanismes gen�tics. S'intentar�n
implementar els m�xims d'aquests mecanismes per tal de determinar
l'influ�ncia de cadascun sobre la plasticitat de l'organisme, 
activant uns i desactivant altres.

\subsection{Projecte `Tierra'}

De cara a plantejar com controlar els gens s'ens presenta una altra
inc�gnita pr�via: Als LEE, el codi gen�tic representa les
interconexions i els pesos inicials de les neurones; qu� �s el que
codifiquen els gens al sistema que proposem?

Com a punt de partida per trobar una codificaci� edient pels gens
s'ha pres com a refer�ncia el projecte `Tierra' que encap�ala Thomas
S. Ray \cite{RayThomas93}. Thomas S. Ray �s un bi�leg que va traduir
els seus coneixements de gen�tica i d'ecologia a un medi inform�tic
on petites porcions de codi competien per la mem�ria i el temps del
processador amb la finalitat de replicar els seus gens. Va obtenir
comportaments molt elaborats al llarg de la evoluci�: parasitisme,
simbiosis, curses de bra�os, comportaments sexuals....
% TODO: Que ous es el nom, Thomas o Ray!!??

A `Tierra', el genoma estava composat per un seguit d'instruccions
que s'executaven c�clicament. Per modelar el comportament dels
organismes de Bioscena, es fan servir, tamb�, r�fegues d'instruccions
que formen el genotip de forma semblant a com funcionava `Tierra'.

Per�, a `Tierra' el prop�sit dels fragments de codi era el de
replicar-se el m�s eficientment possible dintre d'un medi que estava
composat nom�s pels fragments de codi en compet�ncia. Aqu�, en canvi,
les instruccions hauran d'implementar funcions per interaccionar amb
un medi que simular� a un de natural, ali� a la seva estructura 
interna. Per aix�, es far� sevir una estructura que far� de pont 
entre el medi i el codi gen�tic. D'ara en endavant, aquesta 
estructura ser� la que anomenarem fenotip\index{fenotip}, tot
i tenir present que el fenotip no el forma nom�s aquesta estructura
sin� que tamb� hi forma part el comportament global observat i les
altres variables d'estat internes de l'organisme. Simplement el diem
fenotip perque la seva �nica funci� �s, la de fer de fenotip.

El fenotip controla:
\begin{itemize}
\item Quin grup d'instruccions s'executa.
\item De quina forma ho fan (par�metres o valors).
\end{itemize}

Tantmateix, el fenotip �s modificat per l'execuci� d'alguns gens.
Aquestes modificacions poden dependre nom�s de la instrucci�
executada, per�, sovint, la modificaci� tamb� depen de:
\begin{enumerate}
\item L'estat de l'entorn de l'organisme
\item L'estat intern de l'organisme
\end{enumerate}

\begin{figure}[h]
    \centering
    \fbox{
    \fbox{genotip} $\leftrightarrow$ \fbox{fenotip} $\leftrightarrow$
    \fbox{Condicions d'entorn i internes} }
    % TODO: Imatge de l'estructura genotip -> fenotip <- medi
    \label{fig:introFenotip}
    \caption{Graf de depend�ncies entre el fenotip i els altres elements de l'organisme.}
\end{figure}

La introducci� del fenotip, juntament amb els mecanismes de control 
de l'expressi� g�nica, marquen una altra difer�ncia amb el model 
`Tierra' que t� a veure amb el control de fluxe.

A `Tierra' el genotip �s un conjunt d'instruccions que s'executen
seq�encialment de forma c�clica. T� un control de fluxe basat en
salts a patrons que supleix la debilitat davant de les mutacions
que tindria un sistema de salts basats en adreces.

A Bioscena el genotip est� composat per diversos conjunts 
d'instruccions. Un conjunt d'instruccions (gen) necessita que es 
donin certes condicions al fenotip per poder ser executat.

\section{Objectius del projecte}

Els objectius principals del projecte s�n:

\begin{enumerate}
\item Fer un estudi dels processos naturals (evolutius, ecol�gics i
etol�gics) que es donen a la natura. Per un costat, cal triar els que
afegeixirien m�s realisme i generalitat al model en cas
d'implementar-se. Per un altre costat, cal triar aquells fen�mens que
es donen en la natura que no cal implementar directament, sin�, que
s'espera que puguin emergir. L'esfor� de l'analisis ha d'anar adre�at
a modelar aquelles caracter�stiques no cognitives de caire general, i
sobretot, dels organismes unicel�lulars, tot i que no cal descartar
que emergeixin estructures pluricel�lulars o cognitives, com s'ha dit
abans.

\item Fer un estudi bibliogr�fic dels processos anteriors que ja
hagin estat implementats. En quines condicions concretes s'han fet i
quins resultats han donat.

\item Dissenyar i implementar el model amb tota la informaci�
recollida, possibilitant, sense dirigir-la, l'aparici� dels fen�mens
emergents esperats, i l'adaptaci� dels organismes a variacions que es
poden donar durant la vida d'un organisme o al llarg de generacions.
El model tindr� diversos elements principals:

\begin{itemize}
\item El biosistema, que coordina la resta d'elements.

\item El bi�top, que �s el medi on viuran els organismes. Cal que
permeti configuracions molt diverses per donar possibilitats
d'experimentaci�. Per modificar el bi�top farem servir, sobretot, els
agents externs, que s�n modificadors del bi�top dels quals es pot 
programar el seu efecte al llarg de temps. \footnote{La majoria de la
bibliografia de vida artificial anomena agents al que aqu� anomenem
organismes. Cal que quedi clar que els agents, en aquest projecte, no
s�n els individuus que poblen el m�n, sin� una eina de configuraci�.}

\item La comunitat, que controla el conjunt d'organismes presents al
medi i la informaci� que no �s intr�nseca a ells, com ara la seva
posici� en el medi i la poblaci� a la que pertany. Els organismes
proveeixen al medi les accions que volen realitzar quan li s�n
requerides, i un conjunt d'operacions que el medi pot fer sobre ells.

\end{itemize}

\item Afegir al model les eines d'an�lisis han de poder donar
informaci� �til del que est� passant al biosistema: Per aix�, cal
dividir la comunitat en poblacions i analitzar les interaccions entre
elles. Cal posar una cura especial en detectar l'aparici� de fen�mens
emergents. Les eines de configuraci� i intervenci� han de servir per
controlar la forma en que canvia aquest medi, de forma que, els
resultats obtinguts amb les eines d'analisis siguin confrontables i
els usuaris puguin extreure conclusions.
\end{enumerate}

%% TODO: Incloure una figura float amb l'esquema global

A aquests objectius es poden sumar altres objectius adicionals que es
cobriran de forma secund�ria.

\begin{enumerate}
\item Els elements del sistema han d'estar d�bilment acoblats per tal
de poder, en un futur, intercanviar-los per uns altres de forma
individual amb un m�nim d'efectes laterals.
\item El sistema ha de poder abocar-se totalment o parcial a
disc per tornar-se a restaurar, posteriorment. Aix� faria possible
execucions molt m�s llargues i obtenir poblacions m�s evolucionades.
\item Ha de ser possible canviar alguns aspectes de la configuraci�
del bi�top o intervindre en la poblaci� amb extraccions,
introduccions, clonacions... sobre la marxa.
\item Establir uns procediments de modificaci� per tal de que
l'usuari-programador no necessiti comprendre les interioritats de tot
el sistema per fer una modificaci� puntual.
\end{enumerate}

Els organismes de la simulaci� han de fer front a un medi variable,
amb canvis ca�tics o peri�dics que es poden produir de dos formes:
freq�entment al llarg de la vida d'un organisme o de forma progressiva 
en el decurs de diverses generacions. Amb la finalitat de que la 
comunitat tingui capacitat de reaccionar davant de tots aquests canvis, 
s'incorporen, als mecanismes evolutius que intervenen en la simulaci�, 
algunes caracter�stiques que es donen en els entorns evolutius naturals 
i que, cl�ssicament, no s'incorporen en els entorns evolutius
computacionals.

Les caracter�stiques naturals que s'estudiar� introdu�r a l'entorn
evolutiu s�n:

\begin{description}
\item[Mecanismes d'expressi� g�nica (transcripci�, maduraci� i
traducci�):] �tils per implementar les altres caracter�stiques

\item[Regulaci� sobre l'expressi� g�nica:]
Control dels gens que es transcriuen segons la pres�ncia o no
d'alguns factors. Ajuda a adaptar-se, sense l'utilitzaci� d'un
sistema cognitiu, a canvis en el medi tan espor�dics que el proc�s
evolutiu no els soporti.

\item[Promotor/teminador, zones no codificadores i longitud de
cromosoma variable:] Permet solucions obertes no parametritzades
(Nombre i longitud variable pels gens)

\item[Par�metres del proc�s evolutiu codificats parcialment al
genotip:]
Ens permetr� tenir uns par�metres optimitzats per a cada situaci�
concreta.

\item[Cariotip multicromos�mic:] Divideix la dotaci� g�nica en
subunitats d'alt nivell que permeten traspasar de cop divers material
gen�tic entre organismes i possibilita les mutacions cariot�piques
\footnote{Per diferenciant-les de les g�niques o cromos�miques. Els
tipus de mutacions possibles es van comentant m�s endavant en la mem�ria.}.
Amdues coses poden emergir en un creuament.
%% TODO: Revisa esto ultimo que es una fantochada

\item[Genotips no haplonts i al�lels (Encara he de considerar si
implementar-ho):] Mantenen la variabilitat gen�tica de la
descend�ncia augmentant aix� la capacitat de canvi i adaptaci�.

\end{description}

Per a la simulaci� es suposa, per a totes les decisions de disseny on
sigui necessari fer-ho, que els organismes s�n unicel�lulars. Aquesta
suposici� no vol dir que els resultats siguin aplicables nom�s a
aquest tipus d'organisme, per�, facilita el disseny, donat que el
model d'un organisme pluricel�lular �s molt m�s complexe. A m�s, hi
ha la possibilitat, si el medi f�s suficientment ampli, de que
l'evoluci� port�s als organismes unicel�lulars a formar estructures
cooperatives m�s grans similars als organismes pluricel�lulars.

\section{Contingut de la mem�ria}

TODO: Quan l'estructura de la mem�ria estigui m�s o menys establerta ho poses

%\part{Part te�rica}
%%%%%%%%%%%%%%%%%%%%%%%%%%%%%%%%%%%%%%%%%%%%%%%%%%%%%%%%%%%%%%%%%%%%%
%% Time Log:
%% 19990709 - 18:00-21:30
%% 19990710 - 01:00-02:50
%% 19990710 - 12:30-15:35
%% 19990710 - 16:30-15:35
%% 19990712 - 6h
%% 19990713 - 6h
%% 19990714 - 6h
%% 19990716 - 00:00-06:00
%% 19990716 - 17:00-24:00
%% 19990717 - 00:00-09:00
%%%%%%%%%%%%%%%%%%%%%%%%%%%%%%%%%%%%%%%%%%%%%%%%%%%%%%%%%%%%%%%%%%%%%
%% Change Log:
%% 19990708 VoK - Creat
%% 19990709 VoK - Reestruccturacio del text a dintre dels apartats
%% 19990716 VoK - Afegits els sumaris de coses que es poden ficar al projecte

%%%%%%%%%%%%%%%%%%%%%%%%%%%%%%%%%%%%%%%%%%%%%%%%%%%%%%%%%%%%%%%%%%%%%
\chapter{Coneixements te�rics sobre biologia}
%%%%%%%%%%%%%%%%%%%%%%%%%%%%%%%%%%%%%%%%%%%%%%%%%%%%%%%%%%%%%%%%%%%%%


%%%%%%%%%%%%%%%%%%%%%%%%%%%%%%%%%%%%%%%%%%%%%%%%%%%%%%%%%%%%%%%%%%%%%
\section{Introducci�}
%%%%%%%%%%%%%%%%%%%%%%%%%%%%%%%%%%%%%%%%%%%%%%%%%%%%%%%%%%%%%%%%%%%%%

Aquest cap�tol introdueix alguns conceptes sobre procediments que es donen
a la natura que necessitem coneixer per, posteriorment, aplicar-los en dos 
�mbits. Per un costat, necessitem coneixer els procediments gen�tics que 
es donen a la natura per adaptar-los als algorismes gen�tics. 
Per un altre costat, necessitem coneixements sobre etologia i ecologia
per interpretar i analitzar els resultats de la simulaci�.

%%%%%%%%%%%%%%%%%%%%%%%%%%%%%%%%%%%%%%%%%%%%%%%%%%%%%%%%%%%%%%%%%%%%%
\section{Gen�tica mendeliana}
%%%%%%%%%%%%%%%%%%%%%%%%%%%%%%%%%%%%%%%%%%%%%%%%%%%%%%%%%%%%%%%%%%%%%

%%%%%%%%%%%%%%%%%%%%%%%%%%%%%%%%%%%%%%%%%%%%%%%%%%%%%%%%%%%%%%%%%%%%%
\subsection{Gen, Al�lel, Genotip i Fenotip}
%%%%%%%%%%%%%%%%%%%%%%%%%%%%%%%%%%%%%%%%%%%%%%%%%%%%%%%%%%%%%%%%%%%%%

Els primers estudis gen�tics de la hist�ria [MENDEL 1866] van pendre una 
perspectiva externa als individuus per estudiar l'her�ncia (als organismes
que es reprodueixen sexualment). �s a dir, a partir dels car�cters observables 
d'individuus de diferents generacions, van intentar deduir els mecanismes o 
factors que intervenen en l'her�ncia.

Aquest estudis, i les seves ampliacions posteriors, van madurar un seguit
de conceptes que han perdurat fins avui en dia. S'expliquen a continuaci�:

El car�cter observable en el que ens fixem l'anomenem {\bf fenotip}, 
per exemple, el color dels ulls.
Un {\bf gen} �s cadascun dels factors hereditaris que controlen aquest fenotip. 
El normal �s que siguin diversos gens els que controlin un fenotip 
el conjunt dels quals formen el seu {\bf genotip}.
Per exemple, imaginem que el color dels ulls el controla un genotip format per dos gens.
Generalment, la correspond�ncia entre genotip i fenotip no �s directa, 
perque en el fenotip pot intervindre l'entorn.
El conjunt de tots els gens que t� un organisme (no pas considerant un sol fenotip)
�s el {\bf genoma}.

Un {\bf al�lel}, �s cadascuna de les alternatives (o ``valors'') que pot adoptar un gen. 
Per exemple, imaginem que tenim les alternatives A, B, C y D.
Dos al�lels es consideren diferents nom�s si, en algun cas, el fet 
de tenir un o l'altre afecta al fenotip obtingut.
Dos gens poden tenir els mateixos al�lels possibles. Si aix� passa i, a m�s,
l'efecte d'un �s equivalent al de l'altre, diem que s�n {\bf gens hom�legs}
El genotip �s {\bf homozigot} si els seus gens nom�s presenten un al�lel per cada conjunt de gens hom�legs (Ra�a pura).
En canvi, el genotip �s {\bf heterozigot} si els seus gens presenten al�lels diversos (H�brid).

%%%%%%%%%%%%%%%%%%%%%%%%%%%%%%%%%%%%%%%%%%%%%%%%%%%%%%%%%%%%%%%%%%%%%
\subsection{Expressi� dels al�lels al fenotip}
%%%%%%%%%%%%%%%%%%%%%%%%%%%%%%%%%%%%%%%%%%%%%%%%%%%%%%%%%%%%%%%%%%%%%

Un individuu homozig�tic presenta el car�cter fenot�pic representatiu de l'al�lel.
Aquest car�cter s'en diu el fenotip homozig�tic per aquest al�lel.

En la natura, sovint, els genotips homozig�tics estan associats, directament o indirecta,
a fenotips negatius. Aix� �s degut a que els individuus homozig�tics donen molt 
poca variabilitat gen�tica. Si una poblaci� acaba convertint-se en homozigotica,
per exemple, per excesiva endog�mia, aquest gen es queda estancat i no d�na 
variabilitat. Si, per adaptar-se a un canvi en l'entorn cal canviar aquest gen,
una poblaci� homozig�tica tindr� menys capacitat de resposta perqu� dependr� de les mutacions. 
En els gens que no interesa aquest estancament la pr�pia dotaci� gen�tica s'autopenalitza. 
Si una poblaci� no castiga la homozigosis, t� m�s probabilitat de tornar-s'en.

A vegades la millor forma de penalitzacio s�n els gens letals. 
Un genotip letal �s aquella combinaci� d'al�lels que no es presenten mai 
donat que els individuus no sobrepassen l'estat embrionari.

Quan el genotip �s heterozigot ens podem trobar les seg�ents relacions entre els al�lels
dels gens hom�legs, segons la forma d'expressar-se en el fenotip:
\begin{itemize}
\item	{\bf Domin�ncia - Recessivitat:} Un al�lel (dominant) s'imposa sobre l'altre (recessiu), 
		i, el fenotip resultant �s el mateix que el d'un homozigot amb l'al�lel dominant.
\item	{\bf Her�ncia intermitja:} El fenotip no �s l'equivalent a cap dels dos fenotips 
		homozigots sin�	que �s un fenotip intermig.
\item	{\bf Codomin�ncia:} Es presenten els fenotips dels dos al�lels a la vegada.
\item	{\bf Superdomin�ncia o heterosis:} La pres�ncia de l'al�lel recessiu refor�a el fenotip
		de al�lel dominant m�s que si f�s un homozigot.
\end{itemize}


%%%%%%%%%%%%%%%%%%%%%%%%%%%%%%%%%%%%%%%%%%%%%%%%%%%%%%%%%%%%%%%%%%%%%
\subsection{Fenotips de distribuci� cont�nua}
%%%%%%%%%%%%%%%%%%%%%%%%%%%%%%%%%%%%%%%%%%%%%%%%%%%%%%%%%%%%%%%%%%%%%

Alguns car�cters de l'individu, com per exemple l'altura o la intelig�ncia, 
no presenten unes alternatives fenot�piques tan discont�nues sin� que
s�n m�s cont�nues.

El car�cter pot dependre de diversos gens {\bf poligen} amb la qual cosa 
es produeix una distribuci� normal en el car�cter. 
Com m�s gens s'hi impliquin, m�s graus discontinus hi haur� a la distribuci�. 
Per exemple:

\begin{tabular}{lc}
N�mero de gens implicats	&	Distribuci� del caracter \\
1 (haploide)&	1:1 \\
2 	&	1:2:1 \\
4	&	1:2:3:4:3:2:1 \\
6	&	1:6:15:20:15:6:1
\end{tabular}

%% TODO: Pegar un parell de gr�fics

Arriba un moment que hi intervenen tants gens que el gradient no �s
identificable. A m�s, considerant el fet de que l'entorn afecta al
fenotip, tenim que un individuu amb genotip AABBBB pot ser fenotipicament
m�s semblant a un homozigot A que un individuu amb genotip AAABBB.

%%%%%%%%%%%%%%%%%%%%%%%%%%%%%%%%%%%%%%%%%%%%%%%%%%%%%%%%%%%%%%%%%%%%%
\subsection{Interaccions entre gens no hom�legs}
%%%%%%%%%%%%%%%%%%%%%%%%%%%%%%%%%%%%%%%%%%%%%%%%%%%%%%%%%%%%%%%%%%%%%

Aix� com diversos gens hom�legs poden contribuir al fenotip d'un car�cter,
gens no hom�legs tamb� poden contribuir-hi, per� en aquest cas la influ�ncia
dels gens no es equivalent (per la definici� anterior de gens hom�legs).

Una de les interaccions entre gens no hom�legs �s la {\bf epist�sia}: un gen 
(epist�tic) controla l'activaci� o desactivaci� d'un altre (hipost�tic).

%%%%%%%%%%%%%%%%%%%%%%%%%%%%%%%%%%%%%%%%%%%%%%%%%%%%%%%%%%%%%%%%%%%%%
\subsection{Mutaci� g�nica}
%%%%%%%%%%%%%%%%%%%%%%%%%%%%%%%%%%%%%%%%%%%%%%%%%%%%%%%%%%%%%%%%%%%%%

El concepte d'her�ncia tal i com el definia Mendel, s'enfrontava amb les noves
idees de Darwin sobre l'evoluci� [DARWIN 1859]. El concepte que tenia Darwin
sobre l'evoluci� �s que als individuus es produien petits canvis (mutacions) 
dels quals la natura escollia els m�s apropiats per a l'entorn.
En el concepte de Mendel sobre l'her�ncia, no es creaven al�lels nous, les generacions
posteriors tenen simplement una recombinaci� dels al�lels de les generacions anteriors.
De Vries, un dels redescobridors dels postulats de Mendel, va introduir el concepte
de mutaci� g�nica [DEVRIES 1900] que ve a reconciliar els dos corrents i donar
una explicaci� a l'aparici� de al�lels diferents dintre d'un gen.

Una mutaci� genica o puntual �s l'aparici� sobtada d'una nova alternativa (al�lel)
per a un gen. 

La mutaci� (tan si �s g�nica com si �s una altra de les que veurem m�s endavant) �s un 
fen�men aleatori que es dona amb una determinada freq��ncia que sol ser molt baixa. 
La {\bf freq��ncia de mutaci�} �s una probabilitat que es mesura per a un gen donat, 
i durant una generaci�.

La probabilitat de mutaci� g�nica �s mant� constant, tot i que diferent per a cada
gen. Aix� s'explica per la diferent longitud de gens, o per la quantitat de mutacions
que no impliquen un canvi d'al�lel (com es veu a la secci� \ref{TeoBioMolecular}).

Als organismes pluricel�lulars, les mutacions poden ser {\bf som�tiques} o {\bf germinals}.
Una mutaci� som�tica, nom�s afecta a una cel�lula i a totes les que en deriven.
Per exemple, els tumors, les pigues... s�n mutacions som�tiques.
Una mutaci� germinal es produeix a les c�l�lules del teixit que donaran lloc a les gametes.
En conseq��ncia, aquesta mutaci� afectara, no nom�s a la c�l�lula i a les que en derivin
sin� que tamb� a la descend�ncia.

%%%%%%%%%%%%%%%%%%%%%%%%%%%%%%%%%%%%%%%%%%%%%%%%%%%%%%%%%%%%%%%%%%%%%
\subsection{Sumari de conceptes aplicables al projecte}
%%%%%%%%%%%%%%%%%%%%%%%%%%%%%%%%%%%%%%%%%%%%%%%%%%%%%%%%%%%%%%%%%%%%%

L'aproximaci� mendeliana als mecanismes de l'her�ncia, tot i ser una primera
aproximaci�, ja ens dona alguns conceptes que no s�n presents a la implementaci�
cl�ssica dels algorismes gen�tics.

El primer concepte �s el de gens hom�legs. 
A l'algorisme gen�tic cl�ssic, els gens no t�nen hom�legs a menys que 
es considerin aix� a la funci� d'evaluaci� i, generalment, no es fa. 
En el cas de tenir hom�legs, caldria resoldre el problema de 
l'heterozigosi la qual cosa implica un cost computacional superior
que �s sovint in�til en els problemes que tenen una funci� d'avaluaci� 
constant. Aquest cost computacional, com a m�nim, �s el que implica 
duplicar la informaci� continguda al cromosoma.

En canvi, als problemes on la funci� d'avaluaci� varia al llarg del 
temps, com �s el cas d'un entorn biol�gic, �s positiu guardar-se 
aquesta variabilitat en forma d'heterozigosis. 
Un genotip que no tingui gens hom�legs t� els mateixos desavantatges 
que s'han comentat abans per un genotip homozigot, per�, amb el 
desavantatge afegit de que mai pot arribar a ser heterozigot.

El concepte de mutaci� g�nica, �s el concepte de mutaci� puntual
dels algorismes gen�tics cl�ssics. 
El que s'aporta de nou �s el concepte de probabilitat de
mutaci� variable segons el gen, i la difer�ncia entre mutaci�
som�tica i germinal de cara a transmetre les mutacions a la 
descend�ncia.
Als apartats seg�ents anirem modificant i enriquint el concepte 
de mutaci� a mida que es vagi desgavellant, en aquest cap�tol, 
la natura dels gens.

%% TODO: Cercar Referencia d'algu que faci servir heterozigosis (sense diplonts?)

De cara a l'an�lisis dels resultats, pot ser interesant detectar si 
hi ha algun mecanisme que afavoreixi els genotips heterozigots i la 
pres�ncia de fenotips continus. 
Tamb� pot interesar detectar si es donen casos d'epist�sia, tot i 
que, com que els mecanismes d'epist�sia poden ser molt complexos, 
caldria limitar a alguns casos de baix nivell.


%%%%%%%%%%%%%%%%%%%%%%%%%%%%%%%%%%%%%%%%%%%%%%%%%%%%%%%%%%%%%%%%%%%%%
\section{Teoria cromos�mica}
%%%%%%%%%%%%%%%%%%%%%%%%%%%%%%%%%%%%%%%%%%%%%%%%%%%%%%%%%%%%%%%%%%%%%

%%%%%%%%%%%%%%%%%%%%%%%%%%%%%%%%%%%%%%%%%%%%%%%%%%%%%%%%%%%%%%%%%%%%%
\subsection{Els cromosomes i l'her�ncia}
%%%%%%%%%%%%%%%%%%%%%%%%%%%%%%%%%%%%%%%%%%%%%%%%%%%%%%%%%%%%%%%%%%%%%

Els estudis de Morgan et al. varen demostrar que els gens no eren 
quelcom independent sin� que estaven en estructures superiors 
formant els {\bf cromosomes}. 

El gens es troben localitzats en un punt concret del cromosoma 
({\bf locus}). Quan reconvinem el material gen�tic de dos
progenitors, els gens a locus m�s propers dins d'un cromosoma
tenen molta m�s probabilitat d'heretar-se conjuntament.
Aix� implica, els gens que controlen cada car�cter no s'hereten de
forma tan independent com formula la tercera llei de Mendel.
La tercera llei de Mendel diu que els caracters s'hereten
de forma independent. Aix� �s veritat, no en els caracters sin� en
els gens i sempre que els gens tinguin un locus suficientment
allunyats o en cromosomes diferents.

Anomenem {\bf cariotip} al conjunt de cromosomes que t� una esp�cie 
(en quant a nombre i morfologia).

Els cromosomes s�n hom�legs si t�nen la mateixa morfologia
i cont�nen gens hom�legs als mateixos {\em locus}.

Un cariotip �s {\bf diplont} si est� format per n parelles de
cromosomes hom�legs. �s a dir, cada cromosoma t� un altre d'hom�leg,
de tal forma que hi ha dos dotacions cromos�miques hom�logues.
Generalment s�n organismes que es reprodueixen sexualment, 
i, de cada parella d'hom�legs, cada progenitor n'ha aportat un.

Quan nom�s hi ha una dotaci� cromos�mica el cariotip es diu que
�s {\bf haplont}. Si n'hi ha tres dotacions hom�logues, {\bf triplont},
si n'hi ha quatre, {\bf tetraplont} i, si n'hi ha m�s, {\bf poliplont}.

%% TODO: Estructura d'un cromosoma: Centr�mer. crom�tides...

%%%%%%%%%%%%%%%%%%%%%%%%%%%%%%%%%%%%%%%%%%%%%%%%%%%%%%%%%%%%%%%%%%%%%
\subsection{Mitosi i meiosi}
%%%%%%%%%%%%%%%%%%%%%%%%%%%%%%%%%%%%%%%%%%%%%%%%%%%%%%%%%%%%%%%%%%%%%

Les cel�lules seg�eixen 

%% TODO!!! Mitosis
%% TODO!!! Gemaci� vs. Bipartici�

%% TODO!!! Meiosis i la recombinaci�

%%%%%%%%%%%%%%%%%%%%%%%%%%%%%%%%%%%%%%%%%%%%%%%%%%%%%%%%%%%%%%%%%%%%%
\subsection{Reproducci� i sexualitat. Cicles biol�gics}
%%%%%%%%%%%%%%%%%%%%%%%%%%%%%%%%%%%%%%%%%%%%%%%%%%%%%%%%%%%%%%%%%%%%%

Sexualitat i reproducci� s�n dos procesos amb origen evolutiu i funci� diferent.
Si b� l'objectiu de la reproduci� era obtindre individuus que siguin c�pies
gen�tiques dels seus progenitors, l'objectiu de la sexualitat �s el de la 
recombinaci� gen�tica per provar generar mes variabilitat gen�tica.

Alguns organismes unicel�lulars, per exemple, tenen comportaments sexuals no lligats 
a la reproducci� com ara la conjugaci� que consisteix en l'intercanvi simple de 
material gen�tic sense que es produeixi cap nou individuu.

%% TODO: La conjugacio no es algo mes? No sera altra cosa el que dius?

Dels procesos sexuals en resulta una gama d'individuus diferents molt m�s amplia 
del que en resulta amb processos exclusivament asexuals. 
Aix� dona m�s agilitat al proc�s evolutiu, sobretot de cara a variacions en el medi. 
Permet que, si un organisme adquireix per mutaci� una caracter�stica positiva, 
aquesta caracter�stica es propagui per la poblaci� sense necessitat de que 
hi hagi una substituci� dr�stica de la descend�ncia sense mutaci� per la 
descend�ncia amb mutaci�.

Als organismes que es reprodueixen sexualment, a la fecundaci�, sempre intervenen
dos c�l�lules haplonts (gametes), una de cada progenitor, que es junten per formar 
un zigot diplont.
Pot passar que la meiosi es produeixi abans de la fecundaci�, i que els individuus
madurs siguin diplonts ({\bf cicle diplont}) o que la meiosis es produeixi despr�s 
de la fecundaci� amb la qual cosa els individuus son haplonts ({\bf cicle haplont}).

Alguns organismes alternen generacions haplonts amb les diplonts ({\bf cicle haplodiplont}.
Una generaci� haplont produeix les gametes que intervenen a la fecundaci�, formant un
zigot que dona un individuu diplont. Aquest crea c�l�lules esporog�nees (diplonts) que 
per meiosi dona lloc a meiospores haplonts de les que es torna a formar un individuu haplont.

%% TODO!!! Diagrames dels tres cicles
%% TODO!!! Avantatges de cada cicle

%%%%%%%%%%%%%%%%%%%%%%%%%%%%%%%%%%%%%%%%%%%%%%%%%%%%%%%%%%%%%%%%%%%%%
\subsection{Mutacions cromos�miques}
%%%%%%%%%%%%%%%%%%%%%%%%%%%%%%%%%%%%%%%%%%%%%%%%%%%%%%%%%%%%%%%%%%%%%

El fet de que els gens estiguin dins de l'estrutura cromos�mica dona 
lloc a altres tipus de mutacions que no s�n pas les g�niques. 
T�nen a veure amb la distribuci� dels gens a dins dels cromosomes, 
i, aix� com les mutacions g�niques explicaven l'origen dels diferents
genotips, les mutacion cromos�miques expliquen l'origen dels diferents 
cariotips per cada esp�cie.

Les {\bf mutacions cromos�miques estructurals} s�n aquelles en les que
no intervenen cromosomes integres sino fragments d'aquests.

\begin{itemize}
\item	Es produeix una {\bf defici�ncia o delecci�} quan un cromosoma en perd un segment.
Les causes poden ser una falla en el proc�s de replica o, en el moment del crossover.
%% TODO: Causes de la delecci�
%% TODO: Efectes gaireb� sempre negatius

\item	La {\bf duplicaci�} consisteix en el fet de que un segment cromos�mic es dupliqui,
generalment, en s�rie.

\item	La {\bf translocaci�} �s una mutaci� cromos�mica que consisteix canviar el locus
d'una seq��ncia g�nica. 
%% TODO: DUDA: En el mateix cromosoma o entre homolegs o entre no homolegs??
%% TODO: Rec�proca o no rec�proca
%% TODO: DUDA: Jo em pensava que la translocaci� era com el shift N de AG.
	Si un organisme es reprodueix sexualment, sovint, la translocaci� 
	causa, en els descendents, duplicaci� o deficiencia del gen translocat.

\item	La {\bf inversi�} consisteix en el canvi de sentit d'un segment de cromosoma.

\end{itemize}

De banda de les mutacion estructurals, es donen {\bf mutacions per canvi en el nombre
de cromosomes}. La causa del canvi en el nombre de cromosomes pot ser:

\begin{itemize}
\item	{\bf Fusi� c�ntrica:} Dos cromosomes no hom�legs es fusionen pel seu centr�mer.
\item	{\bf Escissi� c�ntrica:} Un cromosoma se escindeix en dos pel seu centr�mer.
\item	{\bf No disyunci� en la meiosi:} En la meiosi no es reparteixen per igual
	les crom�tides de tal forma que una gameta es queda amb m�s crom�tides del
	normal i l'altre, amb menys.
	Es tracta d'una {\bf euploidia} si �s un canvi en el nombre de 
	dotacions g�niques. Per exemple, que d'un diploide en sorgeixi un triploide.
	Pel contrari, si el que hi ha hagut �s la falta o la duplicaci� d'un sol cromosoma,
	es tracta d'una {\bf aneuploidia}
\end{itemize}

%%%%%%%%%%%%%%%%%%%%%%%%%%%%%%%%%%%%%%%%%%%%%%%%%%%%%%%%%%%%%%%%%%%%%
\subsection{Sumari de conceptes aplicables al projecte}
%%%%%%%%%%%%%%%%%%%%%%%%%%%%%%%%%%%%%%%%%%%%%%%%%%%%%%%%%%%%%%%%%%%%%

La influ�ncia del locus de cada gen en el fet de que dos gens tendeixin
a heretar-se junts, �s un efecte que, als algorismes gen�tics cl�ssics
i amb segons quins problemes, resulta negatiu. A la natura �s un efecte 
positiu per que la posici� dels gens no es fixa sino que evoluciona
juntament amb els individuus i, si dos gens s�n bons si s'hereten junts, 
l'evoluci� tendir� a posar-los junts en el cariotip, i, si f�s bo que
es recombinin de forma equitativa, l'evoluci� tendir� a posar-los separats.

A la simulaci�, de cara a deixar via oberta a diversos comportaments 
sexuals o a organismes asexuals, farem servir la separaci� entre els 
conceptes de sexualitat i reproducci�.
%% TODO: Perque la farem servir?

%% TODO: Refer�ncies a implementacions diplonts dels GA

La introducci� d'organismes diplonts o poliplonts suposaria implementar 
un mecanisme de resoluci� del fenotip heterozig�tic als gens hom�legs. 
Tamb� sorgirien alguns altres problemes, relacionats amb el creuament, 
que es comenten en el seg�ent apartat.


%%%%%%%%%%%%%%%%%%%%%%%%%%%%%%%%%%%%%%%%%%%%%%%%%%%%%%%%%%%%%%%%%%%%%
\section{Gen�tica biomol�lecular}
%%%%%%%%%%%%%%%%%%%%%%%%%%%%%%%%%%%%%%%%%%%%%%%%%%%%%%%%%%%%%%%%%%%%%

%%%%%%%%%%%%%%%%%%%%%%%%%%%%%%%%%%%%%%%%%%%%%%%%%%%%%%%%%%%%%%%%%%%%%
\subsection{Concepte cl�ssic de gen}
%%%%%%%%%%%%%%%%%%%%%%%%%%%%%%%%%%%%%%%%%%%%%%%%%%%%%%%%%%%%%%%%%%%%%

Fins ara, hem considerat un gen com quelcom indivisible. Abans de coneixer a fons
l'estructura mol�lecular de l'ADN, els bi�legs consideraven el gen com a:
\begin{itemize}
\item	{\bf Unitat funcional:} De cara a controlar un car�cter.
\item	{\bf Unitat de reconbinaci�:} Unitat estructural b�sica e indivisible del cromosoma.
\item	{\bf Unitat de mutaci�:} El gen �s el que canvia com un tot.
\end{itemize}

M�s tard, van descobrir que els cromosomes estaven formats per seq��ncies d'ADN.
L'estudi mol�lecular de l'ADN  va donar una idea m�s en detall de com 
s'expressen, com es reconbinen i com muten els gens.

%%%%%%%%%%%%%%%%%%%%%%%%%%%%%%%%%%%%%%%%%%%%%%%%%%%%%%%%%%%%%%%%%%%%%
\subsection{Estructura mol�lecular de l'ADN}
%%%%%%%%%%%%%%%%%%%%%%%%%%%%%%%%%%%%%%%%%%%%%%%%%%%%%%%%%%%%%%%%%%%%%

monocatenario
bicatenario
	circular
	linial


Model de Watson i Crick (Heleicoidal)

direccional, si �s bicatenari cada cadena �s complementaria per�, la direcci� de
transcripci� �s inversa.

Bases al ADN
\begin{tabular}{|ccc|}
\hline
{\bf P�riques}	& \vline & {\bf Pirim�diniques} \\
\hline
\hline
Adenina		& lliga amb & Timina \\
\hline
Guanina		& lliga amb & Citosina \\
\hline
\end{tabular}

Bases al ARN
\begin{tabular}{|ccc|}
\hline
{\bf P�riques}	& \vline & {\bf Pirim�diniques} \\
\hline
\hline
Adenina		& lliga amb & Uracil \\
\hline
Guanina		& lliga amb & Citosina \\
\hline
\end{tabular}



%%%%%%%%%%%%%%%%%%%%%%%%%%%%%%%%%%%%%%%%%%%%%%%%%%%%%%%%%%%%%%%%%%%%%
\subsection{Expressi� g�nica}
%%%%%%%%%%%%%%%%%%%%%%%%%%%%%%%%%%%%%%%%%%%%%%%%%%%%%%%%%%%%%%%%%%%%%

L'expressi� g�nica �s la forma que t�nen els gens, continguts en els cromosomes,
per arribar a afectar al car�cter fenot�pic que controlen. En general, cada gen 
contingut als cromosomes est� associat a la producci� d'una proteina que pot ser 
enzimatica o estructural. A m�s, aquesta associaci� �s linial, com ja veurem, 
aix� vol dir que 

A la natura, l'expressi� g�nica no es fa directament de l'ADN a les proteines
sin� que es fa servir unes cadenes d'ARN com a intermediaries. En conseq��ncia
la expressi� g�nica es fa en tres passos:

%%%%%%%%%%%%%%%%%%%%%%%%%%%%%%%%%%%%%%%%%%%%%%%%%%%%%%%%%%%%%%%%%%%%%
\subsubsection{Transcripci�}
%%%%%%%%%%%%%%%%%%%%%%%%%%%%%%%%%%%%%%%%%%%%%%%%%%%%%%%%%%%%%%%%%%%%%

La transcripci� �s el primer pas de l'expressi� g�nica i l'�nic en el que pren part l'ADN. 
Consisteix en sintetitzar una cadena d'ARN complement�ria a una subseq��ncia d'ADN.
Es divideix en tres fases:

\begin{itemize}
\item	{\bf Fase d'iniciaci�:} 
Primer, l'enzim transcriptor reconeix una seq��ncia de nucle�tids, 
anomenada {\bf regi� promotora}, que �s la que indica el punt de la 
cadena on cal comen�ar una transcripci�. 

\item	{\bf Fase de allargament de cadena:}
Despr�s, a mida que avan�a en la direcci� de transcripci�, va enganxant 
nucle�tids d'ARN complementaris als que es va trobant a la cadena d'ADN.

%% TODO: Comentar el tema de la direccio de transcripci�.

\item	{\bf Fase de finalitzaci�:} 
El proc�s arriba a la seva fi, quan l'enzim arriba a una seq��ncia 
d'ADN determinada, {\bf regi� terminadora}, que indica la seva fi.
\end{itemize}

%%%%%%%%%%%%%%%%%%%%%%%%%%%%%%%%%%%%%%%%%%%%%%%%%%%%%%%%%%%%%%%%%%%%%
\subsubsection{Processat o maduraci�}
%%%%%%%%%%%%%%%%%%%%%%%%%%%%%%%%%%%%%%%%%%%%%%%%%%%%%%%%%%%%%%%%%%%%%

Les cadenes d'ARN$_m$ inmadur, a les c�l�lules eucariotes (amb nucli diferenciat), 
pateixen tot un seguit de transformacions abans de tradu�r-se als ribosomes. 
Aquest processat no es fa als organismes procariotes (nucli dispers) donat que 
ARN$_m$ es tradueix directament, abans, i tot, de acabar-se de transcriure.

Per un costat, la seq��ncia d'ADN que hi ha transcrita al ARN$_m$, no tota cont� 
informaci� �til. Els {\bf exons} s�n els segments que codifiquen informaci�
�til, i els {\bf introns} s�n els segments que no. Una de les transformacions que es fan
en aquesta fase �s eliminar els introns.

%% TODO: Lider i trailer

Una altra transformaci� �s afegir al lider i al trailer un cap i una 
cua respectivament que ajudaran a iniciar i finalitzar la traducci�.

Algunes mol�lecules d'ARN (ARN$_t$, ARN$_r$...) que tamb� es transcriuen 
de l'ADN, pateixen un processat m�s especialitzat. Aquestes mol�l�cules
d'ARN no es fan servir per codificar proteines per�, com es veur�
en la fase seg�ent, tenen un paper important�sim en la s�ntesis de
proteines.


%%%%%%%%%%%%%%%%%%%%%%%%%%%%%%%%%%%%%%%%%%%%%%%%%%%%%%%%%%%%%%%%%%%%%
\subsubsection{Traducci�}
%%%%%%%%%%%%%%%%%%%%%%%%%%%%%%%%%%%%%%%%%%%%%%%%%%%%%%%%%%%%%%%%%%%%%

Durant la traducci�, l'ARN$_m$ madur s'interpreta per anar enganxant 
la seq��ncia d'amino�cids d'una proteina.

Cada tres nucle�tids d'ARN$_t$ formen un cod�. Cada cod� t� un 
amino�cid associat, segons la taula de sota. La concatenaci� dels 
amino�cids segons la seq��ncia de codons �s el que forma la proteina.

Els quatre valors possibles pels tres nucle�tids d'un cod� donen 
$4^3=64$ combinacions. Per�, a la natura, es donen nom�s 21 amino�cids.
Es dedueix, llavors, que {\bf la codificaci� �s redundant} i produeix {\bf sin�nims}.
Un parell o tres de codons s�n {\bf muts} i no tenen traducci�.
Tot i no tenir un amino�cid associat, veurem que els codons muts s�n
molt �tils.

Aquesta taula �s el codi gen�tic que tradueix els codons a amino�cids:

%% TODO: Fig: Taula del codi gen�tic

L'associaci� la fan mol�l�cules d'ARN$_t$ que t�nen a un extrem un anticod�
per lligar-se a un cod�, i, per l'altre extrem un radical amb el qual
es lliguen a l'amino�cid corresponent. Les mol�l�cules d'ARN$_t$ tamb� estan
codificades a l'ADN i, per tant, l'associaci� cod�-amino�cid �s tamb� 
informaci� gen�tica. Sorprenentment, el codi gen�tic �s pr�cticament 
universal, donant-se petites variacions, nom�s a l'ADN mitocondrial. 
La ra� �s que en els organismes m�nimament evolucionats, un canvi
en el codi gen�tic suposaria tants canvis que segurament serien letals,
en canvi en els organismes primigenis, seria possible trobar m�s diversitat
de codis.
\footnote{Aix� d�na una idea de en quin punt de l'evoluci�, les mitocondries
passaren simbi�ticament a formar part dels altres organismes}

%% TODO: AUG (TAC) Sempre el primer, UGA (ACT) (Mut) sempre l'�ltim
%% TODO: Reutilitzaci� d'ARN$_t$


%%%%%%%%%%%%%%%%%%%%%%%%%%%%%%%%%%%%%%%%%%%%%%%%%%%%%%%%%%%%%%%%%%%%%
\subsection{Regulaci� de l'expressi� dels gens}
%%%%%%%%%%%%%%%%%%%%%%%%%%%%%%%%%%%%%%%%%%%%%%%%%%%%%%%%%%%%%%%%%%%%%

%% TODO: Refer�ncia a Jacob i Monod

Mutacions i creuaments s�n els mecanismes d'adaptaci� al medi que
permeten a una poblaci� adaptar-se de generaci� en generaci� als 
canvis graduals en el medi.
Per�, hi ha canvis que s�n tan freq�ents que els ha d'afrontar
l'individuu que els pateix i no es pot esperar a que varii el
genotip de la seva descend�ncia. S�n les respostes que produeix 
l'individuu a les variacions del medi.

Perque un organisme es pugui adaptar a diverses situacions de l'entorn
cal que tingui mecanismes per regular l'expressi� dels seus gens segons
aquestes situacions. Per exemple, si el medi es torna massa �cid cal 
generar enzims que ho compensin, per�, quan es massa b�sic, la producci�
d'aquests enzims, no nom�s �s un gast in�til d'energies sin� que podria
ser contraproduent.

Cal llavors un mecanisme que permeti a l'organisme controlar la seva producci�
enzim�tica. Aquest control no es podria fer efectiu, si l'ARN$_m$ no tingu�s
una vida molt limitada. La vida de l'ARN$_m$ i la vida de la majoria
de proteines enzim�tiques, ve fixada per un comprom�s entre economia en 
la seva producci� i la capacitat de reacci� de l'organisme.

Aquest l�mit en la vida del ARN$_m$ i de les proteines que en genera, 
ens permet una regulaci� basada en el control de la producci� d'ARN$_m$.
Si es deixa de produir, hi haur� un moment en que les proteines que 
genera no hi seran presents. A continuaci� s'explica els factors que 
intervenen en la s�ntesis d'ARN$_m$ per a un gen donat.

%% TODO: Confirmar que ronda els 3-6 minuts i posar-ho amb la referencia.

El gens tenen una {\bf probabilitat transcripci�} que depen de l'afinitat de 
l'enzim transcriptor amb el seu promotor i de la seva repetici� al llarg del genotip. 
Aquesta probabilitat �s inherent al genotip, per�, pot ser modificada amb {\bf regulaci� activa}.
Segon l'efecte de la regulaci� diem que el control que fa �s:
\begin{itemize}
\item	{\bf Control negatiu:} Si es fa mitjan�ant {\bf agents represors}
	que impideixen l'uni� de l'enzim transcriptor amb el promotor
	bloquejant la s�ntesi de ARN$_m$.
\item	{\bf Control positiu:} Si es fa mitjan�ant {\bf agents activadors}
	que es junten amb el promotor per fer-lo m�s af� amb l'enzim 
	transcriptor.
\end{itemize}

Els agents represors/activadors s�n proteines que es sintetitzen tamb�
a partir de l'ADN i que actuen o no, segons les condicions d'ambient.
Quan aquestes condicions ambientals s�n principis actius es diu que s�n:

\begin{itemize}
\item	{\bf Sistemes inducibles:} Si actuen per la ausencia d'un principi actiu.
\item	{\bf Sistemes represibles:} Si actuen per la pres�ncia d'un principi actiu (correpressor o coactivador).
\end{itemize}

%% TODO: No �s massa clar que tamb� siguin aquests noms al control positiu

Per exemple, si el gen que volem regular �s un catabolitzador d'una sust�ncia A, 
l'organisme pot fer servir un control negatiu induible, de tal forma que es 
desactivi quan no hi hagi A, i/o un control positiu represible, perque s'acceleri 
la producci� quan hi hagi A.

%%%%%%%%%%%%%%%%%%%%%%%%%%%%%%%%%%%%%%%%%%%%%%%%%%%%%%%%%%%%%%%%%%%%%
\subsection{Mutaci� i creuament a nivell mol�lecular}
%%%%%%%%%%%%%%%%%%%%%%%%%%%%%%%%%%%%%%%%%%%%%%%%%%%%%%%%%%%%%%%%%%%%%

Com que amb la gen�tica mol�lecular hem vist que la unitat de convinaci� no era el
gen sin� el nucle�tid, cal reformular els conceptes de mutaci� g�nica i cromos�mica.

La mutaci� puntual �s causada per un canvi en les bases dels nucle�tids de l'ADN,
deguda a l'inestabilitat qu�mica del propi ADN o a agents externs.
\begin{itemize}
\item	{\bf Mutaci� per transici� de bases:} Intercanvi per l'altre base del mateix grup (p�riques o pirim�rique)
\item	{\bf Mutaci� per transversi� de bases:} Intercanvi per la base no complementaria de l'altre grup.
\item	No es donen, de forma natural, les mutacions directes entre bases complementaris.
\end{itemize}

%% TODO: Contrastar l'afirmaci� anterior amb un expert (Pepi)
%% TODO: Figura amb el quadre de mutacions de base

Tant la transici� com la transversi�, es donen per la modificaci� 
qu�mica d'una de les bases (tautomeritzaci�) que, en duplicar-se
l'ADN, no es lliga amb la seva base complement�ria sin� amb una
altra. Quedant les bases desajustades. 
Encara cal que passi per uns quants filtres perque la mutaci� es
faci efectiva a la descend�ncia:
\begin{itemize}
\item	Existeixenn mecanismes de reparaci� basats en el fet de que, tot i que una
	base hagi canviat, l'altre pot seguir igual, si no s�n complement�ries, vol dir
	que hi ha hagut una mutaci�. L'enzim corrector modifica una de les dues bases
	per tal de que siguin complement�ries, per�, pot modificar la mutada o la bona.
\item	En duplicar-se la cadena, per una divisi� cel�lular, si encara no s'ha 'corregit� 
	la mutaci�, la meitat de la descend�ncia dur� la mutaci� complerta i l'altra meitat
	el genoma primitiu.
\item	Tant si es repara com si es queda la mutaci� sense reparar, la probabilitat de 
	traspasar-la a un descendent �s del 50\%.
\item	La mutaci� es pot produir a un segment d'ADN no codificant (introns i zones intermitges)
\item	Els codons sin�nims fan que alguns canvis en les bases no impliquin canvi de p�ptid.
%% TODO: Calcular el tant percent, quan tinguis una estona
\item	Alguns canvis de p�ptids als extrons tampoc no s�n significatius pel fenotip.
%% TODO: Refer�ncia a Kimura
\item	En organismes pluricel�lulars, cal que sigui una mutaci� al teixit germinal
	que dona lloc als nous individuus. Si es dona al teixit som�tic, nom�s afecta
	a les c�l�lules filles al mateix organisme.
\end{itemize}

%% TODO: Mutacions no per canvi de base
%% TODO: Mutacions cromos�miques a nivell mol�lecular
%% TODO: Creuament a nivell mol�lecular


%%%%%%%%%%%%%%%%%%%%%%%%%%%%%%%%%%%%%%%%%%%%%%%%%%%%%%%%%%%%%%%%%%%%%
\subsection{Sumari de conceptes aplicables al projecte}
%%%%%%%%%%%%%%%%%%%%%%%%%%%%%%%%%%%%%%%%%%%%%%%%%%%%%%%%%%%%%%%%%%%%%

A nivell mol�lecular, trobem alternatives molt riques al AG cl�ssic.
Algunes de les quals han estat estudiades en la bibliografia.

El gens s�n de longitud variable. El que ho permet s�n les seq��ncies 
promotora i terminadora que els delimiten. Mayer va experimentar
aquest tipus de codificaci�, mitja�ant promotors i terminadors, en 
cromosomes de longitud fixa, variant el nombre, posici� i 
longitud dels gens/par�metres. Raich va trobar ideal una codificaci�
molt semblant (feia servir una longitud en comptes d'una seq��ncia
terminadora) per problemes orientats a disseny que t�nen un nombre
variable de par�metres i que requereixen solucions obertes. 

No tot l'ADN es transcriu, com a m�nim les seq��ncies promotores i 
terminadores, no ho fan. A m�s, tenim els introns que s'eliminen
durant la fase de maduraci�.

%% L'unitat de mutaci�, convinaci�... �s molt m�s petita que un gen.
%% La s�ntesis �s direccional
%% Els gens estan limitats per un promotor i un terminador
%% El transcriptor cerca els promotors
%% No tots els promotors t�nen la mateixa probabilitat de sintetitzar-se
%% El material transcrit no s'expressa tal qual -> Maduraci�
%% Eliminaci� d'introns, afegits per fer-ho executable
%% Codi redundant -> Mutacions sense efecte
%% Codificaci� depenent del genoma
%% Existeixen mecanismes de regulaci� activa que afecten a la s�ntesis per se
%% Enzimes correctores
%% Algunes mutacions directes no permeses
%% Possible codificaci� de les mutaci�ns b�siques
%% Mutacions a segments no codificants
%% No codificadores com a separador de locus pel crossover (3a Mendel)


%%%%%%%%%%%%%%%%%%%%%%%%%%%%%%%%%%%%%%%%%%%%%%%%%%%%%%%%%%%%%%%%%%%%%
\section{Gen�tica de poblacions}
%%%%%%%%%%%%%%%%%%%%%%%%%%%%%%%%%%%%%%%%%%%%%%%%%%%%%%%%%%%%%%%%%%%%%


%%%%%%%%%%%%%%%%%%%%%%%%%%%%%%%%%%%%%%%%%%%%%%%%%%%%%%%%%%%%%%%%%%%%%
\section{Ecologia}
%%%%%%%%%%%%%%%%%%%%%%%%%%%%%%%%%%%%%%%%%%%%%%%%%%%%%%%%%%%%%%%%%%%%%



% Time Log
%

%%%%%%%%%%%%%%%%%%%%%%%%%%%%%%%%%%%%%%%%%%%%%%%%%%%%%%%%%%%%%%%%%%%%%%
\chapter{Tecnologia emprada}
%%%%%%%%%%%%%%%%%%%%%%%%%%%%%%%%%%%%%%%%%%%%%%%%%%%%%%%%%%%%%%%%%%%%%%

%%%%%%%%%%%%%%%%%%%%%%%%%%%%%%%%%%%%%%%%%%%%%%%%%%%%%%%%%%%%%%%%%%%%%%
\section{Orientaci� a objectes}
%%%%%%%%%%%%%%%%%%%%%%%%%%%%%%%%%%%%%%%%%%%%%%%%%%%%%%%%%%%%%%%%%%%%%%

%%%%%%%%%%%%%%%%%%%%%%%%%%%%%%%%%%%%%%%%%%%%%%%%%%%%%%%%%%%%%%%%%%%%%%
\section{Tecniques de disseny}
%%%%%%%%%%%%%%%%%%%%%%%%%%%%%%%%%%%%%%%%%%%%%%%%%%%%%%%%%%%%%%%%%%%%%%

%%%%%%%%%%%%%%%%%%%%%%%%%%%%%%%%%%%%%%%%%%%%%%%%%%%%%%%%%%%%%%%%%%%%%%
\section{Llibreria est�ndard de C++}
%%%%%%%%%%%%%%%%%%%%%%%%%%%%%%%%%%%%%%%%%%%%%%%%%%%%%%%%%%%%%%%%%%%%%%


%%%%%%%%%%%%%%%%%%%%%%%%%%%%%%%%%%%%%%%%%%%%%%%%%%%%%%%%%%%%%%%%%%%%%%

%\part{Part pr�ctica}
\chapter{Disseny de l'aplicatiu}
%%%%%%%%%%%%%%%%%%%%%%%%%%%%%%%%%%%%%%%%%%%%%%%%%%%%%%%%%%%%%%%%%%%%%
%
%

%%%%%%%%%%%%%%%%%%%%%%%%%%%%%%%%%%%%%%%%%%%%%%%%%%%%%%%%%%%%%%%%%%%%%
\section{Metodologia de disseny}
%%%%%%%%%%%%%%%%%%%%%%%%%%%%%%%%%%%%%%%%%%%%%%%%%%%%%%%%%%%%%%%%%%%%%


%%%%%%%%%%%%%%%%%%%%%%%%%%%%%%%%%%%%%%%%%%%%%%%%%%%%%%%%%%%%%%%%%%%%%
\section{Metodologia d'implementaci� i estil de programaci�}
%%%%%%%%%%%%%%%%%%%%%%%%%%%%%%%%%%%%%%%%%%%%%%%%%%%%%%%%%%%%%%%%%%%%%

%%%%%%%%%%%%%%%%%%%%%%%%%%%%%%%%%%%%%%%%%%%%%%%%%%%%%%%%%%%%%%%%%%%%%
\subsection{Registre de canvis}
%%%%%%%%%%%%%%%%%%%%%%%%%%%%%%%%%%%%%%%%%%%%%%%%%%%%%%%%%%%%%%%%%%%%%

Cada arxiu d'implementaci�, porta al inici un registre dels canvis
(Change Log) que s'anat fent al fitxer. Cada entrada d'aquest 
registre porta la data, un indicador de l'autor de la modificaci� i
una breu explicaci� d'una o dos l�nies, suficient per deduir en qu�
consisteix i a quins llocs afecta.

%%%%%%%%%%%%%%%%%%%%%%%%%%%%%%%%%%%%%%%%%%%%%%%%%%%%%%%%%%%%%%%%%%%%%
\subsection{Registre de coses pendents}
%%%%%%%%%%%%%%%%%%%%%%%%%%%%%%%%%%%%%%%%%%%%%%%%%%%%%%%%%%%%%%%%%%%%%
Per tenir const�ncia de les coses que s'han anat deixant pendents,
s'han anat mantenint tres punts de registre de coses a fer (TODO's):
\begin{itemize}
\item A peu de codi, �s a dir al mateix lloc on cal fer el que quedi
pendent. Sempre es tracta d'un comentari que comen�a amb 
\begin{verbatim}
  // TODO:
\end{verbatim}
El comentari ha de ser autoexplicatiu per s� mateix, perque no 
es necessiti veure'l en un context per entendre'l. D'aquesta forma, 
es poden extreure totes les modificacions d'aquest tipus que queden 
pendents executant la comanda:
\begin{verbatim}
  grep -n TODO *.h *.cpp *.c 
\end{verbatim}
\item A l'inici del fitxer d'implementaci�, a continuaci� del {\em 
Change Log}, es posa els canvis pendents que afectan al m�dul en 
general que no es puguin localitzar a cap lloc en concret.
\item En un fitxer a part anomenat {\tt TODO.txt}, s'han anat 
recopilant i actualitzant peri�dicament els canvis pendents que 
persisteixen d'entre els anteriors i alguns d'ambit m�s global o
que afecten a m�duls que encara no s'han constru�t.
\end{itemize}

El control de les coses pendents resulta molt important per no deixar
q�estions deslligades, donada la quantitat de coses que cal tenir
presents durant la implementaci�.

%%%%%%%%%%%%%%%%%%%%%%%%%%%%%%%%%%%%%%%%%%%%%%%%%%%%%%%%%%%%%%%%%%%%%
\subsection{Control de versions}
%%%%%%%%%%%%%%%%%%%%%%%%%%%%%%%%%%%%%%%%%%%%%%%%%%%%%%%%%%%%%%%%%%%%%

Cada cop que es compila, es genera de forma automatitzada una 
entrada a un log de compilacions amb la data i el n�mero de 
compilaci�. Al mateix temps es modifica un arxiu font per tal
de que aquest n�mero de compilaci� i la data estiguin disponibles
per al programa.

Aix� ens permetr� saber, donat un executable, fins a quin punt est�
actualitzat, i amb els {\em Change Logs} quines caracter�stiques
inclou. El registre de compilacions en facilitar�, a m�s, 
l'aproximaci� del temps d'implementaci�.


%%%%%%%%%%%%%%%%%%%%%%%%%%%%%%%%%%%%%%%%%%%%%%%%%%%%%%%%%%%%%%%%%%%%%
\subsection{Fitxers}
%%%%%%%%%%%%%%%%%%%%%%%%%%%%%%%%%%%%%%%%%%%%%%%%%%%%%%%%%%%%%%%%%%%%%

Tot i que la intenci� inicial era mantenir per a cada classe un 
fitxer de prototipus i un altre d'implementaci�, l'�s massiu de les 
classes ha obligat a fusionar algunes classes en el mateix parell de 
fitxers.

Aix� s�, nom�s s'ha fussionat en un fitxer classes molt intimament
lligades com ara subclasses d'una mateixa classe abstracta factoria 
en els casos en els que el codi que aportava cada subclasse era molt 
poc i molt uniforme.

En aquests casos, la classe abstracta factoria t� el seu propi fitxer 
de prototipus de cara a que els seus clients el puguin incloure sense 
que interfereixi l'exist�ncia de les subclasses. La resta s'ha agrupat
en un o m�s.

Generalment el fitxer de la classe abstracta te el nom de la classe
en singular i el de les classes derivades en plural. Fent servir
un exemple t�pic, el fitxer {\tt Persona.h} contindria el prototipus 
de la classe abstracta {\tt CPersona}, i el fitxer {\tt Persones.h} 
podria contenir les especialitzacions de la classe {\tt CClient} i 
{\tt CEmpleat} sempre que aquestes classes no afegissin m�todes
adicionals al protocol p�blic de {\tt CPersona}.

%%%%%%%%%%%%%%%%%%%%%%%%%%%%%%%%%%%%%%%%%%%%%%%%%%%%%%%%%%%%%%%%%%%%%
\subsection{Criteris de nomenclatura d'identificadors}
%%%%%%%%%%%%%%%%%%%%%%%%%%%%%%%%%%%%%%%%%%%%%%%%%%%%%%%%%%%%%%%%%%%%%

En molts, casos s'han adoptat alguns criteris que es fan servir en 
la programaci� d'Smalltalk.

En general, els identificadors que representen diverses paraules hem
adoptat el criteri de fer servir les maj�scules per separar-les en
comptes del s�mbol de subratllat com �s costum entre alguns 
programadors de C. Aix� doncs, farem servir {\tt unIdentificadorLlarg}
en comptes de {\tt un\_identificador\_llarg}.

Els identificadors de les funcions, m�todes de classe (est�tics en 
nomenclatura C) i objectes globals, els hem comen�at preferentment 
per una maj�scula. Tamb� els noms de les classes i els {\tt namespace}'s. 

La primera paraula dels altres identificadors (dades locals o membres, 
funcions membres no est�tiques...) he adoptat el conveni de comen�ar-la
en min�scula.

Tamb� he pres alguns convenis estesos en la programaci� per a
windows. Per exemple:
\begin{itemize}
\item Preposem una {\tt C} maj�scula als identificadors de les classes: {\tt CComunitat}
\item Preposem {\tt m\_} als identificadors de dades membres no est�tiques: {\tt m\_unaVariableMembre}
\item Preposem {\tt s\_} als identificadors de dades membres est�tiques: {\tt s\_unaVariableEst�tica}
\end{itemize}

%%%%%%%%%%%%%%%%%%%%%%%%%%%%%%%%%%%%%%%%%%%%%%%%%%%%%%%%%%%%%%%%%%%%%
\subsection{Proves}
%%%%%%%%%%%%%%%%%%%%%%%%%%%%%%%%%%%%%%%%%%%%%%%%%%%%%%%%%%%%%%%%%%%%%

Cada classe t� una funci� membre est�tica anomenada {\tt ProvaClasse}
on es deixa tota la bateria de proves unit�ries que s'han fet sobre
la classe, per, en cas de modificacions, tornar-les a passar.

Els m�duls que no estiguin encapsulats en classes tamb� tindran una
funci� similar. Generalment per ortogonalitat i per no interferir
en l'espai de noms, la funci� de proves del m�dul es fica a dins
d'un {\tt namespace} sin� hi est� ficat ja tot el m�dul.


%%%%%%%%%%%%%%%%%%%%%%%%%%%%%%%%%%%%%%%%%%%%%%%%%%%%%%%%%%%%%%%%%%%%%
\subsection{Tipus de comentaris}
%%%%%%%%%%%%%%%%%%%%%%%%%%%%%%%%%%%%%%%%%%%%%%%%%%%%%%%%%%%%%%%%%%%%%

\begin{description}
\item [Comentaris de m�dul:] Serveixen per explicar qu� va al m�dul
que encap�alen i si hi ha alguna consideraci� global que fer en 
usar-lo o mantenir-lo.
\item [Comentaris de secci�:] Separen visualment les diferents 
seccions d'un m�dul.
\item [Comentaris d'encap�alament:] Es troben just despr�s de 
l'encap�alament d'una funci� o m�tode i just abans de que s'obrin
els claud�tors de l'implementaci�. Aquests comentaris van adre�ats
als usuaris de la funci� evitant qualsevol menci� als detalls 
d'implementacio. Si cal indicar, precondicions o postcondicions es 
far� aqu�, dedicant una linia a cadascuna que vindr� precedida de les 
part�cules {\tt Pre:} o {\tt Post:}.
\item [Comentaris de manteniment:] Aquests comentaris es troben 
al cos de la funci� o m�tode (entre els claud�tors), parlen de 
detalls d'implementaci� i estan adre�ats als mantenidors.
\end{description}

%%%%%%%%%%%%%%%%%%%%%%%%%%%%%%%%%%%%%%%%%%%%%%%%%%%%%%%%%%%%%%%%%%%%%
\section{Visi� global del disseny}
%%%%%%%%%%%%%%%%%%%%%%%%%%%%%%%%%%%%%%%%%%%%%%%%%%%%%%%%%%%%%%%%%%%%%


%%%%%%%%%%%%%%%%%%%%%%%%%%%%%%%%%%%%%%%%%%%%%%%%%%%%%%%%%%%%%%%%%%%%%
\subsection{Disseny modular}
%%%%%%%%%%%%%%%%%%%%%%%%%%%%%%%%%%%%%%%%%%%%%%%%%%%%%%%%%%%%%%%%%%%%%
\label{sec:moduls}

L'aplicatiu que es vol dissenyar consta de diversos elements
principals, cadascun, amb funcions determinades dintre del sistema.

\unitlength 1mm
\begin{figure}[h]
\begin{picture}(200,30)
TODO: Esquemeta del disseny modular global
\put(0,0){\circle*{4}}
\end{picture}
\caption{Esquema conceptual del sistema}
\label{fig:EsquemaConceptual}
\end{figure}


\begin{description}
\item[Biosistema:] �s l'objecte coordinador de la resta
d'elements. Les seves funcions s�n:
\begin{enumerate}
\item Multiplexar l'execuci� concurrent dels diferents organismes. 
\item Demanar als organismes les instruccions que volen executar.
\item Realitzar les operacions de modificaci� i consulta sobre
la resta d'elements del sistema, necess�ries per executar les
instruccions prove�des per la comunitat d'organismes.
\item Mantenir dintre d'uns m�nims la poblaci� de la comunitat 
introduint nous organismes quan aquesta baixa.
\item Accionar els agents externs encarregats de variar el medi
al llarg del temps.
\end{enumerate}

\item[Topologia:] Determina la geometria del medi on viuen els
organismes. Les seves funcions s�n:
\begin{enumerate}
\item Associar un identificador a cada posici� dins del substrat
\item Establir interconexions entre les parceles de substrat
\item Determinar moviments, direccions, camins... i tota l'operativa
que t� a veure amb la geometria (topologia) del medi segons aquestes
interconexions.
\item Proporcionar l'acc�s, mitjan�ant l'identificador de posici�, a
les propietats del medi en aquesta posici�.
\end{enumerate}

\item[Substrat:] Determina les propietats del medi en una posici�
donada. Les seves funcions s�n:
\begin{enumerate}
\item Determinar si la posici� l'ocupa un organisme i, en cas afirmatiu,
quin �s l'organisme ocupant.
\item Contenir els nutrients lliures al medi.
\item Altres caracter�stiques associades a la localitat que es 
vulguin afegir m�s endavant.
%TODO: Posar altres funcions del Substrat si apareixen
\end{enumerate}

\item[Agents Configuradors:] Determinen l'evoluci� de certs
par�metres (posici�, composici�, probabilitat, estacionalitat...) que
intervenen  en les propietats dels elements del sistema al llarg del
temps. Les seves aplicacions s�n:
\begin{enumerate}
\item Afegir o eliminar nutrients lliures dins del medi.
\item Modificar els par�metres del substrat.
\item Generar expont�neament organismes.
%% TODO: Posar altres funcions dels Agents si apareixen
%% TODO: Considerar les que hi ha perque t'has lluit
\end{enumerate}

\item[Comunitat:] Representa al conjunt d'organismes que viuen al
bi�top. La comunitat compleix amb les seg�ents funcions.
\begin{enumerate}
\item Associar un identificador a cada organisme dintre de la comunitat
\item Afegir-ne o extreure'n organismes.
\item Controlar la informaci� referent a l'organisme que el relaciona
amb el seu entorn, com ara, la posici�, el grup reproductiu al que
pertany... (Informaci� externa de l'organisme)
\item Proporcionar l'acc�s, mitjan�ant l'identificador d'organisme,
tant a la informaci� externa com al propi organisme.
%TODO: Posar altres funcions de la Comunitat si apareixen
\end{enumerate}

\item[Organismes:] Representen als individuus que viuen al
biosistema. Contenen la informaci� gen�tica i les estructures
internes que els fan anar.
\begin{enumerate}
\item Oferir instruccions al biosistema del que volen fer.
\item Proporcionar al biosistema acc�s al fenotip.
\item Proporcionar al biosistema operacions per modificar el seu estat intern.
\item Generar organismes nous.
%TODO: Posar altres funcions del Organismes si apareixen
\end{enumerate}

\item[Taxonomista:] Reuneix un conjunt d'eines que permeten fer un
an�lisis de l'evoluci� d'un grup reproductiu (poblaci�) i de les
interaccions amb els altres grups. Aquest seguiment requereix que el
taxonomista estigui intimament lligat al funcionament del biosistema.
\begin{enumerate}
\item Mantenir informaci� hist�rica sobre l'aparici� de grups
reproductius.
\item Mantenir un cens per edats de la poblaci� de cada grup
reproductiu i a cada edat.
\item Mantenir un llistat sobre la dieta de cada grup reproductiu.
\item Mantenir un llistat dels agressors de cada grup reproductiu.
%TODO: Posar altres funcions del Taxonomista si apareixen
%TODO: Considerar les que hi ha perque t'has lluit
\end{enumerate}

\end{description}

%%%%%%%%%%%%%%%%%%%%%%%%%%%%%%%%%%%%%%%%%%%%%%%%%%%%%%%%%%%%%%%%%%%%%
\newpage

% Time Log
%

%%%%%%%%%%%%%%%%%%%%%%%%%%%%%%%%%%%%%%%%%%%%%%%%%%%%%%%%%%%%%%%%%%%%%%
\section{Eines i ajudes a la implementaci�}
%%%%%%%%%%%%%%%%%%%%%%%%%%%%%%%%%%%%%%%%%%%%%%%%%%%%%%%%%%%%%%%%%%%%%%

En aquest apartat s'expliquen algunes eines que s'han implementat per
tal d'afavorir l'implementaci� de la resta del sistema.

%%%%%%%%%%%%%%%%%%%%%%%%%%%%%%%%%%%%%%%%%%%%%%%%%%%%%%%%%%%%%%%%%%%%%%
\subsection{Dispositius d'entrada i sortida portables}
%%%%%%%%%%%%%%%%%%%%%%%%%%%%%%%%%%%%%%%%%%%%%%%%%%%%%%%%%%%%%%%%%%%%%%

TODO: Dispositius d'entrada i sortida portables

%%%%%%%%%%%%%%%%%%%%%%%%%%%%%%%%%%%%%%%%%%%%%%%%%%%%%%%%%%%%%%%%%%%%%%
\subsection{Funci� de compatibilitat de claus}
%%%%%%%%%%%%%%%%%%%%%%%%%%%%%%%%%%%%%%%%%%%%%%%%%%%%%%%%%%%%%%%%%%%%%%
\label{sec:compatibilitat}

Es tracta d'obtenir una funci� que, a partir de dues claus, en
determini si es compleix que s�n compatibles o no. La compatibilitat
entre claus es fara servir, per exemple, per a la identificaci�
d'organismes (amb l'objectiu de cercar preses, triar la parella,
identificar fills i pares...), identificaci� de nutrients (amb
l'objectiu d'ingerir-los, evitar-los, detectar excrecions ajenes,
controlar els processos metab�lics interns...) i contesa (mecanismes
de depredaci� i defensa). Donat que aquesta funci� �s una de les m�s
utilitzades al sistema, ha de ser molt poc costosa.

Cal que la funci� no tingui en compte la ponderaci� dels bits que
formen la clau i que els tracti tots de la mateixa forma perque no es
converteixi en una optimitzaci� num�rica. A mes, �s desitjable que
aquesta funci� permeti nivells de toler�ncia variables i un cert
indeterminisme.

Necessitem tenir en compte llavors tres elements:
\begin{itemize}
\item El grau de compatibilitat entre les claus.
\item Una toler�ncia quantificable sobre les variacions entre claus.
\item Un element indetermin�stic que permeti resultats diferents amb les mateixes entrades.
\end{itemize}

Si les claus les representem amb dos enters de 32 bits, el grau de
coincid�ncia el podrem obtenir fent-ne la o exclusiva bit a bit i
complementant el resultat. Al n�mero obtingut l'anomenarem
coincid�ncia (C). El nivell de toler�ncia tambe pot ser un enter (T)
que ens vindra donat i l'indeterminisme el pot introduir un altre
enter (R) tret d'una funcio pseudo-aleatoria.

\subsubsection{Opci� 1}

Els uns del n�mero generat pseudo-aleatoriament (R) es 'filtren' per
la coincid�ncia (C) de tal forma ke nomes arribin els uns que
estiguin en una posici� on no hi havia coincid�ncia entre claus. La
toler�ncia (T) indica el n�mero d'uns que admetem com a m�xim per
aceptar les claus com a compatibles.

\begin{equation}
    ComptaUns(R \& \sim C)<T
    \label{eq:tolerancia1}
\end{equation}

El punt negre d'aquest m�tode �s l'alt cost de la funci� ComptaUns,
donat que no �s una operacio nativa a la majoria de m�quines i cal
implementar-la a base de despla�aments i enmascaraments.

La seg�ent gr�fica mostra la probabilitat de que dos claus de 32 bits
siguin compatibles segons el bits que tinguin igual i per diferents
valors de T. La T pot oscilar entre 0 i 32 tot i que veiem que la
distribuci� no pateix variacions apreciables per valors a partir de
24 o potser abans. Podriem molt be limitar-la entre 0 i 15 sense
perdre gaire significat.

\begin{figure}[h]
    \centering
    %\includegraphics[width=6.1 in, keepaspectratio, draft]{compatibilitat1}
    \framebox[6.2 in]{TODO: Posar la gr�fica compatibilitat1}
    \label{fig:compat1}
    \caption{Probabilitat d'encerts segon el nombre d'uns de la coincid�ncia i el nombre d'uns tolerats amb la funci� de compatibilitat n�mero 1}
\end{figure}

La distribuci� sembla ideal pel que volem: Per valors de poca
toler�ncia, la probabilitat �s gaireb� nul�la, per toler�ncies molt
grans ho deixa passar gairebe tot i per a una s�rie de valors
intermitjos on es mantenen tres zones:
\begin{itemize}
\item Una zona de pas incondicional, per les coincid�ncies m�s altes.
\item Una zona intermitja on la probabilitat de pas depen de la coincid�ncia.
\item Una zona de tall incondicional, per les coincid�ncies m�s baixes.
\end{itemize}

\subsubsection{Opci� 2}

Una altra opci� �s fer servir la toler�ncia com una altra m�scara. Un
bit a un a la toler�ncia voldria dir que es tolera que aquest bit
resulti a un despr�s del filtratge. La condici� que determina que dos
claus s�n compatibles quedaria com segueix:

\begin{equation}
    {\tt(R \& \sim C \& \sim T)==0}
    \label{eq:tolerancia2}
\end{equation}

Aqu� s� que T agafa tota la franja dels 32 bits. Per fer la gr�fica i
obtenir un resultat comparable amb l'anterior, s'ha considerat el
n�mero de uns a la T en comptes del seu valor.

\begin{figure}[h]
    \centering
    \includegraphics[width=6.1 in, keepaspectratio, draft]{compatibilitat2}
    \label{fig:compat2}
    \caption{Probabilitat d'encerts segons el nombre d'uns presents a la coincid�ncia i a la toler�ncia amb la funci� de compatibilitat n�mero 2}
\end{figure}

Observem que hem perdut les zones d'aceptaci� i rebuig incondicional
a les toler�ncies intermitges, per�, la funci� �s bastant v�lida pels
objectius donat que �s una aceptaci� no deterministica on la
probabilitat depen de la toler�ncia i de la coincid�ncia, i, a m�s,
hem optimitzat moltissim el cost d'evaluaci�.

Com a caracter�stica afegida, aquesta funci� permet, mitjan�ant la
toler�ncia, un control m�s acurat de quins bits son els que poden no
coincidir. Aquesta peculiaritat pot donar a peu a mecanismes m�s
complexes, que no pas una toler�ncia cega. A m�s, tot i que es t� en
compte la posici� dels bits, no els pondera, com les altres formules
provades.

\subsubsection{Altres opcions desestimades}

Altres funcions de compatibilitat han estat provades i del tot
desestimades pel seu alt cost i/o per la seva poca idoneitat.

Per exemple, es va provar la funci�
\begin{equation}
    ComptaUns(R \& \sim C \& \sim T1) < (T2 \& 0x5)
    \label{eq:tolerancia3}
\end{equation}

per sintetitzar en una f�rmula els dos conceptes de toler�ncia que
hem vist, una toler�ncia que d�na significat a la posici� dels uns i
una altra que permet tolerar globalment un cert nombre d'uns
independentment de la posici�.

Degut als pocs bits (32) amb els que juga i a que hi havia dos punts on es tolera, la funci�, lluny de donar tot el significat que voliem, dona molt poca variaci� amb els par�metres. A m�s, tornem a tenir el problema de la funci� ComptaUns.

\begin{equation}
    R>>(T\&0x7) < (C<< ( (T>>3) \& 0xf) )
    \label{eq:tolerancia4}
\end{equation}


%%%%%%%%%%%%%%%%%%%%%%%%%%%%%%%%%%%%%%%%%%%%%%%%%%%%%%%%%%%%%%%%%%%%%%
\subsection{Estructures de dades adicionals}
%%%%%%%%%%%%%%%%%%%%%%%%%%%%%%%%%%%%%%%%%%%%%%%%%%%%%%%%%%%%%%%%%%%%%%

De banda a les estructures de dades implementades a la biblioteca
est�ndard de C++, es necessitaven algunes estructures amb prestacions
adicionals.

Una d'elles era una estructura de dades que permet�s l'acc�s de dades
amb un cost unitari per� que permet�s la mateixa flexibilitat de
reordenaci� que una llista enlla�ada.

La primera opci� que es va triar es fer un mapa d'identificadors
(num�rics?) amb els elements, per�, aquesta opci�, tot i que
proporcionava un cost d'acc�s logaritmic, l'ordre que mantenien era
l'ordre dels identificadors, i el que es necessitava era mantenir un
ordre arbitrari entre els elements.

TODO: Seguir descrivint les estructures de dades adicionals

%%%%%%%%%%%%%%%%%%%%%%%%%%%%%%%%%%%%%%%%%%%%%%%%%%%%%%%%%%%%%%%%%%%%%%
\subsection{Colors ANSI}
%%%%%%%%%%%%%%%%%%%%%%%%%%%%%%%%%%%%%%%%%%%%%%%%%%%%%%%%%%%%%%%%%%%%%%

Els objectes de la classe CColors, s�n objectes que representen un
color per a un terminal de text que compleixi l'est�ndard ANSI. En
ser inserits a un stream de C++, inserten la seq��ncia per fixar-ho.

TODO: Propietats i operacions de CColor

TODO: Convencions de cara a la utilitzaci� de CColor.


%%%%%%%%%%%%%%%%%%%%%%%%%%%%%%%%%%%%%%%%%%%%%%%%%%%%%%%%%%%%%%%%%%%%%
\newpage

%%%%%%%%%%%%%%%%%%%%%%%%%%%%%%%%%%%%%%%%%%%%%%%%%%%%%%%%%%%%%%%%%%%%%
% Change Log:
% 20000518 VoK - Afegida una explicaci� de com funciona l'objecte bi�top
% 20000518 VoK - 
% 20000701 VoK - Revisada la redacci� (sembla OK)

%%%%%%%%%%%%%%%%%%%%%%%%%%%%%%%%%%%%%%%%%%%%%%%%%%%%%%%%%%%%%%%%%%%%%
\chapter{Representaci� del medi}
\label{sec:biotop}
%%%%%%%%%%%%%%%%%%%%%%%%%%%%%%%%%%%%%%%%%%%%%%%%%%%%%%%%%%%%%%%%%%%%%

%%%%%%%%%%%%%%%%%%%%%%%%%%%%%%%%%%%%%%%%%%%%%%%%%%%%%%%%%%%%%%%%%%%%%
\section{Visi� general}
%%%%%%%%%%%%%%%%%%%%%%%%%%%%%%%%%%%%%%%%%%%%%%%%%%%%%%%%%%%%%%%%%%%%%

El present apartat descriu el funcionament del medi on viuen els
organismes i alguns detalls de disseny i d'implementaci�.
De cara a permetre ampliar f�cilment el model, s'ha volgut fer
un disseny del medi que permeti adaptacions a futures necessitats
de modelatge.
Es descriu el model gen�ric, les particularitzacions b�siques 
implementades i els passos a seguir si es volgu�s implementar 
particularitzacions pr�pies.

Malgrat que el model �s general, est� restringit a bi�tops
que cont�nen posicions discretes. Aix� implica dues coses:
\begin{itemize}
\item   Les posicions dins del bi�top estan quantitzades.
No hi ha m�s que un nombre limitat de posicions a difer�ncia
de l'espai continu real amb infinites posicions possibles.
\item   Es pot considerar una posici� discreta com a representaci� 
d'una zona limitada del substrat continuu real, per�, les 
propietats dins d'aquesta zona del substrat es mantenen uniformes.
\end{itemize}

La discretitzaci� de l'espai juntament amb la discretitzaci� del 
temps �s una caracter�stica comuna a la majoria de sistemes de 
simulaci� i vida artificial. 
L'�nica forma de limitar els efectes artificiosos que aix� pot crear 
�s fer una quantitzaci� tan petita que el seu efecte quedi minimitzat.

El model general per al bi�top �s un model que consta de posicions 
discretes amb les seves corresponents propietats i que t�nen relacions 
de veinatge amb les altres posicions segons una topologia.

Aix� doncs, tenim dos elements que podem modelar independentment.

\begin{itemize}
\item   {\bf Posicions del substrat:} Elements discrets que indiquen
    les propietats d'una zona del bi�top.
\item   {\bf Topologia:} Controla les relacions de veinatge i la
    identificaci� de les posicions per part de la resta del sistema.
\end{itemize}

Combinant aquests dos elements, es pot obtindre un conjunt molt ric
de bi�tops.


%%%%%%%%%%%%%%%%%%%%%%%%%%%%%%%%%%%%%%%%%%%%%%%%%%%%%%%%%%%%%%%%%%%%%
\section{Topologies}
%%%%%%%%%%%%%%%%%%%%%%%%%%%%%%%%%%%%%%%%%%%%%%%%%%%%%%%%%%%%%%%%%%%%%

La topologia determina les relacions de veinatge entre les cel�les.
Si les posicions del bi�top f�ssin nodes d'un graf, la topologia
representaria els vertexs que els uneixen.

Per exemple, podem adaptar la topologia per convertir-la en una
topologia 2D, molt v�lida per simular biosistemes terrestres sense
estratificar, o podem adaptar-ho a una topologia 3D que �s m�s realista
per simular medis fluids, com l'aigua o estratificats com els boscos.

Dissenyar una topologia comporta decidir que �s el que representen
geom�tricament els identificadors de posici� i els identificadors
de despla�ament, i, segons aquesta decisi�, especificar quin ha de 
ser el comportament de les funcionalitats de la topologia que 
relacionen posicions i despla�aments.

Aquestes funcionalitats es basen en dos relacions principals:
Donats un despla�ament i una posici�, la topologia ha de poder determinar 
quina �s la posici� dest�, i, donades dos posicions, la topologia 
ha de poder determinar el despla�ament entre elles.

Com que el nombre de cel�les sovint ser� limitat, el conjunt de 
cel�les formaran una regi� limitada. En aquests casos, el significat
geom�tric dels despla�aments tambe inclou la decisi� de quines s�n 
les ve�nes de les cel�les situades a les vores.

Per exemple, considerem una topologia 2D rectangular. 
Si f�ssim que les cel�les lim�trofes no tinguin ve�nes m�s
enll� dels l�mits ens trobarem davant d'una regi� limitada. Si fem que
les cel�les d'un dels costats es conectin amb les cel�les del costat
oposat, obtindrem una topologia de superf�cie cil�ndrica. Si fem el
mateix amb tots quatre costats obtindrem una topologia de superf�cie
toroidal.

\begin{figure}[ht]
    \centering
    \fbox{TODO: Solucions per a les vores}
%    \fbox{\pdfimage{vores.png}}
	\caption{Conexions de les vores}
	\label{fig:ConexionsVores}
\end{figure}

El conjunt de serveis que una topologia ofereix als seus clients s�n:
\begin{itemize}
\item   Donar l'identificador de la posici� dest� a partir de
    l'identificador d'una posici� origen i d'un despla�ament.
	Possibilita els moviments.
\item   Donar el despla�ament que apropa una posici� d'origen donada
   	a una posici� dest� per l'itinerari m�s curt.
	Possibilita l'orientaci� per objectiu.
\item   Donar el despla�ament oposat a un altre, quan sigui aplicable.
	Possibilita l'orientaci� per evasi�.
\item   Donar l'identificador d'una posici� escollida aleat�riament.
	Possibilita el posicionament aleatori.
\item	Donar la posici� dest� despres de n despla�aments aleatoris 
	des d'una posici� origen 
	Possibilita el moviment aleatori en l'entorn.
\item   Determinar si un identificador de posici� �s v�lid dintre de 
	la topologia.
	Facilita la depuraci� (debugging) i implementart el posicionament aleatori de forma general per�, poc �ptima.
\end{itemize}


%%%%%%%%%%%%%%%%%%%%%%%%%%%%%%%%%%%%%%%%%%%%%%%%%%%%%%%%%%%%%%%%%%%%%
\section{La topologia toroidal}
%%%%%%%%%%%%%%%%%%%%%%%%%%%%%%%%%%%%%%%%%%%%%%%%%%%%%%%%%%%%%%%%%%%%%

La topologia b�sica que ja �s implementada al prototip �s una 
topologia 2D toroidal.

S'ha escollit una topologia en 2D per al primer prototip donat 
que �s molt f�cilment representable en pantalla i a m�s, els c�lculs 
dels despla�aments surten molt senzills i �ptims.
L'optimitzaci� de les funcions de la topologia �s crucial donat
que s'utilitzen de forma intesiva al sistema.

S'ha triat fer-la toroidal perqu�, en els primers prototipus s'ha 
preferit no afegir complexitat al medi tot afegint situacions 
excepcionals per als organismes com ara s�n les vores.

Per optimitzar la implementaci�, hem fet que la ve�na directa d'una posici�
a l'extrem dret sigui una posici� a l'extrem esquerre per�, no pas la que
est� a la mateixa l�nia sin� la que est� una l�nia abaix.
Aix�, tots els despla�aments es resolen amb una �nica suma o resta i
nom�s cal ajustar quan la posici� dest� es surt del domini de les 
posicions existent.

\begin{figure}[ht]
    \centering
    \fbox{TODO: Posar esquema de la topologia}
%    \fbox{\pdfimage{toroidal.png}}
	\caption{Topologia toroidal}
	\label{fig:TopologiaToroidal}
\end{figure}


%%%%%%%%%%%%%%%%%%%%%%%%%%%%%%%%%%%%%%%%%%%%%%%%%%%%%%%%%%%%%%%%%%%%%
\subsection{Despla�aments}
%%%%%%%%%%%%%%%%%%%%%%%%%%%%%%%%%%%%%%%%%%%%%%%%%%%%%%%%%%%%%%%%%%%%%

Cada posici� t� 8 cel�les veines directes. Un despla�ament a
qualsevol d'aquestes 8 cel�les es pot codificar amb 3 bits com indica
la figura \ref{tab:veinesDirectes}

\begin{table}[ht]
\centering
\begin{tabular}{|c|c|c|}
\hline
100 &   000 &   001 \\
\hline
101 &   Origen  &   010 \\
\hline
110 &   111 &   011 \\
\hline
\end{tabular}
    \caption{Ve�nes directes d'una posici�}
    \label{tab:veinesDirectes}
\end{table}

La concatenaci� de N despla�aments b�sics aleatoris tendeix a
formar una distribuci� normal entorn al centre.
Com els vectors de despla�ament tenen 32 bits podriem codificar fins
a 10 despla�aments b�sics consecutius en un vector de despla�ament.
Per�, com ens ser� molt �til poder activar i desactivar cada despla�ament
b�sic, farem servir un quart bit per a cada b�sic per dir si est�
habilitat o inhibit el despla�ament, i en un vector hi caben, doncs,
8 despla�aments b�sics amb els seus bits d'inhibici�.

\begin{table}[ht]
\centering
\begin{tabular}{|l|l|l|l|l|l|l|l|l|l|l|l|l|l|l|l|}
\hline
    h0&d0& h1&d1& h2&d2& h3&d3& h4&d4& h5&d5& h6&d6& h7&d7\\
\hline
    1&101& 1&101& 1&010& 1&110& 1&110& 1&110& 1&110& 1&110\\
\hline
\end{tabular}
    \caption{Codificaci� dels despla�aments al bi�top}
    \label{tab:codiDesplacaments}
\end{table}

Si cal considerar cap ordre en el c�lcul dels despla�aments b�sics,
es fa de m�s significatiu a menys.

La codificaci� dels despla�aments b�sics de la taula 
\ref{tab:veinesDirectes} s'ha fet de tal manera que, si invertim bit 
a bit un vector de despla�ament, obtenim un despla�ament invers.
�s a dir, si invertim tots el bits d'un vector de despla�ament {\tt desp}
a excepci� dels bits d'inhibici� amb l'expressi� {\tt (desp XOR 0x77777777)},
en d�na el despla�ament invers.

%%%%%%%%%%%%%%%%%%%%%%%%%%%%%%%%%%%%%%%%%%%%%%%%%%%%%%%%%%%%%%%%%%%%%
\subsection{Posicions}
%%%%%%%%%%%%%%%%%%%%%%%%%%%%%%%%%%%%%%%%%%%%%%%%%%%%%%%%%%%%%%%%%%%%%

L'altre funci� important de la topologia �s assignar a cada posici� un
identificador �nic dins de la topologia. La resta del sistema far�
servir aquest identificador per referenciar una posici� i, en cas
de necessitar-ho, demanar� a la topologia quin �s el substrat per
aquesta posici�.

En la present implementaci� s'han fet servir enters del 0 a N-1 com a
identificadors de les posicions, on N es el nombre de cel.les.
Aquests identificadors ens permeten fer els despla�aments de forma
molt �ptima si assignem el n�meros per ordre a les cel�les de cada fila.
Per calcular l'identificador de la posici� dest�, nom�s cal afegir 
(o treure), a l'identificador de la posici� origen, un n�mero, que 
dep�n del despla�ament, i ajustar el resultat en cas de sortir-se 
de l�mits.

%TODO: topologia toroidal: I l'uni� seria...

%%%%%%%%%%%%%%%%%%%%%%%%%%%%%%%%%%%%%%%%%%%%%%%%%%%%%%%%%%%%%%%%%%%%%
\section{Substrats}
\label{sec:substratImplementat}
%%%%%%%%%%%%%%%%%%%%%%%%%%%%%%%%%%%%%%%%%%%%%%%%%%%%%%%%%%%%%%%%%%%%%

Un substrat particularitza les propietats del medi per a una posici�.
El substrat pot tenir propietats diferents segons la complexitat que
desitjem per al medi. No hi ha cap restricci� pel disseny dels 
substrats de cara a que es pugui muntar un bi�top amb ells.

A continuaci�, es descriu com est� definit el substrat b�sic 
implementat a l'aplicatiu.
Les dos propietats m�s importants del substrat s�n qui ocupa el
substrat, i quins nutrients hi han.

%%%%%%%%%%%%%%%%%%%%%%%%%%%%%%%%%%%%%%%%%%%%%%%%%%%%%%%%%%%%%%%%%%%%%
\subsection{Ocupants}
%%%%%%%%%%%%%%%%%%%%%%%%%%%%%%%%%%%%%%%%%%%%%%%%%%%%%%%%%%%%%%%%%%%%%

S'ha restringit l'ocupaci� de les posicions per part dels organismes
a un s�l organisme per posici� i a una sola posici� per organisme.
La ra� ha estat fer possible referenciar un organisme per la seva posici�.
Aix� simplificar�, a les relacions entre organismes, com referir l'altre organisme.
Tamb�, com a avantatge adicional, simplifica la interf�cie amb l'usuari
per seleccionar un organisme seleccionant la seva posici�.

%%%%%%%%%%%%%%%%%%%%%%%%%%%%%%%%%%%%%%%%%%%%%%%%%%%%%%%%%%%%%%%%%%%%%
\subsection{Nutrients}
%%%%%%%%%%%%%%%%%%%%%%%%%%%%%%%%%%%%%%%%%%%%%%%%%%%%%%%%%%%%%%%%%%%%%

En quant els nutrients que hi pot haver en una mateixa posici� del substrat,
cada posici� t� un nombre de nutrients m�xim definit.
Quan s'afegeixen nutrients per damunt d'aquest nombre, els nutrients m�s
antics desapareixen.

Els nutrients estan diferenciats qualitativament amb un sencer que indica
el seu tipus.

La recollida de nutrients, es fa amb un patr� de cerca pel tipus de
nutrient i una toler�ncia a nivell de bit segons la funci� de
compatibilitat est�ndard.
%%\footnote{
Com s'explica a l'apartat \ref{sec:compatibilitat}, aquesta funci�
retorna cert si
\begin{verbatim}
    ((Patro^Clau)&Random&~Tolerancia)==0
\end{verbatim}
, de tal forma que un 1 a una posici� de la toler�ncia significa que,
encara que no es correspongui aquest bit no es tindr� en compte.
%%}
La cerca es fa des dels m�s nous fins els m�s vells.

%%%%%%%%%%%%%%%%%%%%%%%%%%%%%%%%%%%%%%%%%%%%%%%%%%%%%%%%%%%%%%%%%%%%%
\section{Implementaci�. Bi�tops}
%%%%%%%%%%%%%%%%%%%%%%%%%%%%%%%%%%%%%%%%%%%%%%%%%%%%%%%%%%%%%%%%%%%%%

En aquest apartat i en els seg�ents del cap�tol s'explica com est� 
implementat el medi i com s'hi poden fer modificacions mitjan�ant programaci�.
Aquests apartats no s�n d'interes als usuaris que no vulguin modificar
el programa.

Per implementar el medi, tenim tres elements: La topologia, el substrat
i el bi�top que �s l'associaci� d'una topologia i un tipus de substrat.

Per definir el tipus de substrat cal una classe les inst�ncies de la 
qual representin el substrat de cada posici� individual, �s a dir, 
una cassella.

Totes les topologies deriven de la classe {\tt CTopologia}, que 
defineix el protocol de m�todes virtuals als que una topologia ha
de respondre per definir la seva geometria.

Per poder escollir diferents substrats i topologies sense gaire
esfor� de programaci� s'ha creat una classe patr� ({\em template}) 
anomenada {\tt CBiotop}.

Un bi�top �s una especialitzaci� d'una topologia. 
Aquesta especialitzaci� es basa en dos fets: 
\begin{itemize}
\item
Per un costat, un bi�top cont� un vector de substrats discrets, 
als quals es pot accedir amb un identificador de posici�.
El tipus del substrat �s el par�metre del patr�.
\item
El comportament que t� un bi�top com a topologia, �s a dir, de cara
a establir la geometria del medi, es pot determinar en temps d'execuci�
tot fent que les crides pr�pies de topologia es redireccionin cap a
un objecte topologia indicat.
\end{itemize}

En resum, es pot determinar en temps de compilaci� el tipus de substrat
amb el par�metre del template i, la topologia, en temps d'execuci�, tot 
assignant-li una topologia o una altra a la que redireccionar les crides.


%%%%%%%%%%%%%%%%%%%%%%%%%%%%%%%%%%%%%%%%%%%%%%%%%%%%%%%%%%%%%%%%%%%%%%
\section{Programaci� de nous substrats}
%%%%%%%%%%%%%%%%%%%%%%%%%%%%%%%%%%%%%%%%%%%%%%%%%%%%%%%%%%%%%%%%%%%%%%

Podem dissenyar un substrat sense preocupar-nos de cap altre element 
del bi�top.
Hi ha, per�, depend�ncies entre el disseny del substrat i la 
resta del sistema. Estan localitzades en:
\begin{itemize}
\item Execucions de les instruccions dels organismes (Biosistema)
\item Interaccions amb els agents externs
\item Interaccions amb el sistema de monitoritzaci� i logs.
\end{itemize}

% TODO: Refer�ncies als cap�tols corresponents.

Es recomana implementar els procediments {\tt load} i {\tt store}
per pasivitzar i recuperar l'estat de la cel�la a un medi serie.

El substrat implementat a l'eina i descrit en l'apartat \ref{sec:substratImplementat} �s la classe {\tt CSubstrat}.

%%%%%%%%%%%%%%%%%%%%%%%%%%%%%%%%%%%%%%%%%%%%%%%%%%%%%%%%%%%%%%%%%%%%%%
\section{Programaci� de noves topologies}
%%%%%%%%%%%%%%%%%%%%%%%%%%%%%%%%%%%%%%%%%%%%%%%%%%%%%%%%%%%%%%%%%%%%%%

Si l'usuari necessit�s crear un nou tipus de topologia, cal que la
faci heretar de CTopologia i redefineixi els m�todes del protocol
d'acc�s que CTopologia estableix.

El protocol est� pensat per aconseguir que la resta de sistema nom�s 
hagui de manegar identificadors de posicions i identificadors de
despla�aments. El significat geom�tric d'aquests identificadors
queda ocult darrera de l'implementacio del protocol.

Quan es deriva de CTopologia, el principal que caldria redefinir,
si cal, �s:
\begin{itemize}
\item   Un {\bf constructor} amb els par�metres significatius per 
    a la topologia. Per exemple, en una topologia rectangular �s 
    significatiu indicar l'altura i l'amplada.
	Cal acutalitzar la variable interna que guarda el nombre de posicions.
\item   {\tt t\_posicio CTopologia::desplacament (t\_posicio origen, t\_desplacament desplacament)}:
    Una funci� per averiguar la posici� dest� en aplicar-li un vector
    de despla�ament a una posici� origen.
    CTopologia, la defineix, per defecte, de tal forma que el 
	resultat �s una posici� dest� aleat�ria v�lida.
\item   {\tt bool CTopologia::esValid(t\_posicio id)}:
    Una funci� per saber si un identificador �s v�lid. Nom�s cal
    redefinir-ho si es modifica la correspond�ncia directa entre
    identificador de posici� i index de casella en l'array de
    substrats reservada per CTopologia::CTopologia
\item   {\tt t\_posicio CTopologia::posicioAleatoria ()}:
    Una funci� per obtindre aleat�riament una posici� v�lida de la
    topologia. La funci� general que no caldria redefinir seria
    {\setlinespacing{1}
    \begin{verbatim}
        {
            uint32 pos;
            do {pos=rnd.get();} while (!esValid(pos));
            return pos
        }
    \end{verbatim}
    }
    per�, CTopologia no fa servir aquest algorisme donat que
    optimitza agafant un n�mero aleatori entre 0 i N. Aquesta
    optimitzaci� funciona mentre es mantingui la correspond�ncia
    entre identificador i index abans comentada.
    Si la subclasse la trenca, es quan cal redefinir la funci�.
\item   {\tt bool CTopologia::unio (t\_posicio origen, t\_posicio desti, t\_desplacament \& desp)}:
    Una funci� per calcular el primer d'un conjunt de desplacaments
    que cal fer per anar de l'origen al dest�.
    Retorna cert si el desplacament �s suficient per arribar a la posici� dest�.
    CTopologia, la defineix de tal forma que el resultat �s un despla�ament
    aleatori i retorna sempre fals (mai hi arriba).
\end{itemize}

Com a exemple, �s molt aconsejable fixar-s'hi en com est� feta la 
classe {\tt CTopologiaToroidal} abans d'implementar una topologia pr�pia.

%%%%%%%%%%%%%%%%%%%%%%%%%%%%%%%%%%%%%%%%%%%%%%%%%%%%%%%%%%%%%%%%%%%%%%
\section{Programaci� de bi�tops din�mics o heterog�nis}
%%%%%%%%%%%%%%%%%%%%%%%%%%%%%%%%%%%%%%%%%%%%%%%%%%%%%%%%%%%%%%%%%%%%%%

Per la forma en que est� dissenyada la classe {\tt CBiotop}, els 
bi�tops que es poden construir actualment t�nen diverses restriccions:

\begin{itemize}
\item {\bf Controlen un grup est�tic de posicions:} El nombre de posicions
est� limitat en el moment de creaci�. La topologia pot controlar el nombre
de posicions disponibles per� sempre dintre d'aquest l�mit establert 
inicialment.
\item {\bf El conjunt de substrats �s homogeni:} No poden conviure
substrats de diferents classes. Poden diferir les propietats per� han
de ser de la mateixa classe.
\item {\bf Es restringeix els identificadors:} Han de estar entre $0$ i 
$N-1$ sent $N$ el nombre de casselles m�xim determinat inicialment.
\end{itemize}

Aquestes restriccions conseq��ncies directes del fet de que el 
conjunt de substrats s'hagi implementat amb un vector est�tic.

Si el bi�top que es vol simular no compleix aquestes caracter�stiques
ni es pot simular el comportament desitjat, llavors caldria modificar
la classe CBiotop, o crear-ne una classe nova similar.

% TODO: Com

%%%%%%%%%%%%%%%%%%%%%%%%%%%%%%%%%%%%%%%%%%%%%%%%%%%%%%%%%%%%%%%%%%%%%
\newpage

% Time Log
% 19990804 23:00 - 23:45

%%%%%%%%%%%%%%%%%%%%%%%%%%%%%%%%%%%%%%%%%%%%%%%%%%%%%%%%%%%%%%%%%%%%%%
\section{Agents externs}
%%%%%%%%%%%%%%%%%%%%%%%%%%%%%%%%%%%%%%%%%%%%%%%%%%%%%%%%%%%%%%%%%%%%%%

%%%%%%%%%%%%%%%%%%%%%%%%%%%%%%%%%%%%%%%%%%%%%%%%%%%%%%%%%%%%%%%%%%%%%%
\subsection{Trets generals}
%%%%%%%%%%%%%%%%%%%%%%%%%%%%%%%%%%%%%%%%%%%%%%%%%%%%%%%%%%%%%%%%%%%%%%

Els agents externs s�n objectes que disparen una acci� determinada 
quan s�n cridats. La majoria d'accions es produeixen sobre el bi�top, 
sobre l'estat d'altres agents externs o sobre par�metres globals de
configuraci�.

El paper dels agents externs a dins del sistema �s permetre a 
l'usuari controlar com evolucionar� l'entorn on es mouran els 
organismes de cara a obtindre resultats que s'hi puguin contrastar.
Per s� mateix, el conjunt d'agents implementats permet configuracions
molt complexes, per�, a m�s, ofereix un seguit d'eines molt �tils
per que l'usuari-programador pugui ampliar aquests agents.

Quan un agent es cridat per realitzar la seva acci�, es diu que
l'agent ha sigut {\bf accionat}. Quan alg� requereix un valor contingut
en l'estat de l'agent es diu que l'agent ha sigut {\bf consultat}.

Direm que un agent A t� com a {\bf subordinat} a un altre agent B si �s
A qui acciona a B. Ho representarem aix�: $A \rightarrow B$.
L'estructura de subordinaci� ha de ser un arbre on els subordinats
s�n els fills d'aquell a qui es subordinen.

Direm que un agent A �s {\bf depenent} d'un altre agent B si A,
quan �s accionat, necessita consultar l'estat de l'agent B.
Ho representarem aix�: A(B). La consulta no ha d'implicar cap
modificaci� ni rec�lcul d'estat en l'agent consultat.
Aix� permet que no hi hagi restriccions en l'estructura de
depend�ncia i que puguin existir depend�ncies creuades.
\footnote{
Tot i aix�, �s important preveure que l'ordre d'accionat entre
agents interdependents podria implicar variacions en els resultats.
}

Tots els agents duen un nom assocciat que, per defecte, coincideix amb
un prefixe i un n�mero de s�rie �nic entre tots els agents. Aquest
nom es pot canviar per un que sigui m�s mnemot�cnic per a l'usuari.

TODO: El tema dels logs

%%%%%%%%%%%%%%%%%%%%%%%%%%%%%%%%%%%%%%%%%%%%%%%%%%%%%%%%%%%%%%%%%%%%%%
\subsection{Agents Subordinadors}
%%%%%%%%%%%%%%%%%%%%%%%%%%%%%%%%%%%%%%%%%%%%%%%%%%%%%%%%%%%%%%%%%%%%%%

Els agents subordinadors s�n agents que, quan s�n accionats, accionen
tot un seguit d'agents subordinats.

%%%%%%%%%%%%%%%%%%%%%%%%%%%%%%%%%%%%%%%%%%%%%%%%%%%%%%%%%%%%%%%%%%%%%%
\subsubsection{Agents M�ltiples}
%%%%%%%%%%%%%%%%%%%%%%%%%%%%%%%%%%%%%%%%%%%%%%%%%%%%%%%%%%%%%%%%%%%%%%

L'agent m�ltiple acciona una i nom�s una vegada cadascun dels
agents subordinats cada cop que �s accionat.

%%%%%%%%%%%%%%%%%%%%%%%%%%%%%%%%%%%%%%%%%%%%%%%%%%%%%%%%%%%%%%%%%%%%%%
\subsubsection{Agents Temporitzadors}
%%%%%%%%%%%%%%%%%%%%%%%%%%%%%%%%%%%%%%%%%%%%%%%%%%%%%%%%%%%%%%%%%%%%%%

Els agents temporitzadors s�n agents multiples que no sempre que
reben un accionat el propaguen cap els subordinats. Estableixen dos
per�odes, un actiu i un altre inactiu. Els accionats nom�s es
propagen als subordinats durant el per�ode actiu.

Els per�odes es defineixen mitjan�ant tres par�metres: El per�ode
m�nim �s el n�mero d'accionats que dura el per�ode com a m�nim.
Aquest m�nim es pot augmentar de forma no determin�stica el resultat
de sumar-li n vegades un n�mero aleatori en l'interval [0,m] (els
corxets indiquen que els extrems estan inclosos). {\em n} �s el
n�mero de daus, i {\em m} �s el valor m�xim o magnitud del dau.

D'aquesta especificaci� es pot deduir algunes dades, pot ser, m�s
intuitives per a l'usuari:
\begin{itemize}
\item   El valor m�xim que pot adoptar el per�ode �s el m�nim m�s n*m
\item   Un s�l dau equival a una distribuci� uniforme entre els l�mits
\item   A mesura que incrementem el nombre de daus, la distribuci� dels
    per�odes s'aproxima a una distribuci� normal entorn al centre
    entre el valor m�xim i m�nim, amb un desviaci� t�pica cada vegada
    menor.
\end{itemize}

Per defecte, els par�metres que introdueixen indeterminisme en els
temporitzadors, com s�n els daus, estan ajustats de manera que el seu
efecte sigui nul. Si no es toca res m�s que els per�odes m�nims,
actuar� de forma determinista. De la mateixa manera, els per�odes
m�nims dels cicles estan ajustats, per defecte, per que sempre
s'estigui en un cicle actiu. D'aquesta forma, si no es configura res,
l'efecte d'un temporitzador �s el d'un agent m�ltiple ordinari.

Par�metres per defecte i una execuci�:
\begin{itemize}
\item   Cicle Actiu ( minim=1 daus=0 magnitud=0)
\item   Cicle Inactiu ( minim=0 daus=0 magnitud=0)
\item   Periode Actual ( actiu )
\item   Periode Restant (1)
\end{itemize}

Par�metres ilustratius:
\begin{itemize}
\item   Cicle Actiu ( minim=0 daus=2 magnitud=3)
\item   Cicle Inactiu ( minim=4 daus=0 magnitud=0)
\item   Periode Actual ( actiu )
\item   Periode Restant (1)
\end{itemize}

A continuaci� hi ha una execuci� dels par�metres anteriors. Els
guions representen accionats durant el per�ode inactiu i les O's
representen accionats durant el per�ode actiu.

{\setlinespacing{1}
\begin{verbatim}
OO----OOOOO----OO----O----OOOO----O----OOO----OOOO----OOO----OOOO----O----OO----
OOO----OOOO--------OOO----O--------OOO----OOOO----O----OOOO----OOOO--------OOOO-
---OOO----OOO----OOO----OOOOO----O----OOOOOO----OOOO----O----OOOO----OOO----OO--
--OO----OOOO----OOO----OOO----OOOOO----OOO----OOOO----OOO----OO----OO----OO----O
OOOOO----OOO----OOOOO----OOOO----OO----OOOOO----OOOO----OOO----OOO----OOOOO----O
\end{verbatim}
}

%%%%%%%%%%%%%%%%%%%%%%%%%%%%%%%%%%%%%%%%%%%%%%%%%%%%%%%%%%%%%%%%%%%%%%
\subsubsection{Agents Probabilitzadors}
%%%%%%%%%%%%%%%%%%%%%%%%%%%%%%%%%%%%%%%%%%%%%%%%%%%%%%%%%%%%%%%%%%%%%%

Els agents probabilitzadors s�n tamb� agents m�ltiples que controlen
si l'accionat es propaga cap els subordinats o no. Per�, a difer�ncia
dels agents temporitzadors, ho fan mitjan�ant una llei probabil�stica.
Si es d�na la probabilitat, s'accionen els subordinats, si no es d�na,
no s'accionen.

La probabilitat es defineix amb el nombre de vegades que es donaria
la probabilitat en un tamany de mostra.
Per exemple, podem definir una probabilitat dient que es d�na 3 de
cada 14 vegades. 14 �s el tamany de mostra i 3 les vegades que es
donaria en la mostra.

Els par�metres estan ajustats per defecte a valors que fan del
probabilitzador un agent m�ltiple ordinari.

Par�metres ilustratius:
\begin{itemize}
\item   Probabilitat ( mostra=40 encerts=25)
\end{itemize}

A continuaci� hi ha una execuci� dels par�metres anteriors. Els
guions representen accionats en els quals no s'ha donat la
probabilitat i les O's, accionats en els quals s� s'ha donat.

{\setlinespacing{1}
\begin{verbatim}
OOOO-O-O-O-OO----OOOO-OOOOOOOO-OOO-OOOOOOOOOOO-OOOOOOOOO---O-OOOOOOO-O--OOOO-OO-
OOO-OOO-O-OOOO-OOO-O-OOO-OOOOOOOOOOOO-OOOOOOO-OOOOOO--OOO-OOO-OO-O-OOOOO--O---O-
OO--O---OOO-OOOOOOOOO-O-OOO-O-OOO-O--OOOOOOOO-OO-O-OO-OOO-OO-OOOO-OOOOOO-OOO-OOO
--OOOO-OOOO-O-OOO-OO-OO-O--OO-OOOO--O--O-O-OO--OO-O-O-OOO---O-O-OO------OO-OO-O-
OOO-OO-O-OO-OO-OOOOO----OO-OO-----O-OOOOO--O---OOOOOOOOOO-O-OO-O-OOOOO---OOO-OO-
\end{verbatim}
}

%%%%%%%%%%%%%%%%%%%%%%%%%%%%%%%%%%%%%%%%%%%%%%%%%%%%%%%%%%%%%%%%%%%%%%
\subsubsection{Agents Iteradors}
%%%%%%%%%%%%%%%%%%%%%%%%%%%%%%%%%%%%%%%%%%%%%%%%%%%%%%%%%%%%%%%%%%%%%%

Els agents iteradors s�n agents m�ltiples que no limiten els
accionats que arriben als seus subordinats, sin� que el que fan �s
multiplicar els accionats que li arriben.

M�s concretament, quan un agent iterador �s accionat, els seus
subordinats, s�n accionats un n�mero de vegades que es calcula a
partir d'un m�nim i uns daus com els que feiem servir per als
per�odes dels temporitzadors.

Per defecte, la part indetermin�stica (els daus) no t� cap efecte,
i la part deterministica (el m�nim) est� posada a un valor (1) que
el fa equivalent a un agent m�ltiple ordinari.

Par�metres ilustratius:
\begin{itemize}
\item   Iteracions ( minim=2 daus=2 magnitud=4)
\end{itemize}

A continuaci� hi ha una execuci� dels par�metres anteriors. Els
par�ntesis agrupen els accionaments dels subordinats que es fan sota
un mateix accionament de l'iterador.

{\setlinespacing{1}
\begin{verbatim}
(OOOOOO)(OOOOOOOO)(OOOOOOO)(OOOOO)(OOO)(OOOOOOO)(OO)(OOOOOOO)(OOOOOOO)(OOOOOO)(O
OOOOOOO)(OOOOO)(OOOOOO)(OOO)(OOOOOO)(OOOO)(OOOOOO)(OOOOOO)(OOO)(OOOOOO)
\end{verbatim}
}

Amb dos accions subordinades una execuci� quedaria com aix�:

{\setlinespacing{1}
\begin{verbatim}
(OEOEOEOEOEOEOEOEOEOE)(OEOEOEOEOEOEOEOEOE)(OEOEOEOEOEOEOE)(OEOEOEOE)(OEOEOEOEOEO
EOEOE)(OEOEOEOEOEOEOEOE)(OEOEOEOEOEOEOEOEOEOE)(OEOE)(OEOEOEOEOEOEOE)(OEOEOEOEOEO
EOE)(OEOEOEOEOEOE)(OEOEOEOEOEOE)(OEOEOEOEOE)(OEOEOEOEOEOEOE)(OEOEOEOEOEOE)(OEOEO
EOEOEOE)(OEOEOEOEOEOE)(OEOEOEOEOE)(OEOEOEOEOE)(OEOEOEOEOE)
\end{verbatim}
}
on les E's representen l'execuci� de la segona acci� subordinada.

%%%%%%%%%%%%%%%%%%%%%%%%%%%%%%%%%%%%%%%%%%%%%%%%%%%%%%%%%%%%%%%%%%%%%%
\subsection{Agents Posicionadors}
%%%%%%%%%%%%%%%%%%%%%%%%%%%%%%%%%%%%%%%%%%%%%%%%%%%%%%%%%%%%%%%%%%%%%%

Els agents posicionadors controlen una posici� en la topologia del bi�top.
No t�nen subordinats, per�, generalment hi ha agents que en depenen del
seu valor i segons el tipus de posicionador per recalcular la seva
posici� fan servir altres agents dels quals depenen.

\begin{description}
\item[{\bf Posicionador B�sic:}] No modifica la seva posici� si �s accionat.
    A menys que, per configuraci�, es fixi a una posici� concreta,
    s'inicialitza amb una posici� aleatoria v�lida dins de la topologia,
\item[{\bf Posicionador Aleatori:}] Cada cop que �s accionat pren una posici� aleatoria
    v�lida dintre de la topologia.
\item[{\bf Posicionador Zonal:}] Cada cop que �s accionat pren una posici�
    aleatoria dintre d'una zona.
    La zona es defineix per una posici� central, determinada per un altre
    posicionador de qual dep�n, i un radi, que no �s m�s que el nombre de
    despla�aments aleatoris que es fan a partir d'aquesta posici� central
    per trobar la posici� final.
    Les posicions tendeixen a adoptar una distribuci� normal en l'entorn
    de la posici� central.
\item[{\bf Posicionador Seq�encial:}] Cada cop que �s accionat pren la
    posici� del seg�ent posicionador que hi ha en una seq��ncia
    de posicionadors. Els agents posicionadors de la seq��ncia s�n
    depend�ncia del posicionador seq�encial.
\item[{\bf Posicionador Direccional (Itinerari):}] Cada cop que �s
    accionat pren la posici� que en resulta d'aplicar-li un
    despla�ament a la posici� anterior.
    El despla�ament el determina un agent direccionador que �s depend�ncia.
    Els agents direccionadors s'expliquen al seg�ent apartat.
\end{description}

A continuaci� es presenten exemples d'execuci� d'un posicionador aleatori,
un de seq�encial i un de zonal aplicats sobre una topologia toroidal.

\begin{figure}[h]
    \centering
    \fbox{TODO: Exemples d'execuci� posicionadors seq�encial, zonal i aleatori}
    \label{fig:exemplesPosicionadors}
    \caption{Exemples de posicionadors: seq�encial, zonal i aleatori}
\end{figure}


%%%%%%%%%%%%%%%%%%%%%%%%%%%%%%%%%%%%%%%%%%%%%%%%%%%%%%%%%%%%%%%%%%%%%%
\subsection{Agents Direccionadors}
%%%%%%%%%%%%%%%%%%%%%%%%%%%%%%%%%%%%%%%%%%%%%%%%%%%%%%%%%%%%%%%%%%%%%%


Els agents direccionadors controlen una direcci� per calcular
despla�aments dintre de la topologia del bi�top. La seva utilitat
principal radica en controlar la posici� d'un posicionador de tipus
direccional, per�, no es descarten altres aplicacions futures.

\begin{description}
\item[{\bf Direccionador B�sic:}] No modifica la seva direcci� si �s accionat.
\item[{\bf Direccionador Aleatori:}] Cada cop que �s accionat pren una direcci� aleatoria.
\item[{\bf Direccionador Seq�encial:}] Cada cop que �s accionat pren la
    direcci� del seg�ent direccionador que hi ha en una seq��ncia
    de direccionadors. Els agents direccionadors de la seq��ncia s�n
    depend�ncia del direccionador seq�encial.
\end{description}

A continuaci� es presenten exemples d'execuci� d'un posicionador
direccional, que dep�n de diferents tipus de direccionadors.

\begin{figure}[h]
    \centering
    \fbox{TODO: Exemple d'execuci� d'un itinerari amb els diferents direccionadors}
    \label{fig:exemplesDireccionadors}
    \caption{Exemples de direccionadors (seq�encial, fixe i aleatori) controlant un posicionador direccional}
\end{figure}

Per obtindre aquests resultats ha calgut fer servir subordinadors
per que el posicionador s'accion�s m�s que el direccionador.
La topologia de l'exemple �s toroidal.



%%%%%%%%%%%%%%%%%%%%%%%%%%%%%%%%%%%%%%%%%%%%%%%%%%%%%%%%%%%%%%%%%%%%%%
\subsection{Agents Actuadors}
%%%%%%%%%%%%%%%%%%%%%%%%%%%%%%%%%%%%%%%%%%%%%%%%%%%%%%%%%%%%%%%%%%%%%%

Els agents actuadors s�n els que finalment modifiquen el substrat.
Els actuadors depenen d'un agent posicionador que els indica la
cel�la que han de modificar.

La majoria dels agents que hem vist fins ara eren molt independents
davant de les modificacions en la topologia i en la composici� del
substrat que es puguin fer m�s endavant. Les especialitzacions dels
actuadors, en canvi, han de dependre per for�a del substrat i la seva
composici�, perqu� actuen sobre ell. Segueixen sent independents,
per�, de la topologia.

Aqu� a sota, expliquem alguns actuadors v�lids pel substrat
implementat en aquest treball.


%%%%%%%%%%%%%%%%%%%%%%%%%%%%%%%%%%%%%%%%%%%%%%%%%%%%%%%%%%%%%%%%%%%%%%
\subsubsection{Agents Nutridors}
%%%%%%%%%%%%%%%%%%%%%%%%%%%%%%%%%%%%%%%%%%%%%%%%%%%%%%%%%%%%%%%%%%%%%%

Aquests actuadors depositen nutrients al substrat.
Un tipus de nutrient es codifica amb un enter de 32 bits sense signe.
El tipus de nutrient que es depositar� s'especifica amb dos nombres
enters de 32 bits sense signe. El primer enter indica el n�mero
del tipus b�sic, i el segon indica els bits del tipus b�sic que
poden variar aleatoriament.

Per exemple, el parell
\begin{tabular}{rl}
element b�sic: & 0x0000000000000000 \\
variabilitat:  & 0xFFFFFFFF00000000
\end{tabular}
genera elements que tenen la part baixa igual que l'element b�sic
(a zero) i la part alta al atzar.

%%%%%%%%%%%%%%%%%%%%%%%%%%%%%%%%%%%%%%%%%%%%%%%%%%%%%%%%%%%%%%%%%%%%%%
\subsubsection{Agents Desnutridors}
%%%%%%%%%%%%%%%%%%%%%%%%%%%%%%%%%%%%%%%%%%%%%%%%%%%%%%%%%%%%%%%%%%%%%%

Els desnutridors s�n molt semblants als nutridors, per�, en comptes
de afegir nutrients, en treuen. Es pot treure selectivament cercant
un element qu�mic que s'apropi al que s'indica o es pot especificar
una toler�ncia per a certs bits.

%%%%%%%%%%%%%%%%%%%%%%%%%%%%%%%%%%%%%%%%%%%%%%%%%%%%%%%%%%%%%%%%%%%%%%
\subsection{Arxius de configuraci� d'agents}
%%%%%%%%%%%%%%%%%%%%%%%%%%%%%%%%%%%%%%%%%%%%%%%%%%%%%%%%%%%%%%%%%%%%%%

%%%%%%%%%%%%%%%%%%%%%%%%%%%%%%%%%%%%%%%%%%%%%%%%%%%%%%%%%%%%%%%%%%%%%%
\subsubsection{Motivaci� i criteris de disseny}
%%%%%%%%%%%%%%%%%%%%%%%%%%%%%%%%%%%%%%%%%%%%%%%%%%%%%%%%%%%%%%%%%%%%%%

De cara a poder passivitzar un biosistema a disc per poder-ho
restaurar posteriorment, caldria tamb� poder passivitzar i restaurar
l'estat dels seus agents.
L'arxiu de configuraci� d'agents �s un arxiu de text que cont�
l'estat i l'estructura dels agents d'un biosistema, que es
pot extreure en un moment donat i restaurar-ho posteriorment.

Els agents a un biosistema, com s'ha dit abans, formen una
estructura d'arbre segons les seves relacions de subordinaci�.
Cada arxiu de configuraci� cont� un arbre d'agents subordinats
partint d'un agent arrel.
Seria possible penjar tota l'estructura d'un arxiu de configuraci�
i subordinar-ho a un agent d'una estructura ja existent a un biosistema.
Es podria formar una mena de biblioteca d'arxius amb configuracions
comunes que es podrien convinar per muntar r�pidament l'estructura
d'agents d'un biosistema.

%%%%%%%%%%%%%%%%%%%%%%%%%%%%%%%%%%%%%%%%%%%%%%%%%%%%%%%%%%%%%%%%%%%%%%
\subsubsection{Estructura}
%%%%%%%%%%%%%%%%%%%%%%%%%%%%%%%%%%%%%%%%%%%%%%%%%%%%%%%%%%%%%%%%%%%%%%

Els arxius de configuraci� d'agents t�nen una estructura molt simple:
Primer van unes linies de text que determinen, per cada agent, el seu
nom i el seu tipus.
Un cop definits els noms i els tipus, es configuren els par�metres
per a cada agent.

La definici� dels noms i els tipus es fa amb una l�nia per cada agent
sent la primera l�nia la que defineix l'agent que est� en l'arrel de
la estructura de subordinaci�.
A cada l�nia es posa, per ordre i {\em separats per espais} un signe
asterisc, el nom i el tipus.

Quan es carrega un arxiu de configuraci� d'agents, si �s possible
a cada agent se li d�na el nom amb el que apareix a l'arxiu, per�
aix� no �s posible si ja existeix un amb el mateix nom. En aquest
cas, se li d�na un nom per defecte i s'hi tradueixen totes les
posteriors refer�ncies al nom antic.

Els noms poden contenir qualsevol caracter que no es consideri
un espai a C (espais, tabuladors, retorns...).

El tipus s'especifica amb un identificador propi de cada tipus d'agent.
Dins d'un arxiu de configuraci� d'agents, es reconeixen els seg�ents tipus:

\begin{table}[h]
\begin{tabular}{|l|l|}\hline
{\bf Nom de tipus a la mem�ria}     & {\bf Nom del tipus a un fitxer de configuraci�}\\\hline
Agent Subordinador Multiple     & Agent/Multiple            \\\hline
Agent Subordinador Temporitzador    & Agent/Multiple/Temporitzador      \\\hline
Agent Subordinador Iterador     & Agent/Multiple/Iterador       \\\hline
Agent Subordinador Aleaturitzador   & Agent/Multiple/Aleaturitzador     \\\hline
Posicionador Fixe           & Agent/Posicionador            \\\hline
Posicionador Aleatori           & Agent/Posicionador/Aleatori       \\\hline
Posicionador Zonal          & Agent/Posicionador/Zonal      \\\hline
Posicionador Direccional (Itinerari)    & Agent/Posicionador/Direccional    \\\hline
Direccionador Fixe          & Agent/Direccionador           \\\hline
Direccionador Aleatori          & Agent/Direccionador/Aleatori      \\\hline
Actuador Nutridor           & Agent/Actuador/Nutridor       \\\hline
Actuador Desnutridor            & Agent/Actuador/Nutridor/Invers    \\\hline
\end{tabular}
    \caption{Tipus d'agents implementats al projecte i identificadors associats}
    \label{tab:tipusAgent}
\end{table}

Es fan servir identificadors jerarquics molt semblants als que es
fan servir a UNIX per identificar els directoris. La jerarquia
de noms el que especifica aqu� �s una jerarquia de tipus i subtipus
de tal forma que si, per exemple, a un lloc es requereix {\em Agent/Posicionador},
aquest lloc el pot ocupar tant un {\em Agent/Posicionador/Direccional} com un
{\em Agent/Posicionador/Zonal}.

Una definici� de noms i tipus podria quedar com segueix:

\begin{verbatim}
* Agent_0000 Agent/Multiple
* Agent_0001 Agent/Multiple/Temporitzador
* Agent_0002 Agent/Direccionador/Aleatori
* Agent_0003 Agent/Posicionador/Direccional
* Agent_0004 Agent/Multiple/Iterador
* Agent_0005 Agent/Actuador/Nutridor
\end{verbatim}

Aqu�, {\em Agent\_0000} seria l'agent arrel.

Un cop definits els noms i els tipus dels agents, cal configurar els
seus par�metres. Per configurar un agent primer cal posar una l�nia
amb el signe + i el nom de l'agent separats per un espai, i, despr�s,
tot un seguit de l�nies de configuraci� de par�metres.
Les linies de configuraci� de par�metres comencen amb un signe menys
i el nom del par�metre i es segueix amb els valors que necessita
el par�metre per configurar-se, tot separat per espais.
Com veurem al seg�ent exemple, �s normal que un par�metre s'especifiqui
amb diversos valors separats per espais.
La posici� dels valors acostuma a ser significativa o sigui que �s
important mantenir l'ordre.

Per configurar un Nutridor es faria de la seg�ent forma
\begin{verbatim}
+ Agent_0004
- Posicionador Agent_0003
- Composicio 31 0
\end{verbatim}

Quan un par�metre necessita com a valor un altre agent, fa servir
els seu nom com a refer�ncia.

Al seg�ent apartat, es detalla els par�metres que controlen cada
tipus d'agent.


%%%%%%%%%%%%%%%%%%%%%%%%%%%%%%%%%%%%%%%%%%%%%%%%%%%%%%%%%%%%%%%%%%%%%%
\subsection{Par�metres configurables per a cada tipus d'agent}
%%%%%%%%%%%%%%%%%%%%%%%%%%%%%%%%%%%%%%%%%%%%%%%%%%%%%%%%%%%%%%%%%%%%%%

El que segueix �s una especificaci� de com es configuren els
par�metres dels tipus d'agent implementats mitjan�ant el fitxer
de configuraci�. Per fer-ho fem servir la seg�ent estructura:

Per a cada par�metre de cada tipus d'agent es fa una petita explicaci�
i es detallen, ordenats tal qual han d'apar�ixer, els valors que el
defineixen.

Els valors dels par�metres es detallen posant un tipus de dada,
dos punts, una petita explicaci� del valor i, entre par�ntesis,
les restriccions que s'hi apliquen.

Als agents implementats, els tipus de dada possibles pels valors s�n:
\begin{itemize}
\item   {\bf agent:} Nom d'un agent especificat a la definici� de noms i tipus
\item   {\bf uint32:} Senser sense signe codificable en 32 bits i expressat en base decimal
\item   {\bf id(Alternativa1/Alternativa2...):} Un dels identificadors posats com a alternativa
\end{itemize}

Despr�s dels par�metres de cada tipus hi ha un exemple de com quedarien
les l�nies de configuraci�.


\subsubsection{MultiAgent (Agent/Multiple)}
\begin{description}
\item[Accio:] Determina un agent subordinat. Es repeteix tantes vegades com subordinats tingui.
        \\- agent: agent que es subordina (No ha de ser subordinat de cap altre)
\end{description}
    Exemple de configuraci� texte
\begin{verbatim}
        + AgentMultiple1:
        - Accio Posicionador1
        - Accio Posicionador2
        - Accio Direccionador1
\end{verbatim}

\subsubsection{Temporitzador (Agent/Multiple/Temporitzador)}
\begin{description}
\item[Accio:] Determina un agent subordinat. Es repeteix tantes vegades com subordinats tingui.
        \\- agent: agent que es subordina (No ha de ser subordinat de cap altre)
\item[CicleActiu:] Determina quan triguen els per�odes de temps actius
        \\- uint32: per�ode m�nim
        \\- uint32: n�mero de daus
        \\- uint32: magnitud dels daus (Van de zero a la magnitud)
\item[CicleInactiu:] Determina quan triguen els per�odes de temps inactius
        \\- uint32: per�ode m�nim
        \\- uint32: n�mero de daus
        \\- uint32: magnitud dels daus (Van de zero a la magnitud)
\item[AntiAccio:] Agent subordinat especial que s'acciona en el cicle inactiu (Nomes un per temporitzador i es opcional)
        \\- agent: agent que es subordina (No ha de ser subordinat de cap altre)
\item[CicleActual:] Valors del temporitzador quan es reemprengui la marxa
        \\- id(Actiu/Inactiu): cicle actiu o inactiu
        \\- uint32: per�ode restant del cicle actual
\end{description}
    Exemple de configuraci� texte
\begin{verbatim}
        + Temporitzador1
        - Accio Posicionador3
        - CicleActiu 34 2 5
        - CicleInactiu 2 4 4
        - CicleActual 3 Inactiu
\end{verbatim}

\subsubsection{Iterador (Agent/Multiple/Iterador)}
\begin{description}
\item[Accio:] Determina un agent subordinat. Es repeteix tantes vegades com subordinats tingui.
        \\- agent: agent que es subordina (No ha de ser subordinat de cap altre)
\item[Iteracions:] Determina quantes vegades es repiteixen els subordinats
        \\- uint32: iteracions minimes
        \\- uint32: n�mero de daus
        \\- uint32: magnitud dels daus (Van de zero a la magnitud)
\item[PreAccio:] Agent subordinat especial que s'executa un sol cop abans de tot (Nomes un per iterador i es opcional)
        \\- agent: agent que es subordina (No ha de ser subordinat de cap altre)
\item[PostAccio:] Agent subordinat especial que s'executa un sol cop despres de tot (Nomes un per iterador i es opcional)
        \\- agent: agent que es subordina (No ha de ser subordinat de cap altre)
\end{description}
    Exemple de configuraci�
\begin{verbatim}
        + Iterador3
        - Accio Posicionador4
        - Accio Actuador2
        - Iteracions 20 3 6
\end{verbatim}

\subsubsection{Aleaturitzador (Agent/Multiple/Aleaturitzador)}
\begin{description}
\item[Accio:] Determina un agent subordinat. Es repeteix tantes vegades com subordinats tingui.
        \\- agent: agent que es subordina (No ha de ser subordinat de cap altre)
\item[Probabilitat:] La que hi ha d'accionar els subordinats
        \\- uint32: n�mero d'encerts que segons la probabilitat tenderien a donar-se en la mostra
        \\- uint32: n�mero de intents o mostra
\item[ReAccio:] Agent subordinat especial que s'acciona si no es dona la probabilitat (Nomes un per temporitzador i es opcional)
        \\- agent: agent que es subordina (No ha de ser subordinat de cap altre)
\end{description}
    Exemple de configuraci�
\begin{verbatim}
        + Aleaturitzador1
        - Accio Posicionador3
        - Probabilitat 20 100
\end{verbatim}

\subsubsection{Posicionador Fixe (Agent/Posicionador)}
\begin{description}
\item[Posicio:] Posici� inicial
        \\- uint32: valor de la posici� (Ha d'existir a la topologia)
\end{description}
    Exemple de configuraci�
\begin{verbatim}
        + Posicionador1
        - Posicio 12
\end{verbatim}

\subsubsection{Posicionador Aleatori (Agent/Posicionador/Aleatori)}
\begin{description}
\item[Posicio:] Posici� inicial
        \\- uint32: valor de la posici� (Ha d'existir a la topologia)
\end{description}
    Exemple de configuraci�
\begin{verbatim}
        + Posicionador2
        - Posicio 23
\end{verbatim}

\subsubsection{PosicionadorSequencial (Agent/Posicionador/Sequencial)}
\begin{description}
\item[Posicio:] Posici� inicial
        \\- uint32: valor de la posici� (Ha d'existir a la topologia)
\item[Sequencia]: Determina una posici� de la seq��ncia. Es repeteix tantes vegades com calgui.
        \\- uint32: valor de la posici� (Ha d'existir a la topologia)
\item[SequenciaActual]:
        \\- uint32: el n�mero de seq��ncia de la seg�ent posici� (Si es passa es pren l'ultim)
\end{description}
    Exemple de configuraci�
\begin{verbatim}
        + Posicionador3
        - Posicio 23
        - Sequencia 27
        - Sequencia 50
        - Sequencia 402
        - SequenciaActual 2
\end{verbatim}

\subsubsection{PosicionadorZonal (Agent/Posicionador/Zonal)}
\begin{description}
\item[Posicio:] Posicio inicial
    \\- uint32: valor de la posici� (Ha d'existir a la topologia)
\item[Posicionador:] Dona la posici� central de la zona
    \\- agent: agent posicionador (depend�ncia)
\item[Radi:] Nombre de desplacaments que pot fer la posici� entorn al centre
    \\- uint32: valor del radi
\end{description}
Exemple de configuraci�
\begin{verbatim}
        + Posicionador4
        - Posicio 23
        - Posicionador Posicionador1
        - Radi 3
\end{verbatim}

\subsubsection{Itinerari (Agent/Posicionador/Direccional)}
\begin{description}
\item[Posicio:] Posici� inicial
        \\- uint32: valor de la posici� (Ha d'existir a la topologia)
\item[Direccionador:] Dona la direcci� del desplacament
        \\- agent: agent direccionador (depend�ncia)
\item[Radi:] Nombre de desplacaments que pot fer la posici� respecte a la posici� anterior
        \\- uint32: valor del radi
\end{description}
Exemple de configuraci�
\begin{verbatim}
    + Posicionador5
    - Posicio 23
    - Direccionador Direccionador3
    - Radi 1
\end{verbatim}

\subsubsection{Direccionador (Agent/Direccionador)}
\begin{description}
\item[Direccio:] Direcci� inicial
        \\- uint32: valor de la direcci�
\end{description}
Exemple de configuraci�
\begin{verbatim}
    + Direccionador1
    - Direccio 876342
\end{verbatim}

\subsubsection{DireccionadorAleatori (Agent/Direccionador/Aleatori)}
\begin{description}
\item[Direccio:] Direcci� inicial
        \\- uint32: valor de la direcci�
\end{description}
Exemple de configuraci�
\begin{verbatim}
    + Direccionador2
    - Direccio 23442684
\end{verbatim}

\subsubsection{DireccionadorSequencial (Agent/Direccionador/Sequencial)}
\begin{description}
\item[Direccio:] Direcci� inicial
        \\- uint32: valor de la direcci�
\item[Sequencia:] Determina una direcci� de la seq��ncia. Es repeteix tantes vegades com calgui.
        \\- uint32: valor de la direcci�
\item[SequenciaActual:] Determina el punt actual de la seq��ncia
        \\- uint32: el n�mero de seq��ncia de la seg�ent direcci� (Si es passa es pren l'ultim)
\end{description}
Exemple de configuraci�
\begin{verbatim}
    + Direccionador3
    - Direccio 23
    - Sequencia 27
    - Sequencia 50
    - Sequencia 402
    - SequenciaActual 2
\end{verbatim}

\subsubsection{Nutridor (Agent/Actuador/Nutridor)}
\begin{description}
\item[Posicionador:] Dona la posici� on s'actua
        \\- agent: Agent posicionador (depend�ncia)
\item[Composicio:] Determina els elements que es depositen
        \\- uint32: element basic
        \\- uint32: variabilitat, a 1 els bits que poden variar
\end{description}
Exemple de configuraci�
\begin{verbatim}
    + Actuador1
    - Posicionador Posicionador4
    - Composicio 13152450903 0
\end{verbatim}

\subsubsection{Desnutridor (Agent/Actuador/Nutridor/Invers)}
\begin{description}
\item[Posicionador:] Dona la posici� on s'actua
        \\- agent: Agent posicionador (depend�ncia)
\item[Composicio:] Determina els elements que s'eliminen
        \\- uint32: element basic
        \\- uint32: tolerancia, a 1 els bits que no importa que coincideixin
\end{description}
Exemple de configuraci�
\begin{verbatim}
    + Actuador2
    - Posicionador Posicionador3
    - Composicio 8943742645 768764258
\end{verbatim}


\subsection{Exemple complert d'arxiu de configuraci� d'agents}

A continuaci� es presenta un exemple complert:

\begin{verbatim}
* Agent_0000 Agent/Multiple
* Agent_0002 Agent/Posicionador/Direccional
* Agent_0005 Agent/Multiple/Iterador
* Agent_0004 Agent/Actuador/Nutridor
* Agent_0003 Agent/Posicionador/Zonal
* Agent_0006 Agent/Multiple/Temporitzador
* Agent_0001 Agent/Direccionador/Aleatori

+ Agent_0002
- Posicio 1271
- Radi 1
- Direccionador Agent_0001

+ Agent_0004
- Posicionador Agent_0003
- Composicio 31 0

+ Agent_0003
- Posicio 8
- Radi 1
- Posicionador Agent_0002

+ Agent_0005
- Accio Agent_0004
- Accio Agent_0003
- Iteracions 20 0 0

+ Agent_0001
- Direccio 2192479406

+ Agent_0006
- Accio Agent_0001
- CicleActiu 1 0 1
- CicleInactiu 5 0 1
- CicleActual 4 Inactiu

+ Agent_0000
- Accio Agent_0002
- Accio Agent_0005
- Accio Agent_0006
\end{verbatim}

%%%%%%%%%%%%%%%%%%%%%%%%%%%%%%%%%%%%%%%%%%%%%%%%%%%%%%%%%%%%%%%%%%%%%%
\subsection{Disseny del abocat a disc i de la recuperaci�}
%%%%%%%%%%%%%%%%%%%%%%%%%%%%%%%%%%%%%%%%%%%%%%%%%%%%%%%%%%%%%%%%%%%%%%

TODO: Agents: Disseny del abocat a disc i de la recuperaci�

%%%%%%%%%%%%%%%%%%%%%%%%%%%%%%%%%%%%%%%%%%%%%%%%%%%%%%%%%%%%%%%%%%%%%
\newpage

%%%%%%%%%%%%%%%%%%%%%%%%%%%%%%%%%%%%%%%%%%%%%%%%%%%%%%%%%%%%%%%%%%%%%
%
%

%%%%%%%%%%%%%%%%%%%%%%%%%%%%%%%%%%%%%%%%%%%%%%%%%%%%%%%%%%%%%%%%%%%%%
\section{Serveis que donen els organismes al biosistema}
%%%%%%%%%%%%%%%%%%%%%%%%%%%%%%%%%%%%%%%%%%%%%%%%%%%%%%%%%%%%%%%%%%%%%

De cara a poder manegar
\begin{itemize}
\item
L'organisme ha d'anar proveint al biosistema de instruccions que
l'indiquin les seves accions. Com es generen les instruccions �s
q�esti� dels propis individuus.
\item
L'organisme proveeix al biosistema un conjunt de registres que formen
el seu fenotip. El biosistema pot modificar-los i consultar-los.
L'acc�s al fenotip no �s exclusiu del biosistema sin� que el propi
organisme i les seves estructures internes tamb� poden accedir-hi
paral�lelament.
\item
L'organisme ha d'implementar unes funcions vitals que modifiquin
l'estat intern de l'organisme als que no tingui acc�s el biosistema
per altres vies. Aix� no inclou, per exemple, canvis de posici�.
El biosistema ha de proporcionar tots els par�metres
de les funcions vitals que no siguin interns que ell coneixi: el
fenotip del propi organisme o un d'ali�.
\item
Tamb� han d'implementar operadors per tal de crear nous organismes a
partir d'altres i organismes aleatoris.

\end{itemize}

%%%%%%%%%%%%%%%%%%%%%%%%%%%%%%%%%%%%%%%%%%%%%%%%%%%%%%%%%%%%%%%%%%%%%
\section{Visi� general de l'estructura interna dels organismes}
%%%%%%%%%%%%%%%%%%%%%%%%%%%%%%%%%%%%%%%%%%%%%%%%%%%%%%%%%%%%%%%%%%%%%


Cariotip->Cromosoma->Base->Codo->

%%%%%%%%%%%%%%%%%%%%%%%%%%%%%%%%%%%%%%%%%%%%%%%%%%%%%%%%%%%%%%%%%%%%%
\section{El cariotip i els cromosomes}
%%%%%%%%%%%%%%%%%%%%%%%%%%%%%%%%%%%%%%%%%%%%%%%%%%%%%%%%%%%%%%%%%%%%%

Cada {\bf cromosoma} est� format per una seq��ncia de bases
representada cadascuna amb un bit o un grup de bits. Cada organisme
cont� un nombre variable de cromosomes que, en conjunt, formen el
{\bf cariotip}.

Tot i que la unitat b�sica del cromosoma, a la implementaci�, no es
considera aquesta unitat m�s que per a fer algun tipus de mutaci�
puntual. Per a la resta de manipulacions, com no fa falta arribar a
nivell de base i �s m�s �ptim accedir d,

El cromosoma, com a tal, no �s una unitat d'informaci� sin� un medi
on estan les dades gen�tiques.


El cromosoma, com s'ha dit abans �s un medi; un medi que pot tenir
errors i provocar mutacions. Dins d'un organisme, la tasa de mutaci�
de cada cromosoma �s proporcional a la seva longitud.


%%%%%%%%%%%%%%%%%%%%%%%%%%%%%%%%%%%%%%%%%%%%%%%%%%%%%%%%%%%%%%%%%%%%%
\subsection{El genotip}
%%%%%%%%%%%%%%%%%%%%%%%%%%%%%%%%%%%%%%%%%%%%%%%%%%%%%%%%%%%%%%%%%%%%%

El genotip �s la traducci� del cariotip a elements significatius.
�s el conjunt de gens que s'interpreten

%%%%%%%%%%%%%%%%%%%%%%%%%%%%%%%%%%%%%%%%%%%%%%%%%%%%%%%%%%%%%%%%%%%%%
\subsection{El fenotip}
%%%%%%%%%%%%%%%%%%%%%%%%%%%%%%%%%%%%%%%%%%%%%%%%%%%%%%%%%%%%%%%%%%%%%

El que anomenem propiament fenotip �s un conjunt de 32 registres de
32 bits que t� cada organisme. Representen el cos f�sic de l'organisme.
El fenotip es modifica per acci� directa del genotip, per� tamb�
es veu afectat pel medi mitjan�ant els sensors i, al mateix temps
afecta al medi mitjan�ant els motors. A m�s, �s un dels dos mitjans
que t�nen els organismes per reconeixer-se juntament amb la detecci�
de mol�l�cules excretades.

%%%%%%%%%%%%%%%%%%%%%%%%%%%%%%%%%%%%%%%%%%%%%%%%%%%%%%%%%%%%%%%%%%%%%
\subsection{Sensors i motors}
%%%%%%%%%%%%%%%%%%%%%%%%%%%%%%%%%%%%%%%%%%%%%%%%%%%%%%%%%%%%%%%%%%%%%

%%%%%%%%%%%%%%%%%%%%%%%%%%%%%%%%%%%%%%%%%%%%%%%%%%%%%%%%%%%%%%%%%%%%%
\subsection{Presa de nutrients}
%%%%%%%%%%%%%%%%%%%%%%%%%%%%%%%%%%%%%%%%%%%%%%%%%%%%%%%%%%%%%%%%%%%%%

%%%%%%%%%%%%%%%%%%%%%%%%%%%%%%%%%%%%%%%%%%%%%%%%%%%%%%%%%%%%%%%%%%%%%
\subsection{Metabolisme}
%%%%%%%%%%%%%%%%%%%%%%%%%%%%%%%%%%%%%%%%%%%%%%%%%%%%%%%%%%%%%%%%%%%%%

Un cop els nutrients estan dins de l'organisme, pot metabolitzar-los,
ja sigui per obtindre energia, per obtindre un producte d'excreci� o
totes dues coses.

L'organisme pot fer �s de l'energia que s'obte de les reaccions
durant un temps limitat, si no es consumeix dintre d'aquest temps,
aquesta es disipa.



Es preten que les diferents circumst�ncies permetin que hi hagi un
equilibri

Pot ser, una esp�cie tendeix a acomular masses nutrients.
Si aix� passes, els seus depredadors tenderien a augmentar en
nombre d'organismes i en nombre d'esp�cies.

%%%%%%%%%%%%%%%%%%%%%%%%%%%%%%%%%%%%%%%%%%%%%%%%%%%%%%%%%%%%%%%%%%%%%
%
%

%%%%%%%%%%%%%%%%%%%%%%%%%%%%%%%%%%%%%%%%%%%%%%%%%%%%%%%%%%%%%%%%%%%%%
\section{Mecanismes d'especiaci� i an�lisi}
%%%%%%%%%%%%%%%%%%%%%%%%%%%%%%%%%%%%%%%%%%%%%%%%%%%%%%%%%%%%%%%%%%%%%

%%%%%%%%%%%%%%%%%%%%%%%%%%%%%%%%%%%%%%%%%%%%%%%%%%%%%%%%%%%%%%%%%%%%%
\subsection{Els taxonomistes}
%%%%%%%%%%%%%%%%%%%%%%%%%%%%%%%%%%%%%%%%%%%%%%%%%%%%%%%%%%%%%%%%%%%%%

Per que l'usuari pugui extreure una informaci� �til, cal que poguem
agrupar els organismes en esp�cies. �s clar que a la nostra comunitat,
al igual que a la natura, l'especiaci� �s un fen�men que cal que
emergeixi, L'esp�cie, no �s quelcom intr�nsec a tot organisme; �s a
dir, que el concepte d'esp�cie no estar� implementat al codi gen�tic
o als mecanismes de funcionament comuns dels organismes sin� que
caldr� observar-ho en el seu comportament.

Els taxonomistes s�n objectes que agrupen organismes per afinitat
evolutiva segons els conceptes abans mencionats.

Com que, abans de definir objectius, em plantejava implementar
organismes amb reproducci� sexual, vaig construir un taxonomista que
soportava tota la complexitat que comporten els creuaments i que he
mencionat abans.

Com finalment, els intercanvis sexuals es van excloure dels objectius 
del projecte, les relacions evolutives es limiten a un arbre i vaig 
implementar un taxonomista molt m�s senzill, que simplement discrimina 
organismes amb diferent cromosoma quan es produeix una mutaci�, per�, 
que compleix el mateix protocol que el m�s elaborat implementat 
anteriorment de forma que es podrien intercanviar.

El taxonomista simple �s el que he fet servir a les proves donat que
era molt m�s r�pid i la informaci� que dona ja �s suficient per la
complexitat dels sistemes generats. 

Tot i aix�, documento el taxonomista per organismes sexuals de cara 
a ilustrar millor el protocol i donat que ser� �til en futures 
ampliacions del sistema si s'afegeixen creuaments.

%%%%%%%%%%%%%%%%%%%%%%%%%%%%%%%%%%%%%%%%%%%%%%%%%%%%%%%%%%%%%%%%%%%%%
\subsection{Qu� es vol solucionar}
%%%%%%%%%%%%%%%%%%%%%%%%%%%%%%%%%%%%%%%%%%%%%%%%%%%%%%%%%%%%%%%%%%%%%

El concepte cl�ssic d'esp�cie considera que dos organismes s�n de la
mateixa esp�cie si s�n capa�os de donar descenc�ncia f�rtil.

A la biologia moderna es considera que la diferenciaci� de les esp�cies
no est� tant en la capacitat sin� en el fet mateix de reproduir-se.
En aix� pot influir:
\begin{itemize}
\item   Compatibilitat gen�tica
\item   Compatibilitat d'acoplament
\item   Compatibilitat geogr�fica
\item   Altres
\end{itemize}

Tamb�, cal adonar-se de que totes aquestes consideracions es
refereixen als organismes que es reprodueixen sexualment i es creuen.
C�m es poden identificar les esp�cies dels organismes que es
reprodueixen asexualment (com �s el cas implementat)? En quin punt es 
considera que dos descendents d'un mateix organisme s�n d'una 
esp�cie diferent?

A m�s, degut a la seva qualitat emergent, l'especiaci� no estar�
sempre ben definida. Tamb� pot ser que la selecci� a l'hora de
reproduir-se tingui en compte altres mecanismes que no pas
l'especiaci�.

Tota aquesta conjuntura ha obligat a deixar de banda el concepte
d'esp�cie cap al concepte, una mica m�s relaxat, de grup reproductiu
que, tot i la relaxaci�, segueix proporcionant informaci� �til sobre
les diferents poblacions del biosistema. Considerem que un grup
reproductiu �s un conjunt d'organismes que es creuen entre s� o
provenen d'ancestres comuns o ancestres que s'han creuat entre s� 
dintre d'un cert per�ode de temps o d'un cert nombre de generacions.

%%%%%%%%%%%%%%%%%%%%%%%%%%%%%%%%%%%%%%%%%%%%%%%%%%%%%%%%%%%%%%%%%%%%%
\subsubsection{Taxonomista d'organismes sexuals}
%%%%%%%%%%%%%%%%%%%%%%%%%%%%%%%%%%%%%%%%%%%%%%%%%%%%%%%%%%%%%%%%%%%%%

La pol�tica que determina els grups reproductius es basa en un marcatge
hist�ric.

Cada individu s'associa amb un tax� que no �s m�s que una seq��ncia de
marques amb diferent antiguitat. Les marques es traspasen id�ntiques a
la descend�ncia via mitosi.

Cada cert temps, es fa una discriminaci� que consisteix en fondre les
dos marques m�s antigues de cada tax� en una sola i afegir una marca
nova que diferenciar� els individus que fins llavors compartien les
mateixes marques i que pendr� import�ncia a mida que adquireixi antiguitat.

C�m es produir� aquesta discriminaci�? Imaginem que existeixen individus
amb els taxons seg�ents:
\begin{center}
\begin{tabular}{p{45 mm}cp{45 mm}cp{45 mm}}
\begin{center}
\begin{tabular}{l}
A A A A A A\\
A A A A A A\\
A B A A A A\\
A B B A A A\\
A B B A A A\\
B A A A A A
\end{tabular}
\end{center}
&&
\begin{center}
\begin{tabular}{l}
{\bf A} A A A A\\
{\bf A} A A A A\\
{\bf B} A A A A\\
{\bf B} B A A A\\
{\bf B} B A A A\\
{\bf C} A A A A
\end{tabular}
\end{center}
&&
\begin{center}
\begin{tabular}{l}
A A A A A {\bf A}\\
A A A A A {\bf B}\\
B A A A A {\bf A}\\
B B A A A {\bf A}\\
B B A A A {\bf B}\\
C A A A A {\bf A}
\end{tabular}
\end{center}
\\Taxons originals dels organismes abans de la discretitzaci� de la poblaci�
&$\Rightarrow$
&Fusi� dels dos taxons m�s antics
&$\Rightarrow$
&Discriminaci� dels individus amb les mateixes marques amb una nova:
\end{tabular}
\end{center}

D'altra banda, hem de considerar el que passa quan es creuen dos
individus que pertanyen a diferents taxons.

Quan dos individus es creuen, s'asimilen les marques des de la marca
m�s antiga fins a la primera que els diferencia als dos (marca discriminant). Es a dir, a tots els taxons cal revisar les marques. Pot ser es veu m�s clar amb un exemple. Considerem el seg�ent conjunt de taxons.

\begin{tabular}{l}
A A A A A A\\
A A A A A B\\
A A A A B B\\
A A A B A B\\
A A A B B A\\
A A A B B B\\
A A B A A A
\end{tabular}

Si es creuen AAAABB i AAABBA, cal considerar equivalemts les
subseq�encies de marques AAAA i AAAB equivalents.
Tots els taxons que comencin per AAAB els canviem per AAAA sense
oblidar-nos de modificar la seg�ent marca m�s jove que la discriminant
amb l'objectiu de que els taxons asimilats mantinguin el sentit.

\begin{tabular}{lcl}
A A A A A A&$\dashrightarrow$&A A A A A A\\
A A A A A B&$\dashrightarrow$&A A A A A B\\
A A A A B B&$\dashrightarrow$&A A A A B B\\
A A A B A B&$\Rightarrow$&A A A {\em A C} B\\
A A A B B A&$\Rightarrow$&A A A {\em A D} A\\
A A A B B B&$\Rightarrow$&A A A {\em A D} B\\
A A B A A A&$\dashrightarrow$&A A B A A A
\end{tabular}

De cara a la implementaci�, he considerat separar la major part
del proc�s associat a la determinaci� de grups reproductius en
un objecte independent anomenat taxonomista. Aquest objecte s'encarrega de mantenir els taxons al dia mitjan�ant una interf�cie estreta que mant� amb el processador.

A dins de la comunitat es mant� una informaci� m�nima: l'identificador del tax� al que pertany cada individu. El tr�fec d'informaci� a l'exterior del objecte taxonomista es basa exclusivament en el pas d'aquests identificadors. La interf�cie oferida permet al processador:

\begin{itemize}
\item   Incrementar o decrementar la poblacio assignada a un tax�
\item   Creuar un parell de taxons
\item   Generar un nou tax� (Per individus generats espont�niament)
\item   Envellir les marques i discriminar la poblaci� que comparteix el mateix tax�
\item   Determinar el grau de parentesc entre dos individus
\end{itemize}

Quan hem de discriminar o quan fussionem dos taxons, necessitem
que la Comunitat i el Taxonomista cooperin.

Quan cal discriminar la poblaci�, la Comunitat demana al Taxonomista,
per a cada individu, un nou tax�, basant-se en el tax� antic i el
n�mero de individus que en queden sense discriminar d'aquest.

El nou taxo duu les marques X.X.X. ... .X.N on N �s el n�mero de
queden sense discriminar.

Quan, fruit d'un creuament, es fusionen dos taxons, un dels dos
taxons �s assimilat per l'altre i, en conseq��ncia, els individus
associats al tax� assimilat, cal associar-los al tax� assimilador.

Per dins, el taxonomista est� compost per una llista indexada de
taxons (taxonari). Els n�meros d'�ndex es referencien des de cada
individu pertanyent a la Comunitat.

Una alternativa al taxonari hagu�s sigut una implementaci� en arbre
en comptes de la llista indexada. En cada node hi hauria una marca
i a les fulles els identificadors de cada tax�. La implementaci� en
arbre simplifica molt la l�gica dels algorismes de discriminaci� i
creuament per� complica altres operacions internes que amb la llista
indexada s�n trivials.

La llista indexada permet un acc�s directe als taxons i, a m�s, els
mant� ordenats de tal forma que les cerques de grups de parentesc
tenen un cost temporal m�nim.



%%%%%%%%%%%%%%%%%%%%%%%%%%%%%%%%%%%%%%%%%%%%%%%%%%%%%%%%%%%%%%%%%%%%%
\newpage

%%%%%%%%%%%%%%%%%%%%%%%%%%%%%%%%%%%%%%%%%%%%%%%%%%%%%%%%%%%%%%%%%%%%%
%
%

%%%%%%%%%%%%%%%%%%%%%%%%%%%%%%%%%%%%%%%%%%%%%%%%%%%%%%%%%%%%%%%%%%%%%
\chapter{Coordinaci� del biosistema}
%%%%%%%%%%%%%%%%%%%%%%%%%%%%%%%%%%%%%%%%%%%%%%%%%%%%%%%%%%%%%%%%%%%%%

\index{biosistema}
L'objecte {\tt CBiosistema} cont� i coordina tots els elements que 
formen part del nucli del simulador.

El cicle intern del biosistema �s una funci� que un controlador pot
executar de forma iterativa. A continuaci� es detallen les accions 
que el biosistema pot fer durant una iteraci�.

%%%%%%%%%%%%%%%%%%%%%%%%%%%%%%%%%%%%%%%%%%%%%%%%%%%%%%%%%%%%%%%%%%%%%
\section{Manteniment de la poblaci� m�nima i la variabilitat gen�tica}
%%%%%%%%%%%%%%%%%%%%%%%%%%%%%%%%%%%%%%%%%%%%%%%%%%%%%%%%%%%%%%%%%%%%%

El biosistema t� la capacitat de generar organismes aleatoris i 
inocular-los al sistema. 

A cada cicle, el biosistema prova d'inocular organismes aleatoris en 
tres casos:
\begin{itemize}
\item Quan es dona una probabilitat configurada
\item Quan la poblaci� no arriba a un cert m�nim configurat
\item Quan no hi ha poblaci�
\end{itemize}

El procediment per fer-ho no sempre �s exit�s donat que es poden 
donar condicions per que la inoculaci� no sigui correcta, per 
exemple, si la posici� del bi�top escollida �s ocupada, com passaria 
segur si no hi hagu�s cap posici� del biosistema lliure.

%%%%%%%%%%%%%%%%%%%%%%%%%%%%%%%%%%%%%%%%%%%%%%%%%%%%%%%%%%%%%%%%%%%%%
\section{Control del qu�ntum i canvi de context}
%%%%%%%%%%%%%%%%%%%%%%%%%%%%%%%%%%%%%%%%%%%%%%%%%%%%%%%%%%%%%%%%%%%%%

Com que es vol simular un sistema paral�lel en una m�quina seq�encial
haurem d'implementar t�cniques de temps compartit.
El qu�ntum �s el nombre d'instruccions d'un organisme que s'executen
seguides abans de canviar a un altre organisme.

\index{temps compartit}
\index{qu�ntum}
Sovint les t�cniques de temps compartit introdueixen artificis en els 
sistemes de simulaci� i, en concret, de vida artificial. Amb l'objectiu 
de dispersar el seu efecte hem introdu�t les seg�ents t�cniques:
\begin{itemize}
\item No intercanviar els organismes de forma ordenada, sino aleat�riament.
\item Fer servir un quantum despreciable, (si no pot ser 1), respecte les instruccions que necessita l'organisme per reproduir-se.
\item Introduir cert indeterminisme en la determinaci� del qu�ntum.
\end{itemize}

L'encarregat de fer el ``canvi de context'' �s un m�tode del 
biosistema que t� com a postcondici� surtir sempre amb un organisme 
com actual. 
Fins i tot, si la comunitat �s buida, en genera un organismes 
aleatoris fins que ho deixi de ser.

Si, per un costat, es bo fer petit el qu�ntum per dispersar els 
artificis, fer el qu�ntum petit tamb� carrega m�s el sistema doncs
cal fer m�s canvis de context.

Tamb�, de cara a visualitzar les instruccions pas a pas, a vegades
tamb� va b� fer el qu�ntum un xic gran per veure m�s instruccions
juntes d'un mateix organisme i entendre millor el seu sentit.

%%%%%%%%%%%%%%%%%%%%%%%%%%%%%%%%%%%%%%%%%%%%%%%%%%%%%%%%%%%%%%%%%%%%%
\section{Defuncions}
%%%%%%%%%%%%%%%%%%%%%%%%%%%%%%%%%%%%%%%%%%%%%%%%%%%%%%%%%%%%%%%%%%%%%

\index{energia!defunci�}
\index{defunci�}
Si l'organisme actual no t� energia, s'executa el procediment de
defunci�. Aquest fa tres accions:

\begin{itemize}
\item Buidar la posici� del bi�top on es trobi l'organisme.
\item Indicar al taxonomista que un organisme de determinat tax� ha mort.
\item Extreure l'organisme de la comunitat.
\end{itemize}

Despr�s d'una defunci�, evidentment, cal canviar a un nou organisme
amb la funci� de canvi de context i tornar a comprovar si aquest nou
organisme tamb� est� sense energia.

%%%%%%%%%%%%%%%%%%%%%%%%%%%%%%%%%%%%%%%%%%%%%%%%%%%%%%%%%%%%%%%%%%%%%
\section{Expedici� d'instruccions}
%%%%%%%%%%%%%%%%%%%%%%%%%%%%%%%%%%%%%%%%%%%%%%%%%%%%%%%%%%%%%%%%%%%%%

Un cop tenim un organisme amb suficient quantum i prou energia per 
continuar, se li demana una instrucci� i s'executa. L'apartat 
\ref{sec:cjtInstruccions} detalla una mica m�s com s'executen les
instruccions.

%%%%%%%%%%%%%%%%%%%%%%%%%%%%%%%%%%%%%%%%%%%%%%%%%%%%%%%%%%%%%%%%%%%%%
\section{Temps simulat}
%%%%%%%%%%%%%%%%%%%%%%%%%%%%%%%%%%%%%%%%%%%%%%%%%%%%%%%%%%%%%%%%%%%%%

De cara a mesurar el temps transcorregut en el sistema, la idea m�s
simple �s comptar el cicles del biosistema, aix� �s, el nombre 
d'instruccions executades.

Per�, es vol trobar una mesura del temps que, d'alguna forma, sigui 
regular des del punt de vista dels organismes. Regular des del punt 
de vista de l'organisme vol dir, que un organisme, en $n$ unitats de 
temps un organisme executi sempre, aproximadament, $i$ instruccions.

El problema que porta considerar els cicles del biosistema com a
unitat de temps �s que la poblaci� varia durant la simulaci�.
Com que la poblaci� varia al llarg del temps, donat un nombre $n$
d'instruccions executades entre tots els organismes, no es mant� una 
proporci� $p$ m�s o menys constant d'instruccions d'un mateix 
organisme.

Aix� doncs, si comptem el nombre de cicles, des del punt de vista
dels organismes es veur� que, quan hi ha m�s poblaci�, el temps passa
m�s poc a poc i, quan hi ha menys poblaci�, el temps passa m�s r�pid.

El problema ve del fet que, en la realitat, un nombre variable 
d'organismes executarien instruccions en paral�lel, i, en aquesta 
simulaci� tots els temps es seq�encien com indica el gr�fic seg�ent.
En ser una poblaci� variable, els intervals oscilen.

A la figura \ref{fig:disipacio}, anomenem {\em temps real o 
temps seq�encial}\index{temps seq\"{u}encial} al temps que es pot 
mesurar en cicles del biosistema i que �s el que nosaltres percebem 
per la pantalla, i {\em temps simulat o temps paral�lel}
\index{temps paral�lel} al temps que veuen els organismes que
�s el que veurien en la realitat.

\begin{figure}[ht]
\centering
%   \includegraphics[height=3 in, keepaspectratio]{temps}
    \fbox{\pdfimage width 6in {temps.png}}
    \caption{Temps seq�encial i temps paral�lel}
    \label{fig:temps}
\end{figure}

Al esquema seq�encial de la figura \ref{fig:disipacio} estem imaginant 
que el model de compartici� de temps entre els organismes seq�encia en 
ordre els organismes de la comunitat. En aquest cas, es podria mesurar
f�cilment el temps paral�lel considerant com unitat de temps el que 
triga el sistema en atendre a cadascun dels organismes.

A Bioscena no es pot fer exactament aquesta soluci�, donat que s'ha 
escollit seq�enciar els organismes de forma aleat�ria per disipar els 
artificis de la seq�encialitat. Per�, podem fer una aproximaci� 
bastant equivalent que consisteix en comptar la poblaci� en un 
determinat instant i esperar tants cicles del biosistema com aquest
n�mero per incrementar el temps.

Cal tenir en compte que aquesta mesura no �s gaire perfecte durant
per�odes de fluctuacions en la poblaci� donat que la poblaci� la
estem mirant, nom�s en un instant.

%%%%%%%%%%%%%%%%%%%%%%%%%%%%%%%%%%%%%%%%%%%%%%%%%%%%%%%%%%%%%%%%%%%%%
\section{Accionament dels agents ambientals}
%%%%%%%%%%%%%%%%%%%%%%%%%%%%%%%%%%%%%%%%%%%%%%%%%%%%%%%%%%%%%%%%%%%%%

\index{agents ambientals!accionament}
El biosistema acciona l'arbre d'agents ambientals que actuen sobre ell
cada unitat de temps simulat. D'aquesta forma els organismes i els
agents ambientals parlen una mateixa llengua en q�esti� de temps.

Com a mitja es d�na que, per cada vegada que s'accionen els agents, 
cada organisme ha executat una instrucci�. 



%%%%%%%%%%%%%%%%%%%%%%%%%%%%%%%%%%%%%%%%%%%%%%%%%%%%%%%%%%%%%%%%%%%%%
\newpage


%%%%%%%%%%%%%%%%%%%%%%%%%%%%%%%%%%%%%%%%%%%%%%%%%%%%%%%%%%%%%%%%%%%%%
%
%

%%%%%%%%%%%%%%%%%%%%%%%%%%%%%%%%%%%%%%%%%%%%%%%%%%%%%%%%%%%%%%%%%%%%%
\chapter{Conjunt d'instruccions}
\label{sec:cjtInstruccions}
%%%%%%%%%%%%%%%%%%%%%%%%%%%%%%%%%%%%%%%%%%%%%%%%%%%%%%%%%%%%%%%%%%%%%

El biosistema pot configurar quins opcodes es relacionen amb quines
operacions mitjan�ant un arxiu de configuraci� especial. Mitjan�ant
aquesta relaci�, el biosistema pot saber, a partir dels n�meros 
d'instruccions que ofereix l'organisme, quina �s la instrucci� que
cal executar.

Aquest apartat detalla com s'executen les diferents instruccions que
pot executar un biosistema. Els par�metres de les instruccions, s�n 
nibbles de 4 bits que generalment indexen un dels 16 registres de 
32 bits que formen fenotip de l'organisme actual, d'on s'extreu o a 
on s'escriu el valor.

%%%%%%%%%%%%%%%%%%%%%%%%%%%%%%%%%%%%%%%%%%%%%%%%%%%%%%%%%%%%%%%%%%%%%
\section{Instruccions fenot�piques}
%%%%%%%%%%%%%%%%%%%%%%%%%%%%%%%%%%%%%%%%%%%%%%%%%%%%%%%%%%%%%%%%%%%%%

El seg�ent grup d'instruccions, s�n instruccions en les que nom�s 
interv� el fenotip. Ni el sistema metab�lic ni els altres elements
del biosistema. S�n operacions que comencen i acaben al fenotip.

\begin{itemize}
\item {\tt And dest op1 op2}: Fica a {\tt dest} la and bit a bit de {\tt op1} i {\tt op2}
\item {\tt Or dest op1 op2}: Fica a {\tt dest} la or bit a bit de {\tt op1} i {\tt op2}
\item {\tt Xor dest op1 op2}: Fica a {\tt dest} la xor bit a bit de {\tt op1} i {\tt op2}
\item {\tt Not dest op1}: Fica a {\tt dest} la negaci� bit a bit de {\tt op1}
\item {\tt Oposa dest op1}: Fica a {\tt dest} el despla�ament en sentit contrari al despla�ament expressat per {\tt op1} (veure \ref{tab:veinesDirectes})
\item {\tt Carrega dest}: Omple el registre amb el valor de la seg�ent instrucci�
\item {\tt Random dest}: Omple el registre amb un valor al atzar
\item {\tt ShiftL dest op1 num}: dest=op1$<<$num (num �s directament el nibble, no pas el valor del registre indexat pel nibble)
\item {\tt ShiftR dest op1 num}: dest=op1$>>$num (num �s directament el nibble, no pas el valor del registre indexat pel nibble)
\end{itemize}

%%%%%%%%%%%%%%%%%%%%%%%%%%%%%%%%%%%%%%%%%%%%%%%%%%%%%%%%%%%%%%%%%%%%%
\section{Instruccions sensorials}
%%%%%%%%%%%%%%%%%%%%%%%%%%%%%%%%%%%%%%%%%%%%%%%%%%%%%%%%%%%%%%%%%%%%%

S�n les instruccions que modifiquen el fenotip segons algun element
del biosistema:
\begin{itemize}
\item {\tt SensorQ}: Sensor qu�mic (nutrients al medi)
\item {\tt SensorP}: Sensor de pres�ncia (altres organismes)
\item {\tt SensorI}: Sensor intern (nutrients al pap i estat energ�tic)
\end{itemize}

Tant el sensor qu�mic com el de pres�ncia, cerquen una posici� en 
una zona del bi�top que acompleixi una condici�. El primer cerca 
per un nutrient que tingui una clau compatible amb el patr� i la 
toler�ncia indicades com a par�metres. 
El segon cerca un organisme que, al registre indicat, tingui un 
contingut compatible amb el patr� i la toler�ncia indicades.

Totes dos indiquen una posici� relativa que marca el centre de la
zona de cerca i un radi que indica el n�mero m�xim de salts als que
es pot trobar l'objectiu.

Es comprova un nombre configurable de vegades una posici� aleat�ria
dintre d'aquesta zona. Si la posici� compleix la condici�, als dos
registres de dest� es posen el valor trobat i el despla�ament 
relatiu que em dirigeix a aquesta posici�.

El sensor intern simplement posa, a un registre dest�, la clau del 
nutrient dins del pap que �s compatible amb el patr� i la toler�ncia 
donades, si n'hi ha cap, i, a un altre, l'energia total actual.

%%%%%%%%%%%%%%%%%%%%%%%%%%%%%%%%%%%%%%%%%%%%%%%%%%%%%%%%%%%%%%%%%%%%%
\section{Instruccions motores}
%%%%%%%%%%%%%%%%%%%%%%%%%%%%%%%%%%%%%%%%%%%%%%%%%%%%%%%%%%%%%%%%%%%%%

Les instruccions motores s�n les que modifiquen altres coses que no
nom�s el fenotip. En resum s�n:
\begin{itemize}
\item {\tt Ingestio}: Incorpora al pap nutrients lliures al medi
\item {\tt Excrecio}: Allibera al medi nutrients del pap
\item {\tt Moviment}: Mou l'organisme a una posici� donada
\item {\tt Mitosi}: Crea un fill de l'organisme
\item {\tt Agressio}: Manlleva nutrients d'un altre organisme
\item {\tt Catabol}: Parteix un nutrient del pap en dos (reacci� ex�gena)
\item {\tt Anabol}: Fusiona dos nutrients del pap en un (reacci� end�gena)
\end{itemize}

Les instruccions motores poden fallar. Quan ho fan, tenen un cost 
adicional. Aquest cost adicional per fallada �s com� a totes les
instruccions motores i es pot configurar.

En principi, en el sistema proposat, l'energia �til s'extreu nom�s 
de reaccions metab�liques. Per�, amb l'objectiu de simular entorns 
m�s senzills (sense metabolisme), �s possible associar un guany 
energ�tic a la ingesti�. Tamb� ho hem est�s a la resta de funcions
motores i sensores, de forma que es permet associar guanys i costos
extres d'energia a cada instrucci�.

%%%%%%%%%%%%%%%%%%%%%%%%%%%%%%%%%%%%%%%%%%%%%%%%%%%%%%%%%%%%%%%%%%%%%
\subsection{Ingesti�}
%%%%%%%%%%%%%%%%%%%%%%%%%%%%%%%%%%%%%%%%%%%%%%%%%%%%%%%%%%%%%%%%%%%%%

Per introduir un nutrient lliure al medi dins del cos d'un organisme,
cal precissar el lloc d'on vol extreure-ho, un patr� i una toler�ncia.
Si existeix l'element que compleixi aquests requisit, el nutrient �s 
introdu�t a l'interior de l'organisme.

En principi, en el sistema proposat, l'energia �til s'extreu nom�s 
de reaccions metab�liques. Per�, amb l'objectiu de simular entorns 
m�s senzills (sense metabolisme), �s possible associar un guany 
energ�tic a la ingesti�.

%%%%%%%%%%%%%%%%%%%%%%%%%%%%%%%%%%%%%%%%%%%%%%%%%%%%%%%%%%%%%%%%%%%%%
\subsection{Excreci�}
%%%%%%%%%%%%%%%%%%%%%%%%%%%%%%%%%%%%%%%%%%%%%%%%%%%%%%%%%%%%%%%%%%%%%

L'excreci� mou un nutrient (especificat per un patr� i una toler�ncia)
del pap cap al medi a una posici� indicada.

Aquesta instrucci� s'ha introduit donat que �s una forma en que
l'organisme pot modificar el medi. �s possible que, al llarg de 
l'evoluci� doni peu a comportaments complexos com el seg�ents:
\begin{itemize}
\item Proporcionar a la descend�ncia nutrients via excreci�.
\item Detectar als membres d'una esp�cie o el seu estat per les
mol�lecules excretades.
\item Transport de nutrients.
\item ...
\end{itemize}

El caracter obert que poden adoptar les soluccions dels organismes
implica que la llista anterior pot no ser tancada, o pot ser els
organismes no arribin a cap dels punts anteriors.

De la mateixa manera que la ingesti� es pot associar un guany 
energ�tic adicional, l'excrecci� es pot configurar amb un guany
o amb un cost adicional.


%%%%%%%%%%%%%%%%%%%%%%%%%%%%%%%%%%%%%%%%%%%%%%%%%%%%%%%%%%%%%%%%%%%%%
\subsection{Agressi� i defensa}
%%%%%%%%%%%%%%%%%%%%%%%%%%%%%%%%%%%%%%%%%%%%%%%%%%%%%%%%%%%%%%%%%%%%%

Un organisme, de banda d'obtindre nutrients del medi, tamb� els pot 
obtindre d'altres organismes. Per fer-ho, nom�s cal indicar una 
posici� on suposadament hi ha un altre organisme, un element base
i una toler�ncia, i una clau d'atac. Aquesta clau d'atac, en 
enfrentar-la a la clau de defensa de la victima en resulta la for�a
final de l'atac. Aquesta for�a indica el nombre de vegades que es 
provar� d'extreure un nutrient del pap de la victima segons l'element
base i la toler�ncia.

Com es pot deduir, d'un atac es poden obtindre molts nutrients al
mateix temps la qual cosa la fa m�s profitosa que la ingestio de
nutrients del medi. Aix� prova de compensar els desavantatges de 
desenvolupar una conducta depredadora i haver de dependre de les
preses.

Es pot associar un guany energ�tic proporcional al nombre de 
nutrients extrets a un altre organisme. De la mateixa forma tamb� 
es pot associar un cost energ�tic proporcional per a la victima.

%%%%%%%%%%%%%%%%%%%%%%%%%%%%%%%%%%%%%%%%%%%%%%%%%%%%%%%%%%%%%%%%%%%%%
\subsection{Moviment}
%%%%%%%%%%%%%%%%%%%%%%%%%%%%%%%%%%%%%%%%%%%%%%%%%%%%%%%%%%%%%%%%%%%%%

La instrucci� per moure's �s ben simple, nom�s cal indicar una
posici� dest�. Es pot realitzar si la posici� est� lliure. T� 
associada un cost configurable.

%%%%%%%%%%%%%%%%%%%%%%%%%%%%%%%%%%%%%%%%%%%%%%%%%%%%%%%%%%%%%%%%%%%%%
\subsection{Mitosi}
%%%%%%%%%%%%%%%%%%%%%%%%%%%%%%%%%%%%%%%%%%%%%%%%%%%%%%%%%%%%%%%%%%%%%

La instrucci� {\tt Mitosi} crea un fill de l'organisme a una posici�
indicada. Aquesta posici� ha de estar lliure, si no, l'instrucci�
no s'executa.

Es crea el cos del nou organisme a partir del cariotip del progenitor, 
amb una certa probabilitat de mutaci�. Es situa al bi�top, i es 
demana per un identificador de taxo al taxonomista tot aportant el
identificador de tax� del pare i indicant si ha mutat o no.

En el moment en el que s'introdueix l'organisme en la comunitat, 
aquest �s elegible pel biosistema per �sser executat.

La mitosi t� molts factors energ�tics configurables, donat que �s
una de les instruccions clau quan es vol equilibrar els costos per
obtindre comportaments complexos.

Es requereix que hi hagi una certa energia disponible abans de
reproduir-se. Si no es pos�s un l�mit d'aquest tipus, no seria
necessari tenir una vida llarga per reproduir-se i amb una vida
excessivament curta (en algunes proves els organismes arribaven a 
reproduir-se en les cinc instruccions que se'ls hi donaven de quantum), 
no hi ha lloc per a trobar comportaments interesants.

Llavors hi ha un m�nim d'energia per poder iniciar la reproducci�. 
Aquesta energia m�nima no �s el cost real de la instrucci�, sin� que 
aquest �s configurable independentment.

Una altra cosa que cal que sigui configurable �s l'energia �til amb 
la que comen�a el nou organisme. Per ser consistents, �s important 
que sigui bastant menor que el cost de reproduir-se perqu� sin� els 
organismes no tindrien perqu� menjar.

Tamb� s'especifica els nutrients del pap que passen a la descend�ncia 
amb un patr�, una toler�ncia i un n�mero d'intents.

�s possible penalitzar en aquesta instrucci� al pares que tinguin
un cariotip superior a un nombre de codons. Aquesta penalitzaci�
es proporcional a la quantitat que es passa.

Aquesta penalitzaci� s'ha introdu�t per corregir l'efecte que 
produien alguns operadors de mutaci� que sovint incrementaven el 
tamany del cariotip afegint m�s redund�ncia de la que era necess�ria
per mantenir la variabilitat.
Els cariotips grans afecten en gran mesura la velocitat i la 
mem�ria ocupada pel sistema. Per�, tampoc semblava correcte limitar 
el tamany restrictivament perqu� algun cop pot arribar a ser �til 
aquest increment. Aix� doncs s'ha adoptat la soluci� tamb� present 
a la natura: Un cariotip excessivament llarg costa m�s de replicar
que un de curt. Aix� possibilita permetre els llargs sempre que
serveixin per codificar millors solucions.

%%%%%%%%%%%%%%%%%%%%%%%%%%%%%%%%%%%%%%%%%%%%%%%%%%%%%%%%%%%%%%%%%%%%%
\subsection{Anabolisme}
%%%%%%%%%%%%%%%%%%%%%%%%%%%%%%%%%%%%%%%%%%%%%%%%%%%%%%%%%%%%%%%%%%%%%

L'anabolisme que implementa aquest prototip extreu dos nutrients
del pap (mitjan�ant els respectius patrons i toler�ncies) i els junta 
per formar un tercer, donant un balanc d'energia negatiu.

\(A + B \rightarrow^{-E} C\)

\begin{verbatim}
   C = A & B
   cost = PAnabol * (comptaUns(C)- min(comptaUns(A),comptaUns(B)))
\end{verbatim}

A la f�rmula, {\tt PAnabol} representa un factor configurable.


%%%%%%%%%%%%%%%%%%%%%%%%%%%%%%%%%%%%%%%%%%%%%%%%%%%%%%%%%%%%%%%%%%%%%
\subsection{Catabolisme}
%%%%%%%%%%%%%%%%%%%%%%%%%%%%%%%%%%%%%%%%%%%%%%%%%%%%%%%%%%%%%%%%%%%%%

El catabolisme que implementa aquest prototip extreu un nutrient
del pap (mitjan�ant un patr� i una toler�ncia) i obt� dos nutrients
mitjan�ant una clau catab�lica.

\(A \rightarrow^{+E} B + C\)

El catabolisme requereix, a m�s una clau catab�lica $T$ que es fa servir
per calcular els productes.

\begin{verbatim}
	B=A & T;
	C=A & ~T;
	energia = PCatabol * (comptaUns(A)- max(comptaUns(B),comptaUns(C)))
\end{verbatim}

A la f�rmula, {\tt PCatabol} representa un factor configurable.


%%%%%%%%%%%%%%%%%%%%%%%%%%%%%%%%%%%%%%%%%%%%%%%%%%%%%%%%%%%%%%%%%%%%%
\newpage


%\input{interficie}


%%%%%%%%%%%%%%%%%%%%%%%%%%%%%%%%%%%%%%%%%%%%%%%%%%%%%%%%%%%%%%%%%%%%%%
\chapter{Proves i resultats}
%%%%%%%%%%%%%%%%%%%%%%%%%%%%%%%%%%%%%%%%%%%%%%%%%%%%%%%%%%%%%%%%%%%%%%

%%%%%%%%%%%%%%%%%%%%%%%%%%%%%%%%%%%%%%%%%%%%%%%%%%%%%%%%%%%%%%%%%%%%%%
\section{Proves preliminars}
%%%%%%%%%%%%%%%%%%%%%%%%%%%%%%%%%%%%%%%%%%%%%%%%%%%%%%%%%%%%%%%%%%%%%%

Durant la progressiva construcci� del sistema, s'han avaluat les
aportacions que anava fent cada nou element que s'hi afegia.
A continuaci�, s'explica els fets m�s significatius que es varen
observar en cadascuna de les etapes.

Cal notar que en aquestes etapes el guany energ�tic es fa per simple 
ingesti�, doncs no estaven implementades encara les reaccions
metab�liques i les instruccions {\tt anabol} i {\tt catabol}.

%%%%%%%%%%%%%%%%%%%%%%%%%%%%%%%%%%%%%%%%%%%%%%%%%%%%%%%%%%%%%%%%%%%%%%
\subsection{Exemple 1: Comportament aleatori}
%%%%%%%%%%%%%%%%%%%%%%%%%%%%%%%%%%%%%%%%%%%%%%%%%%%%%%%%%%%%%%%%%%%%%%

En aquest punt, el sistema de control dels organismes expedia, 
simplement, instruccions aleat�ries. Els par�metres, en comptes 
d'agafar-los del fenotip, tamb� eren generats aleat�riament.
Es va fer la prova de cara a comprovar la consist�ncia del sistema, 
independentment del sistema de control. Funcionaven nom�s les
quatre instruccions: {\tt ingestio}, {\tt moviment}, {\tt agressio} i 
{\tt mitosi}. Tamb� es varen relaxar les condicions de compatibilitat 
limitant les claus a 3 bits.

Per aquest primer sistema, es va preparar un bi�top de 30x30 on uns
agents externs deixaven caure nutrients en densitat molt baixa 
aleatoriament per tot el m�n. Altres agents externs en dipositaven 
massivament dins d'una zona de radi 8.

Tot i que no hi havia cap tipus d'her�ncia ni comportament definit,
es sabia que l'�nica cosa que s'heretaven els fills era la proximitat
geogr�fica al seu progenitor. Aix�, s'esperava que, com els organismes
que sobreviurien serien els que estaven a la taca de nutrients al final
hi hagu�s un c�mul d'organismes concentrats.

Per sorpresa, no va ser exactament aix�, sino que es varen disposar
fent un anell al voltant de la taca i un parell o tres es disposaven
just en mig.

Un ex�men m�s detallat de la situaci� va donar l'explicaci�: Els 
organismes que es posaven entre el centre i la vora de la taca eren 
atacats aleat�riament pels altres, i nom�s a les voreres, tenien menys
probabilitat de ser atacats. I els del centre, simplement, quedaven 
fora de l'abast dels que estaven a les vores.

Per corraborar aquesta hip�tesi es va procedir a fer la taca m�s gran
i es va poder comprovar com s'acabaven formant anells i espirals 
irregulars m�s o menys equidistants entorn al centre.

%%%%%%%%%%%%%%%%%%%%%%%%%%%%%%%%%%%%%%%%%%%%%%%%%%%%%%%%%%%%%%%%%%%%%%
\subsection{Exemple 2: Comportament amb her�ncia}
%%%%%%%%%%%%%%%%%%%%%%%%%%%%%%%%%%%%%%%%%%%%%%%%%%%%%%%%%%%%%%%%%%%%%%

Es va procedir seguidament a activar el mecanisme d'her�ncia
tot i que sense mutaci�. Donat que no hi havia mutaci�, la 
variabilitat gen�tica del sistema es basaba en els organismes generats 
expont�niament. 

El que es va observar es un relleu continu de les esp�cies: un cop 
els descendents d'un primigeni (organisme generat de forma expont�nia) 
dominaven amb la seva pres�ncia, apareixia un nou grup m�s optimitzat
i acabava substituint a l'altre. Com que cada vegada els organismes 
eren menys dolents els intervals de temps entre relleus es dilatava 
com m�s avan�ava la simulaci�.

%%%%%%%%%%%%%%%%%%%%%%%%%%%%%%%%%%%%%%%%%%%%%%%%%%%%%%%%%%%%%%%%%%%%%%
\subsection{Exemple 3: Comportament amb mutaci�}
%%%%%%%%%%%%%%%%%%%%%%%%%%%%%%%%%%%%%%%%%%%%%%%%%%%%%%%%%%%%%%%%%%%%%%

El canvi m�s important que es va produir en afegir la mutaci� va ser
que els relleus no es produien ja tant entre organismes de diferents
esp�cies sin� que es produien entre mutacions de la mateixa esp�cie.

La rapidesa amb la que milloren els organismes amb la mutaci� no 
deixa espai a la generaci� expont�nia, de tal forma que el primer
organisme que t� exit �s el que es queda i evoluciona.

Els organismes nom�s podien modificar el contingut dels registres 
amb operacions de fenotip i, inicialment, el fenotip s'inicialitza 
amb valors que, llegits com a direccions, no indiquen cap moviment.

Llavors les tres �niques formes de menjar, moure's, atacar o excretar
a una posici� que no f�s la pr�pia era fent una negaci�, una c�rrega
o un Random a un registre de fenotip.

%%%%%%%%%%%%%%%%%%%%%%%%%%%%%%%%%%%%%%%%%%%%%%%%%%%%%%%%%%%%%%%%%%%%%%
\section{Resultats obtinguts amb el sistema final}
%%%%%%%%%%%%%%%%%%%%%%%%%%%%%%%%%%%%%%%%%%%%%%%%%%%%%%%%%%%%%%%%%%%%%%



%%%%%%%%%%%%%%%%%%%%%%%%%%%%%%%%%%%%%%%%%%%%%%%%%%%%%%%%%%%%%%%%%%%%%%
\subsection{Biosistema amb m�ltiples climes}
%%%%%%%%%%%%%%%%%%%%%%%%%%%%%%%%%%%%%%%%%%%%%%%%%%%%%%%%%%%%%%%%%%%%%%

El seg�ent exemple mostra un biosistema de 160x80 on s'han diferenciat
algunes regions amb l'objectiu que els diferents organismes 
s'especialitzin, a la regi� on viuen. Es vol demostrar que cada grup
d'organismes, potser i tot d'origens gen�tics comuns, es pot 
especialitzar per viure en cada zona.

Aix� doncs s'han diferenciat, mitjan�ant els agents externs quatre
zones amb {\em climes diferents}:
\begin{enumerate}
\item [A] 
Una taca de 20 posicions de radi que dipositava nutrients de tipus A
amb molta freq��ncia, per�, que tenia per�odes d'inactivitat.
\item [B] 
Una taca de 20 posicions de radi que dipositava nutrients de tipus B
amb un xic menys de freq��ncia, per�, de forma constant.
\item [C]
Una taca molt dispersa per� molt ampla (60 posicions de radi) que 
dipositava nutrients de tipus C.
\item [D] 
Per la resta del bi�top es deixen petits grup�scles molt densos
de forma aleat�ria.
\end{enumerate}

Entre la zona C i D al llarg del temps no hi ha masses difer�ncies
tret una difer�ncia en la densitat dels organismes. Diguem que el
tipus d'organisme que trionfa a D (que normalment s�n els que millor
cerquen l'aliment) acaba dominant en C.

A les altres dos zones, el resultat m�s clar es que a la llarga 
(aproximadament 1.500.000 d'unitats temporals) els organismes de A i B,
que sempre s�n molt vora�os i competitius, acabaven no movent-se. 
Com que la font de menjar que tenien no es mou, si ells es mouen
es temps que perden de menjar.

A les zones A i B, al comen�ament hi ha molta densitat d'organismes
la qual cosa crea que en sorgir comportaments d'agressi� aleat�ria, 
�s a dir sense saber si hi ha ning� a on ataquem, aquests siguin
beneficiosos. 

En A els comportaments d'agressi�, de banda d'un 
complement per la nutrici� resultaven una forma de debilitar a 
la compet�ncia.

En canvi, en B, l'augment de la voracitat dels organismes causa que
durant el per�ode de temps en que no es diposita aliment no hi hagi
res a menjar. En aquesta situaci�, l'agressi� deixa de ser un 
complement i passa a ser vital. De tal forma que en B, els organismes
que fan servir els sensors de pres�ncia comencen a tindre avantatge
i es desenvolupen comportaments d'aquest tipus.

Despr�s d'aix�, la densitat en B, descendeix i els organismes que
hi sobreviuen s'alimenten del medi, quan hi ha menjar i dels 
organismes que, atrets per la quantitat de nutrients, s'apropen
des de C i D.


%%%%%%%%%%%%%%%%%%%%%%%%%%%%%%%%%%%%%%%%%%%%%%%%%%%%%%%%%%%%%%%%%%%%%%
\subsection{Pressi� selectiva depenent de la densitat}
%%%%%%%%%%%%%%%%%%%%%%%%%%%%%%%%%%%%%%%%%%%%%%%%%%%%%%%%%%%%%%%%%%%%%%

Les primeres experi�ncies amb el prototip van servir, per clarificar 
un dilema que em plantejava al comen�ament de les proves: Cal fer un 
biosistema molt restrictiu per que hi hagi pressi� evolutiva i aix�
es desenvolupin comportaments bons o cal fer-ne un de menys restrictiu 
per que es puguin desenvolupar els organismes primigenis amb major 
facilitat. 

Realment vam comprovar que configurant un biosistema on els primigenis
puguessin desenvolupar-se amb facilitat, quan la poblaci� comen�ava
a pujar, els mateixos organismes, en la compet�ncia pels recursos 
limitats es creaven la pressi� evolutiva.

%%%%%%%%%%%%%%%%%%%%%%%%%%%%%%%%%%%%%%%%%%%%%%%%%%%%%%%%%%%%%%%%%%%%%%
\subsection{Distribuci� d'edats i organismes inmortals}
%%%%%%%%%%%%%%%%%%%%%%%%%%%%%%%%%%%%%%%%%%%%%%%%%%%%%%%%%%%%%%%%%%%%%%

Un altre tema relacionat �s el de la distribuci� de les edats i
el problema dels organismes inmortals apuntat per Todd en \cite{ADeath}.
El problema es basa en el fet de que, si no limitem artificialment
l'edat dels organismes, ens podriem trobar situacions en les que
hi haguessin organismes que no morissin mai.

Beleg i Menezer \cite{LEEIntro1994} apunten el fet de que sense 
limitar artificialment l'edat dels organismes a LEE, emergeixen 
distribucions d'edats estables semblants a les corbes de la equaci�
cl�ssica d'Euler-Lotka. Aix� �s degut al fet de que els organismes 
comparteixen recursos finits i, en conseq��ncia, com hem vist a
l'apartat anterior, la pressi� selectiva depen de la densitat de la
poblaci�.

Quan la poblaci� es mant� estable deixa de cr�ixer, la distribuci� 
en edats tendeix a ser tal que la fracci� d'organismes amb una 
determinada edat equival a la probabilitat de que un organisme
arribi a aquesta edat. \cite{Stearns1992}

En el nostre cas, el problema �s que, a difer�ncia del que passa
al sistema LEE, els organismes de Bioscena no es reprodueixen
autom�ticament quan arriben a un nivell energ�tic, sin� que
han de desenvolupar en el seu codi gen�tic la capacitat de 
reproduir-se nom�s quan tenen prou energia.

Mentre que no hi ha cap organisme que sapiga reproduir-se 
adientment, a grans trets hi ha dos tipus d'organismes. Uns que
proven de reproduir-se abans de tenir l'energia necess�ria i en
fer-ho moren. D'altres que no proven mai de reproduir-se i nom�s
menjen sense que ning� els faci compet�ncia.

Aquests darrers es podrien considerar organismes inmortals. Per�,
en el moment en el que apareix un organisme que es sap reproduir
correctament, la poblaci� puja explossivament i arriba al punt en 
que comen�a a haver pressi� selectiva. En aquest punt, ja es donen 
les condicions perqu� aparegui el comportament emergent descrit 
pels autors del LEE i com he pogut contrastar es d�na.



%\part{Conclusions}
\chapter{Conclusions i l�nies de futur}


%%%%%%%%%%%%%%%%%%%%%%%%%%%%%%%%%%%%%%%%%%%%%%%%%%%%%%%%%%%%%%%%%%%%%%
\section{Conclusions}
%%%%%%%%%%%%%%%%%%%%%%%%%%%%%%%%%%%%%%%%%%%%%%%%%%%%%%%%%%%%%%%%%%%%%%

En aquest apartat de la mem�ria es recullen les conclusions a les que
s'han arribat en aquest projecte.

Primer, faig una valoraci� del que ha estat el desenvolupament de l'eina.

Seguidament, comentar� els resultats que han donant en les experi�ncies
els mecanismes d'expressi� g�nica i el model implementat.

Finalment, s'inclou com un resum dels seus costos econ�mics.


%%%%%%%%%%%%%%%%%%%%%%%%%%%%%%%%%%%%%%%%%%%%%%%%%%%%%%%%%%%%%%%%%%%%%%
\subsection{Conclusions sobre l'eina implementada}
%%%%%%%%%%%%%%%%%%%%%%%%%%%%%%%%%%%%%%%%%%%%%%%%%%%%%%%%%%%%%%%%%%%%%%

%%%%%%%%%%%%%%%%%%%%%%%%%%%%%%%%%%%%%%%%%%%%%%%%%%%%%%%%%%%%%%%%%%%%%%
\subsubsection{Metodologia de treball}
%%%%%%%%%%%%%%%%%%%%%%%%%%%%%%%%%%%%%%%%%%%%%%%%%%%%%%%%%%%%%%%%%%%%%%

La metodologia d'implementaci� i les estrat�gies de disseny han ajudat
molt a fer m�s efectiu el desenvolupament de l'eina. 

%%%%%%%%%%%%%%%%%%%%%%%%%%%%%%%%%%%%%%%%%%%%%%%%%%%%%%%%%%%%%%%%%%%%%%
\subsubsection{La interf�cie}
%%%%%%%%%%%%%%%%%%%%%%%%%%%%%%%%%%%%%%%%%%%%%%%%%%%%%%%%%%%%%%%%%%%%%%

Ha estat un gran inconvenient el fet de presentar una primera 
interf�cie portable en consola de text perqu� limita molt la quantitat
i la qualitat de la informaci� representable i tamb� resta 
interactivitat amb l'usuari.

Aquests dos inconvenients s'han compensat, per un costat, amb 
l'incorporaci� de seq��ncies d'escapament ANSI per augmentar la
capacitat de representaci� del terminal, i, per un altre, la
possibilitat de carregar fitxers de configuraci� en calent amb 
comandes r�pides del teclat.

Tot i aix�, segueix sent interesant la implementaci� d'una interf�cie
gr�fica, front-ends o altres eines complement�ries per crear de forma
m�s interactiva el que ara s�n els fitxers de configuraci� o 
modificar la configuraci� i el mateix sistema directament.

%%%%%%%%%%%%%%%%%%%%%%%%%%%%%%%%%%%%%%%%%%%%%%%%%%%%%%%%%%%%%%%%%%%%%%
\subsubsection{Possibilitats de configuraci�}
%%%%%%%%%%%%%%%%%%%%%%%%%%%%%%%%%%%%%%%%%%%%%%%%%%%%%%%%%%%%%%%%%%%%%%

El model implementat �s altament parametritzable. Fins i tot, la
majoria de par�metres es poden canviar en mig de la simulaci�.

�s molt destacable en aquest sentit el paper dels agents externs. 
Nom�s amb els agents externs implementats en aquest projecte, es 
poden construir escenaris molt diversos per als organismes.
A m�s, com la programaci� de nous tipus d'agents externs �s molt
senzilla i est� documentada al detall, aquests nous agents poden 
oferir noves possibilitats de configuraci� que no hagin estat 
incloses amb els ja existents.

%%%%%%%%%%%%%%%%%%%%%%%%%%%%%%%%%%%%%%%%%%%%%%%%%%%%%%%%%%%%%%%%%%%%%%
\subsubsection{Components intercanviables}
%%%%%%%%%%%%%%%%%%%%%%%%%%%%%%%%%%%%%%%%%%%%%%%%%%%%%%%%%%%%%%%%%%%%%%

Les interf�cies entre els diferents m�duls del nucli implementat, 
en ser tan estretes i gen�riques, permeten no nom�s millorar la
testabilitat dels m�duls i encapsular la implementaci�. Tamb� 
permeten substituir de forma molt simple un m�dul per un altre.

Alguns m�duls an�legs, fins i tot poden conviure en el mateix 
sistema. �s el cas dels organismes i dels sistemes de control dels 
organismes. Aquests dos casos, amplien les opcions d'experimentaci� 
en el futur, per exemple es podria fer una comparaci� en compet�ncia 
entre organismes amb diferents elements de control o sistemes 
metab�lics.

Altres grups de m�duls com els agents externs, els operadors de
mutaci�, i les vistes, directament fan servir aquesta caracter�stica
per complir amb la seva funcionalitat.

%%%%%%%%%%%%%%%%%%%%%%%%%%%%%%%%%%%%%%%%%%%%%%%%%%%%%%%%%%%%%%%%%%%%%%
\subsubsection{Utilitat de l'eina}
%%%%%%%%%%%%%%%%%%%%%%%%%%%%%%%%%%%%%%%%%%%%%%%%%%%%%%%%%%%%%%%%%%%%%%

Donat que, en el Departament d'Inform�tica d'Enginyeria la Salle, 
pot ser no es treballa prou sovint en temes de vida artificial, crec 
que l'exist�ncia d'una eina d'aquest estil, que t� prou generalitat 
com per ser transportada a un conjunt molt ample de problemes, pot 
contribuir a l'increment dels treballs en aquesta �rea.


%%%%%%%%%%%%%%%%%%%%%%%%%%%%%%%%%%%%%%%%%%%%%%%%%%%%%%%%%%%%%%%%%%%%%%
\subsection{Conclusions sobre el model i les experi�ncies realitzades}
%%%%%%%%%%%%%%%%%%%%%%%%%%%%%%%%%%%%%%%%%%%%%%%%%%%%%%%%%%%%%%%%%%%%%%

En quant al model concret d'organisme implementat, cal dir que ha donat
proves de ser suficientment flexible al llarg de l'evoluci�. 

En interval de temps entre 10 i 12 hores d'execuci� s'han trobat 
alguns comportaments complexos o emergents. Exemples s�n:
\begin{itemize}
\item En zones on hi ha molt diferencial de menjar que atreien als
organismes del voltant, s'han observat conductes de depredaci� que 
seguien una estrategia `de trampers'. �s a dir, quedar-se quiet 
on hi ha el menjar i esperar a que arribin les preses. 
\item En zones on hi ha aliments agrupats en petis mont�culs s'han
observat patrons complexos de cerca de nutrients al medi que 
prioritzen els pr�xims, per�, tenint sempre l'opci� de cercar-los a 
dist�ncia fent dues cerques seguides sobre el mateix registre amb 
diferent radi.
\item A zones amb suficient menjar uniformement distribu�t 
s'observen comportaments per mantenir la dist�ncia amb els altres 
organismes, ja sigui per evitar compet�ncia, o per evitar ser
depredat. 
\item En casos semblants a l'anterior, per�, quan la sufici�ncia de 
nutrients no era una situaci� perp�tua, a vegades havien sorgit 
organismes que subordinaven el fet de mantenir-se equidistant al 
fet de trobar-se a una zona amb suficient menjar.
\item Altres...
\end{itemize}

Al comen�ament ha costat molt trobar conductes purament depredadores, 
per� han comen�at a apar�ixer aix� com hem posat els incentius 
energ�tics i configurant un medi m�s variable.

Per que els organismes desenvolupin comportaments complexos, cal 
que els organismes tinguin una vida mitjanament llarga. Si �s possible
reproduir-se de forma temprana, la vida dels organismes es fa molt
curta i en conseq��ncia simple. Dificultar la reproducci� temprana
sembla ser la soluci�.

Primer, vaig intentar compensar-ho posant costos de reproducci� molt
grans. Vaig observar que funcionava, els organismes acabaven tenint
una vida m�s llarga que donava peu a comportaments m�s complexos.
El problema �s que costava molt que els organismes comencessin a
reproduir-se de forma correcta, doncs i ho provaven a fer de forma
incorrecta morien doncs pagaven igual l'energia.

La conclusi� a la que vaig arribat en aquest aspecte �s que, m�s que 
fer pagar-ho amb energia, que evidentment caldria posar requisits 
energia per� no fer-los pagar.

Per acabar amb els comentaris sobre el model, caldria dir que el model
original, preveia comportaments sexuals no lligats directament a la 
reproducci� (intercanvi de fragments cromos�mics com fan els bacteris).
No es va fer perqu� el model ja era prou complex com per afegir un
element m�s per�, crec que hagu�s millorat molt la capacitat d'adaptaci�
del sistema.

Tamb� cal comentar que els significat de les zones operadores �s una
mica pobre i que generalment tendeixen a fer una funci� probabil�stica
m�s que de depend�ncia.


%%%%%%%%%%%%%%%%%%%%%%%%%%%%%%%%%%%%%%%%%%%%%%%%%%%%%%%%%%%%%%%%%%%%%%
\subsection{Estudi econ�mic}
%%%%%%%%%%%%%%%%%%%%%%%%%%%%%%%%%%%%%%%%%%%%%%%%%%%%%%%%%%%%%%%%%%%%%%


De cara a evaluar el temps invertit en la realitzaci� del projecte,
el dividirem en diverses parts. En aquesta taula es representa en
hores.

\begin{table}[ht]
\centering
\begin{tabular}{|l|r|r|r|}
\hline
Part		& {\centering 1998.05 - 1999.06} & {\centering 1999.07 - 2000.01} & Total \\
\hline
Documentaci�	& 120 &  10 & 130 \\
Eines i entorn	&  35 &   8 &  38 \\
Disseny		&  50 &   5 &  55 \\
Implementaci�	& 110 & 490 & 600 \\
Proves		&  30 &  10 &  40 \\
Experimentaci�	&   0 &  80 &  80 \\
Mem�ria		&  25 & 150 & 175 \\
\hline
Total		& 370 & 753 & 1123\\
\hline
\end{tabular}
\caption{Hores destinades al projecte}
\end{table}

La documentaci� del projecte va comen�ar al Maig de 1998. El temps
invertit en el projecte en cada part fins al Juny de 1999 �s estimat. 
Des del Juliol de 1999 fins el Gener de 2000 el temps d'implementaci�,
proves, experimentaci� i mem�ria est� recollit aproximadament, per
una eina de Logging.






%%%%%%%%%%%%%%%%%%%%%%%%%%%%%%%%%%%%%%%%%%%%%%%%%%%%%%%%%%%%%%%%%%%%%%
\section{L�nies de futur}
%%%%%%%%%%%%%%%%%%%%%%%%%%%%%%%%%%%%%%%%%%%%%%%%%%%%%%%%%%%%%%%%%%%%%%



%%%%%%%%%%%%%%%%%%%%%%%%%%%%%%%%%%%%%%%%%%%%%%%%%%%%%%%%%%%%%%%%%%%%%%
\subsection{Experimentaci� amb el model presentat}
%%%%%%%%%%%%%%%%%%%%%%%%%%%%%%%%%%%%%%%%%%%%%%%%%%%%%%%%%%%%%%%%%%%%%%

La principal via de futur d'aquest projecte �s la seva aplicaci�.
Aix� vol dir plantejar experiments que es puguin realitzar sobre 
aquest sistema.

Com a exemple tenim la bateria d'experiments que s'han anat fent a la
Universitat de San Diego sobre el sistema LEE.

%%%%%%%%%%%%%%%%%%%%%%%%%%%%%%%%%%%%%%%%%%%%%%%%%%%%%%%%%%%%%%%%%%%%%%
\subsection{Sistema de control}
%%%%%%%%%%%%%%%%%%%%%%%%%%%%%%%%%%%%%%%%%%%%%%%%%%%%%%%%%%%%%%%%%%%%%%

Es pot adaptar el sistema d'una forma molt directa a altres models 
d'organisme. Per exemple, es podria implementar f�cilment organismes 
cognitius basats en diferents tipus de xarxes neuronals, de forma 
similar a com s'ha fet als sistemes LEE, o en sistemes classificadors,
entre d'altres.

Dels resultats d'aquesta primera experi�ncia hem resaltat que potser
es milloraria el sistema si s'introdueixen o es modifiquen alguns dels
seus elements. Uns exemples serien:

\begin{description}
\item [Millors zones operadores:] El significat que he donat a les zones
operadores �s un xic pobre. La majoria de casos adquireixen m�s un
significat de probabilitat que no pas de depend�ncia. Si s'enriquissin 
una mica m�s, pot ser donaria peu a comportaments molt m�s elaborats.
\item [Instruccions de test i predicats:] Pot ser de cara a millorar
el significat de les zones operadores, convindria implementar 
instruccions de test que modifiquessin bits de predicat, que al mateix
temps es facin servir per validar o no una zona operadora.
\end{description}

%%%%%%%%%%%%%%%%%%%%%%%%%%%%%%%%%%%%%%%%%%%%%%%%%%%%%%%%%%%%%%%%%%%%%%
\subsection{Sistema gen�tic}
%%%%%%%%%%%%%%%%%%%%%%%%%%%%%%%%%%%%%%%%%%%%%%%%%%%%%%%%%%%%%%%%%%%%%%

Tamb� �s possible afegir altres elements interesants a la part 
gen�tica.

\begin{description}
\item [Operadors de mutaci�:] Els que hi ha implementats potser no 
s�n del tot els idonis per un problema donat. 
\item [Evoluci� dels operadors de mutaci�]
Donada la hetereogeneitat
en la distribuci� dels gens als cromos�mes, pot ser �s �til que el 
tipus d'operador de mutaci�, o la probabilitat de mutar amb un o altre, 
tamb� evolucioni amb el mateix cromosoma.
\item [Sexualitat:] Cal recordar que el model presentat no considera
sexualitat, la qual cosa, no permet gaire variabilitat gen�tica. �s
trivial introduir mecanismes sexuals no lligats a la reproducci� com
els dels organismes unicel�lulars. Nom�s caldria considerar l 
\end{description}

%%%%%%%%%%%%%%%%%%%%%%%%%%%%%%%%%%%%%%%%%%%%%%%%%%%%%%%%%%%%%%%%%%%%%%
\subsection{Metabolisme}
%%%%%%%%%%%%%%%%%%%%%%%%%%%%%%%%%%%%%%%%%%%%%%%%%%%%%%%%%%%%%%%%%%%%%%

Les reaccions metab�liques s'han implementat d'una forma molt simple
si tenim en compte les possibilitats de configuraci� que donen els
sistemes LEE. Els sistemes LEE permeten establir en una taula de 
reaccions bin�ries on cada casella cont� una llista de productes i 
el balan� d'energia per a cada dos possibles reactius.


%%%%%%%%%%%%%%%%%%%%%%%%%%%%%%%%%%%%%%%%%%%%%%%%%%%%%%%%%%%%%%%%%%%%%%
\subsection{Interf�cies}
%%%%%%%%%%%%%%%%%%%%%%%%%%%%%%%%%%%%%%%%%%%%%%%%%%%%%%%%%%%%%%%%%%%%%%

En aquest projecte, de cara a mantenir la portabilitat, s'ha volgut
mantenir una interf�cie de tipus terminal de text. Malgrat tot, 
aquesta interf�cie probablement no li ser� prou amigable a un usuari 
final, a m�s, la interf�cie en mode text presenta moltes limitacions 
en la presentaci� de gr�fics i, sobretot, en la quantitat de dades 
representables i la seva llegibilitat.

Aix� doncs, es podria implementar una interf�cie gr�fica que 
facilit�s la feina als usuaris del sistema.

Aquesta tasca ve facilitada pel fet de que, durant l'elaboraci� del 
nucli, s'ha tingut molt present el paradigma model-vista-controlador:

El nucli �s molt independent de la interf�cie. Fins i tot les traces 
i els missatges de debug que han d'estar implementats a dintre del 
nucli, estan preparats per no necessitar d'una plataforma concreta.
Fan servir una classe adaptadora que pot visualitzar-les en un 
terminal, en una finestra de la llibreria Curses, en MessageBoxes o 
controls d'edici� de MS-Windows o potencialment d'X-Windows.

Els pocs objectes visuals implementats en mode text (mapes, gr�fics 
comparatius i gr�fics d'evoluci� temporal) ofereixen un protocol 
d'acc�s que podria implementar a sota elements gr�fics d'alta 
resoluci�.

%%%%%%%%%%%%%%%%%%%%%%%%%%%%%%%%%%%%%%%%%%%%%%%%%%%%%%%%%%%%%%%%%%%%%%
\subsection{Eines adicionals}
%%%%%%%%%%%%%%%%%%%%%%%%%%%%%%%%%%%%%%%%%%%%%%%%%%%%%%%%%%%%%%%%%%%%%%

A mida que compliquem el problema, les modificacions dels fitxers de 
configuraci� dels agents ambientals es tornen molt feixugues. De forma
no massa complicada es pot reutilitzar el codi del nucli per fer una
petita eina que permeti editar l'estructura arboriforme dels agents.

Una altra eina, podria ser una que ens permet�s editar els individus 
o el seu material gen�tic. Editar els organismes d'una forma f�cil
facilitaria els experiments dels bi�legs i obtenir un organisme
primigeni eficient per estalviar temps d'evoluci�. Proposant com a
cultiu primigeni les cepes dominants de diversos experiments editades
convenientment per falcilitar la variabilitat gen�tica.


%%%%%%%%%%%%%%%%%%%%%%%%%%%%%%%%%%%%%%%%%%%%%%%%%%%%%%%%%%%%%%%%%%%%%%
\subsection{Funcionalitats}
%%%%%%%%%%%%%%%%%%%%%%%%%%%%%%%%%%%%%%%%%%%%%%%%%%%%%%%%%%%%%%%%%%%%%%

Donat que les simulacions s'allarguen molt de temps convindria
implementar mecanismes per fer persistent el biosistema un cop
es finalitza l'execuci� de l'eina i recuperar l'estat del biosistema
en una execuci� posterior.

Tamb� seria interesant una funcionalitat que permet�s importar i 
exportar organismes.

%%%%%%%%%%%%%%%%%%%%%%%%%%%%%%%%%%%%%%%%%%%%%%%%%%%%%%%%%%%%%%%%%%%%%%
\subsection{Optimitzacions}
%%%%%%%%%%%%%%%%%%%%%%%%%%%%%%%%%%%%%%%%%%%%%%%%%%%%%%%%%%%%%%%%%%%%%%

Hi ha punts del sistema dels qual encara es pot optimitzar el seu 
comportament:

Per exemple, es podria fer compartir a tots els organismes amb el 
mateix material gen�tic un punter a la mateixa estructura de dades.
La millora que representa aix� en temps i en mem�ria tot i que �s molt
clara amb mutaci� o intercanvis sexuals no reproductius, deixaria de
ser �til amb intercanvis sexuals lligats a la reproducci� donat que
la variabilitat gen�tica seria molt alta.





\appendix
% Time Log
%

%%%%%%%%%%%%%%%%%%%%%%%%%%%%%%%%%%%%%%%%%%%%%%%%%%%%%%%%%%%%%%%%%%%%%%
\chapter{Manual d'usuari}
%%%%%%%%%%%%%%%%%%%%%%%%%%%%%%%%%%%%%%%%%%%%%%%%%%%%%%%%%%%%%%%%%%%%%%

%%%%%%%%%%%%%%%%%%%%%%%%%%%%%%%%%%%%%%%%%%%%%%%%%%%%%%%%%%%%%%%%%%%%%%
\section{Instalaci� de l'entorn}
%%%%%%%%%%%%%%%%%%%%%%%%%%%%%%%%%%%%%%%%%%%%%%%%%%%%%%%%%%%%%%%%%%%%%%

%%%%%%%%%%%%%%%%%%%%%%%%%%%%%%%%%%%%%%%%%%%%%%%%%%%%%%%%%%%%%%%%%%%%%%
\subsection{Generalitats}
%%%%%%%%%%%%%%%%%%%%%%%%%%%%%%%%%%%%%%%%%%%%%%%%%%%%%%%%%%%%%%%%%%%%%%

Per a l'execuci� del programa, nom�s s�n necessaris els seg�ents arxius:

\begin{tabular}{lp{6in}}
bioscena.exe	& L'executable. Pot tenir un altre nom segons el sistema \\
bioscena.ini	& Cont� la configuraci� dels par�metres del biosistema\\
agents.ini	& Cont� la configuraci� dels agents externs que actuen sobre el medi\\
opcodes.ini	& Cont� la correspond�ncia dels bytecodes amb les intruccions
\end{tabular}
Han d'estar tots quatre al mateix directori.

L'executable necessita una pantalla de text de 80x50 i seq��ncies de 
terminal ANSI. Els seg�ents apartats expliquen com fer-ho a diferents
sistemes.

Si, al sistema dest�, no �s possible treballar amb un terminal 80x50
es poden canviar f�cilment el codi font les coordenades dels objectes
gr�fics.

% TODO: Aixo fins que posi quelcom per configurar-ho

%%%%%%%%%%%%%%%%%%%%%%%%%%%%%%%%%%%%%%%%%%%%%%%%%%%%%%%%%%%%%%%%%%%%%%
\subsection{Terminal ANSI de 50 linies sota Windows 95}
%%%%%%%%%%%%%%%%%%%%%%%%%%%%%%%%%%%%%%%%%%%%%%%%%%%%%%%%%%%%%%%%%%%%%%

Per suportar les seq��ncies de control ANSI, cal haver iniciat l'ordinador
o una sessi� MSDOS amb la l�nia 
\begin{verbatim}
DEVICE=C:\WINDOWS\COMMAND\ANSI.SYS
\end{verbatim}
dintre del config.sys.

Per posar la pantalla a 80x50, si �s un executable de DOS (DJGPP), 
cal crear un acces directe i, a les seves propietats, al separador 
'Pantalla' especificar 50 linies.

Si �s un executable Win32 (compilat amb Microsoft Visual C++), 
automaticament es posa a 80x50.

%%%%%%%%%%%%%%%%%%%%%%%%%%%%%%%%%%%%%%%%%%%%%%%%%%%%%%%%%%%%%%%%%%%%%%
\subsection{Terminal ANSI de 50 linies sota Windows-NT}
%%%%%%%%%%%%%%%%%%%%%%%%%%%%%%%%%%%%%%%%%%%%%%%%%%%%%%%%%%%%%%%%%%%%%%

Per fer servir les seq��ncies ANSI amb l'executable MS-DOS, cal 
seguir les seg�ents passes:
\begin{itemize}
\item Si ja existeix l'arxiu {\tt config.nt} al comprimit, cal 
copiar-lo al directori del executable, i saltar-se els dos seg�ents 
passos.
\item Copiem l'arxiu \verb"c:\WINNT\SYSTEM32\config.nt" al directori de l'executable.
\item L'editem i afegim la l�nia:
\begin{verbatim}
DEVICE=$WINNT$\SYSTEM32\ANSI.SYS
\end{verbatim}
\item Obrim les propietats de l'executable DOS
\item Premem el bot� 'Avanzada' del separador 'Programa'
\item A la capsa pel {\tt config.nt} posem l'encaminament del personalitzat.
\end{itemize}

Per canviar el nombre de linies de la pantalla, si ho fem al mateix lloc
que cal fer-ho a Windows 95, no en fa cas. Cal seguir les seg�ents passes:
\begin{itemize}
\item Executar el programa
\item Accedir a l'opcio propietats del menu de sistema de la finestra.
Surtir� un di�leg de propietats diferent que el de Windows 95. 
\item Cal canviar a una lletra suficient petita, si no, ignorar� la resta de canvis. 
\item Augmentar el tamany del buffer de sortida i de l'�rea de pantalla fins a 50 l�nies com a m�nim.
\item En aceptar, posar que guardi les opcions per a proximes execucions.
\end{itemize}
Si tot va b�, la proxima vegada que ho executem, sortir� b�.

Un executable Win32 es posar� automaticament a 80x50, per�, no s'ha 
provat encara com fer funcionar l'ANSI.SYS amb un executable Win32 sota 
Windows NT.

%%%%%%%%%%%%%%%%%%%%%%%%%%%%%%%%%%%%%%%%%%%%%%%%%%%%%%%%%%%%%%%%%%%%%%
\subsection{Terminal ANSI de 50 linies sota Linux}
%%%%%%%%%%%%%%%%%%%%%%%%%%%%%%%%%%%%%%%%%%%%%%%%%%%%%%%%%%%%%%%%%%%%%%

A linux ja hi ha ANSI per defecte tant a terminals en mode text com a X-terms.

Per obtindre un terminal 80x50 aqu� hi han unes solucions.

\begin{tabular}{lp{5in}}
Xterms & Simplement, cal augmentar el tamany de la finestra fins que hi
capiguen 80x50 linies. \\
Terminals ordinaris & Cal especificar al {\tt lilo} el mode gr�fic que fa servir. Consulta el manual.
\end{tabular}

%%%%%%%%%%%%%%%%%%%%%%%%%%%%%%%%%%%%%%%%%%%%%%%%%%%%%%%%%%%%%%%%%%%%%%
\section{Configuraci�}
%%%%%%%%%%%%%%%%%%%%%%%%%%%%%%%%%%%%%%%%%%%%%%%%%%%%%%%%%%%%%%%%%%%%%%

Existeixen tres arxius de configuraci�:

\begin{tabular}{lp{6in}}
bioscena.ini	& Par�metres del biosistema\\
agents.ini	& Agents externs\\
opcodes.ini	& Correspond�ncia entre codis i intruccions
\end{tabular}

A continuaci� s'expliquen en detall.

%%%%%%%%%%%%%%%%%%%%%%%%%%%%%%%%%%%%%%%%%%%%%%%%%%%%%%%%%%%%%%%%%%%%%%
\subsection{Arxiu {\tt bioscena.ini}}
%%%%%%%%%%%%%%%%%%%%%%%%%%%%%%%%%%%%%%%%%%%%%%%%%%%%%%%%%%%%%%%%%%%%%%

Aquest arxiu defineix una s�rie de par�metres num�rics que es fan servir
per configurar el biosistema. Consta d'un seguit de l�nies on cadascuna
defineix un par�metre i t� una estructura semblant a aquesta:

\begin{verbatim}
* Apartat/SubApartat/../NomDelPar�metre 4
\end{verbatim}

A continuaci� s'expliquen els par�metres m�s importants. Entre par�ntesis
es posa el valor que l'eina adopta per omisi�.

\begin{description}
\item[Biosistema/Energia/FixeInstruccio]	(1)
Cost que, com a m�nim, t� una instrucci� pel fet d'executar-se.
\item[Biosistema/Energia/AdicionalInutil]	(1)
Cost adicional que t� una instrucci� motora, si falla la seva execuci�.
\item[Biosistema/Energia/Engolir]	(6)
Guany que rep un organisme pel fet d'engolir un nutrient.
\item[Biosistema/Energia/Excretar]	(3)
Cost adicional d'excretar un nutrient
\item[Biosistema/Energia/Extraccio]	(6)
Guany per cada nutrient extret per agressio a una victima (i cost per la victima)
\item[Biosistema/Energia/InicialExpontani]	(10)
Energia amb la que comen�a un organisme aleatori
\item[Biosistema/Energia/Mitosi/Cedida]	(15)
Energia que es cedeix al fill en la mitosi
\item[Biosistema/Energia/Mitosi/Factor]      (2)
Factor pel que es multiplica l'energia que es cedeix al fill per obtindre el cost de la Mitosi
\item[Biosistema/Energia/Mitosi/Minima]      (20)
M�nima energia requerida per entrar a fer la mitosi (no es consumeix)
\item[Biosistema/Energia/Mitosi/CostNoMinima]        (2)
Cost d'haver entrat a fer la mitosi sense l'energia m�nima
\item[Biosistema/Energia/Moviment]   (0)
Cost adicional per instruccions de moviment
\item[Biosistema/Metabolisme/BitsSignificatius]      (7)
M�scara amb els bits significatius de les claus dels nutrients
\item[Biosistema/OpCodes/BitsOperacio]       (5)
El bits del bytecode que s�n significatius
\item[Biosistema/Quantum/Maxim]	(4)
M�xim nombre d'instruccions que s'executen de cop (El primer quantum que s'acabi mana)
\item[Biosistema/Quantum/Utils]	(2)
M�xim nombre d'instruccions �tils que s'executen de cop (El primer quantum que s'acabi mana)
\item[Biotop/CasellesAltitud]	(30)
Posicions que t� el bi�top d'altura
\item[Biotop/CasellesAmplitud]	(30)
Posicions que t� el bi�top d'amplitud
\item[Biotop/Substrat/MaximInicial]  (7)
Maxims nutrients que pot contenir una sola posici� del bi�top
\item[Biotop/Substrat/MolleculesInicials/Numero]     (2)
Nombre de nutrients amb els que s'omple cada casella del bi�top en iniciar l'execuci�
\item[Biotop/Substrat/MolleculesInicials/Element]    (0)
Element base que es fa servir per omplir cada casella del bi�top en iniciar l'execuci�
\item[Biotop/Substrat/MolleculesInicials/Tolerancia] (3)
Tolerancia sobre l'element base per omplir cada casella del bi�top en iniciar l'execuci�
\item[Comunitat/TamanyRegeneracio]   (27)
Tamany de la poblaci� per sota del qual el biosistema sempre genera organismes aleatoris.
\item[Comunitat/ProbabilitatGeneracioExpontanea/Encerts]     (1)
Probabilitat de que es generi un organisme aleatori independentment del tamany de regeneraci� (Numerador)
\item[Comunitat/ProbabilitatGeneracioExpontanea/Mostra]      (50)
Probabilitat de que es generi un organisme aleatori independentment del tamany de regeneraci� (Denominador)
\item[Organisme/Cromosoma/LongitudMaxima]    (10)
Nombre m�xim de codons d'un cromosoma d'un organismes aleatori
\item[Organisme/Cromosoma/LongitudMinima]    (1)
Nombre m�nim de codons d'un cromosoma d'un organismes aleatori
\item[Organisme/Cariotip/LongitudMaxima]     (10)
Nombre m�xim de cromosomes d'un organismes aleatori
\item[Organisme/Cariotip/LongitudMinima]     (3)
Nombre m�nim de cromosomes d'un organismes aleatori
\item[Organisme/Cariotip/PenalitzacioLlarg/BitsTamanyMaxim]  (8)
Bits que es shifta a la dreta el tamany en codons del cariotip per penalitzar els cariotips massa llargs
\item[Organisme/Cariotip/PenalitzacioLlarg/Factor]   (6)
Factor per el que es multiplica el tamany del cariotip shiftat per obtindre el cost adicional pels cariotips massa llargs
\item[Organisme/Energia/CaducitatCompartiments]      (20)
El nombre d'instruccions que triga en caducar el compartiment d'energia m�s vell.
\item[Organisme/Energia/Compartiments]       (8)
Nombre de compartiments d'energia que t� l'organisme.
\item[Organisme/Fenotip/Longitud]    (16)
Nombre de registres que formen el fenotip.
\item[Organisme/Genotip/Intro/Mascara]       0
La m�scara que determina si un cod� �s un intr�
\item[Organisme/Genotip/Promotor/Mascara]    1048576
La m�scara que determina si un cod� �s un promotor
\item[Organisme/Genotip/Terminador/Mascara]  1048576
La m�scara que determina si un cod� �s un terminador
\item[Organisme/Genotip/Traduibilitat/Intents]       5
El nombre de gens que es miren si s�n tradu�bles abans de refusar traduir-ne cap.
\item[Organisme/Genotip/ZonaOperadora/Mascara]       (65536)
La m�scara que determina si un cod� pertany encara a la zona operadora
\item[Organisme/Pap/Capacitat]       (10)
Nombre de nutrients que pot contenir l'organismes
\item[Organisme/ProbabilitatMutacio/Encerts] (1)
Determina la probabilitat de que es doni una mutaci� (Numerador).
\item[Organisme/ProbabilitatMutacio/Mostra]  (15)
Determina la probabilitat de que es doni una mutaci� (Denominador).
\item[Sensor/Intern/Intents]	(10)
Les vegades que un organismes prova trobar un nutrient en el pap quan vol detectar-ho.
\item[Sensor/Presencia/Intents]	(30)
El nombre de posicions en les que es prova mirar si hi ha alg� que compleix les condicions de pres�ncia.
\item[Sensor/Quimic/Intents]	(10)
El nombre de posicions en les que es prova mirar si hi ha nutrients amb les condicions especificades.

\end{description}


%%%%%%%%%%%%%%%%%%%%%%%%%%%%%%%%%%%%%%%%%%%%%%%%%%%%%%%%%%%%%%%%%%%%%%
\subsection{Arxiu {\tt opcodes.ini}}
%%%%%%%%%%%%%%%%%%%%%%%%%%%%%%%%%%%%%%%%%%%%%%%%%%%%%%%%%%%%%%%%%%%%%%

Aquest arxiu simplement configura la correspond�ncia entre els opcodes
i les instruccions que s'executen.

Amb la variable {\tt Biosistema/OpCodes/BitsOperacio} de l'arxiu 
{\tt bioscena.ini} es pot limitar els bits �tils d'aquest byte 
escollint els n menys significatius. 
Aix� no cal generar les 256 entrades.

El seg�ent arxiu serviria per a 5 bits significatius. 
{\setlinespacing{1}
\begin{verbatim}
* 00 NoOperacio
* 01 Mitosi
* 02 Anabol
* 03 Engoleix
* 04 Excreta
* 05 Ataca
* 06 Avanca
* 07 SensorI
* 08 SensorP
* 09 SensorQ
* 0A Random
* 0B Carrega
* 0C And
* 0D Xor
* 0E Oposa
* 0F ShiftR
\end{verbatim}
}

El n�meros que es fan servir per l'opcode estan codificats en hexadecimal.
Cada l�nia ha d'anar precedida d'un signe {\tt *} i separant els tres
elements de cada l�nia, va un espai o tabulador.

La taula \ref{tab:mnemonics} es citen tots els mnem�nics implementats:

\begin{table}[ht]
\centering
\begin{tabular}{|cccc|}
\multicolumn{4}{c}{Instruccions Motores}\\
\hline
 NoOperacio	& Mitosi 	& Avanca	& Catabol	\\
 Engoleix	& Excreta	& Ataca		& Anabol	\\
\hline
\multicolumn{4}{c}{}\\
\multicolumn{4}{c}{Instruccions Fenotip}\\
\hline
 And		& Or		& Xor		& Carrega	\\
 Copia		& Not		& Oposa		& Random	\\
 ShiftR		& ShiftL	&		&		\\
\hline
\multicolumn{4}{c}{}\\
\multicolumn{4}{c}{Instruccions Sensorials}\\
\hline
 SensorI	& SensorP	& SensorQ	& 		\\
\hline
\end{tabular}
\caption{Mnem�nics implementats}
\label{tab:mnemonics}
\end{table}

Si cap opcode queda est� sense assignar-li un mnem�nic, es donar�
un missatge d'advertiment i se li asignara {\tt NoOperacio}.

%%%%%%%%%%%%%%%%%%%%%%%%%%%%%%%%%%%%%%%%%%%%%%%%%%%%%%%%%%%%%%%%%%%%%%
\subsection{Arxiu {\tt agents.ini}}
%%%%%%%%%%%%%%%%%%%%%%%%%%%%%%%%%%%%%%%%%%%%%%%%%%%%%%%%%%%%%%%%%%%%%%

El funcionament d'aquest fitxer de configuraci� s'explica en m�s 
detall a l'apartat \ref{sec:agentsExterns} de la mem�ria.

%%%%%%%%%%%%%%%%%%%%%%%%%%%%%%%%%%%%%%%%%%%%%%%%%%%%%%%%%%%%%%%%%%%%%%
\section{Operaci� normal}
%%%%%%%%%%%%%%%%%%%%%%%%%%%%%%%%%%%%%%%%%%%%%%%%%%%%%%%%%%%%%%%%%%%%%%

%%%%%%%%%%%%%%%%%%%%%%%%%%%%%%%%%%%%%%%%%%%%%%%%%%%%%%%%%%%%%%%%%%%%%%
\subsection{La pantalla}
%%%%%%%%%%%%%%%%%%%%%%%%%%%%%%%%%%%%%%%%%%%%%%%%%%%%%%%%%%%%%%%%%%%%%%

Quan executem el programa, un cop carregats els fitxers de configuraci�,
s'ens presenta una pantalla d'ajuda a la que podem tornar sempre que
volguem prement 'H' i return. Aquesta pantalleta explica breument les
comandes que hi ha. Per comen�ar a executar la simulaci� premem return.

El que es pot veure es una pantalla semblant a la de la figura 
\ref{fig:pantalla1}. El de la esquerra �s un mapa del m�n, el de la
dreta �s un gr�fic comparatiu dels organismes que hi ha vius en cada 
moment. De banda d'aquests dos elements, hi ha un indicadors del tamany
de la poblaci�, el temps transcorregut i les coordenades on es troba
la cantonada superior esquerra del mapa.

\begin{figure}[ht]
\fbox{\pdfimage width 6in {pantalla1.png}}
\label{fig:pantalla1}
\caption{Pantalla Mapa/Comparativa}
\end{figure}

Per entrar qualsevol comanda, podem entrar una tecla, en aquest moment
ens sortirar un prompt {\tt Comanda:} que ens permet visualitzar i
corregir la nostra entrada. Un cop feta premem return i s'executar�.


%%%%%%%%%%%%%%%%%%%%%%%%%%%%%%%%%%%%%%%%%%%%%%%%%%%%%%%%%%%%%%%%%%%%%%
\subsection{El mapa del m�n}
%%%%%%%%%%%%%%%%%%%%%%%%%%%%%%%%%%%%%%%%%%%%%%%%%%%%%%%%%%%%%%%%%%%%%%

El mapa representa dues coses:

\begin{itemize}
\item Els organismes, representats per n�meros de colors. Un n�mero i
un color diferencien cada grup reproductiu.
\item La quantitat de nutrients en cada posici�, representada per un 
punt de color. Per ordre de magnitud, els colors s�n: negre(0), vermell(1),
verd(2), taronja(3), blau fosc(4), lila(5), cyan(6) i blanc fosc(7).
\end{itemize}

Si el m�n representat �s m�s petit que la zona del mapa, aquest es
repetir� en forma de mosaic, la qual cosa ajuda a veure les conexions
de la superf�cie toroidal.
Si �s m�s gran, podem fer servir el teclat per veure les altres zones
en direcci� nord(W), sud (S), est(D) i oest(A). La lletra en min�scula
avan�a una posici�, en maj�scula avan�a una pantalla sencera.

L'indicador de coordenades ajuda a saber on som.

Per desactivar el mapa i deixar m�s espai pels altres elements gr�fics
cap premer a la 'M' maj�scula i retorn. Per tornar-ho a activar es prem
la 'm' min�scula.

%%%%%%%%%%%%%%%%%%%%%%%%%%%%%%%%%%%%%%%%%%%%%%%%%%%%%%%%%%%%%%%%%%%%%%
\subsection{El gr�fic comparatiu}
%%%%%%%%%%%%%%%%%%%%%%%%%%%%%%%%%%%%%%%%%%%%%%%%%%%%%%%%%%%%%%%%%%%%%%

En quant al grafic conparatiu, representa per a cada organisme
\begin{itemize}
\item l'energia (amb el s�mbol '\verb"*"'), 
\item l'edat (el s�mbol '\verb"-"'), i 
\item el grup reproductiu, amb la mateixa codificaci� que el mapa
\end{itemize}

L'escala que es fa servir per l'energia i l'edat es logar�tmica o 
linial segons s'hagi configurat. Per defecte, �s logar�tmica per 
tots dos valors.

El factor d'escala de les gr�fiques �s din�mic, aix� vol dir que
canvia a mesura que el valor m�xim creix o decreix.
Per tenir una refer�ncia, la banda de color verda que apareix al
gr�fic representa el valor 255 per la edat i, la taronja, 65535.

Cada organisme t� durant la seva vida un lloc fixe al gr�fic, que
es correspon amb un n�mero en base octal que l'identifica. 
El numero que es trova a la primera filera de n�meros �s el digit
octal de menys pes, el color indica el segon. Podem veure un
organisme per dins prement la 'v' i escrivint el seu codi octal.

Quan un lloc en la gr�fica ja no est� ocupat per cap organisme, 
s'indica amb una {\tt X}.

Per desactivar la comparativa d'organismes i deixar m�s espai pels
altres elements gr�fics cap premer a la 'O' maj�scula i retorn. 
Per tornar-ho a activar es prem la 'o' min�scula.

%%%%%%%%%%%%%%%%%%%%%%%%%%%%%%%%%%%%%%%%%%%%%%%%%%%%%%%%%%%%%%%%%%%%%%
\subsection{Detall de les accions}
%%%%%%%%%%%%%%%%%%%%%%%%%%%%%%%%%%%%%%%%%%%%%%%%%%%%%%%%%%%%%%%%%%%%%%

Per veure un log amb el detall de les accions que es produeixen al 
biosistema, cal pr�mer la tecla 'l' min�scula. Per treure el log,
cal pr�mer la 'L' maj�scula.

%%%%%%%%%%%%%%%%%%%%%%%%%%%%%%%%%%%%%%%%%%%%%%%%%%%%%%%%%%%%%%%%%%%%%%
\subsection{Pas de visualtitzaci�}
%%%%%%%%%%%%%%%%%%%%%%%%%%%%%%%%%%%%%%%%%%%%%%%%%%%%%%%%%%%%%%%%%%%%%%

La visualizaci� enlenteix molt el sistema, si es vol anar m�s r�pid
es pot saltar un nombre de intervals de temps determinats. Per indicar
un pas de visualitzaci� m�s relaxat cal pr�mer la 'j' i despr�s el 
n�mero de passos que ens volem saltar entre visualitzaci� i 
visualitzaci�.

A les proves, s'ha pogut comprovar que, en un Pentium II, un pas de 10 
fa que els moviments dels organismes semblin m�s naturals. Un pas de 
100 no et permet veure en detall qu� �s el que passa per�, �s suficient
per advertir a temps els canvis bruscos que es donen sovint en la 
poblaci�.

Cal recordar que les unitats de temps que es calculen s�n les de 
simulaci�, i poden durar m�s o menys segons el tamany de la poblaci�.

%%%%%%%%%%%%%%%%%%%%%%%%%%%%%%%%%%%%%%%%%%%%%%%%%%%%%%%%%%%%%%%%%%%%%%
\subsection{Modificant la configuraci� durant la simulaci�}
%%%%%%%%%%%%%%%%%%%%%%%%%%%%%%%%%%%%%%%%%%%%%%%%%%%%%%%%%%%%%%%%%%%%%%

A vegades, �s complex programar una configuraci� d'agents per fer que 
vari� al llarg de molt temps. Pot resultar m�s f�cil modificar-ho
quan ho necessitem, carregant l'arxiu de configuraci� d'agents amb un
contingut diferent. Tamb� seria �til

La tecla {\tt r} recarrega el fitxer de configuraci� d'agents 
{\tt bioscena.ini} i {\tt R}, en maj�scules, carrega el fitxer de 
configuraci� d'agents que s'indiqui a continuaci�.
Si no s'especifica cap o un d'erroni, donar� un error i es continuar�
sense cap agent que actui sobre el biosistema. Aix� �s molt �til per
simular un cataclisme o una crisis, doncs desapareixen tots els agents 
que diposisten nutrients.

La forma d'actuar recomanada es tenir un arxiu {\tt bioscena.ini} amb
una situaci� normal, i diversos arxius amb configuracions d'agents 
que implementin situaci�ns excepcionals.

%%%%%%%%%%%%%%%%%%%%%%%%%%%%%%%%%%%%%%%%%%%%%%%%%%%%%%%%%%%%%%%%%%%%%%
\subsection{Altres comandes}
%%%%%%%%%%%%%%%%%%%%%%%%%%%%%%%%%%%%%%%%%%%%%%%%%%%%%%%%%%%%%%%%%%%%%%

\begin{tabular}{lp{6in}}
p & Congela el biosistema, permetent fer altres operacions sense que
    la situaci� canvi� \\
P & Descongela el biosistema \\
e & Fa que els taxons de pare i fill siguin diferents en fer una mutaci� \\
E & Fa que els taxons de pare i fill siguin iguals encara que hi hagi una mutaci� \\
H/h & Presenta la pantalla d'ajut \\
Q/q & Surt del programa \\
\end{tabular}

%%%%%%%%%%%%%%%%%%%%%%%%%%%%%%%%%%%%%%%%%%%%%%%%%%%%%%%%%%%%%%%%%%%%%%

% Time Log
% 

%%%%%%%%%%%%%%%%%%%%%%%%%%%%%%%%%%%%%%%%%%%%%%%%%%%%%%%%%%%%%%%%%%%%%%
\chapter{Manual del programador}
%%%%%%%%%%%%%%%%%%%%%%%%%%%%%%%%%%%%%%%%%%%%%%%%%%%%%%%%%%%%%%%%%%%%%%

%%%%%%%%%%%%%%%%%%%%%%%%%%%%%%%%%%%%%%%%%%%%%%%%%%%%%%%%%%%%%%%%%%%%%%
\section{Programaci� de noves topologies}
%%%%%%%%%%%%%%%%%%%%%%%%%%%%%%%%%%%%%%%%%%%%%%%%%%%%%%%%%%%%%%%%%%%%%%

Si l'usuari necessit�s crear un nou tipus de topologia, cal que la 
faci heretar de CTopologia, que �s la classe que defineix el m�nim per 
reservar memoria pel substrat de cada posici�. CTopologia tamb� 
estableix el protocol que han de seguir les subclasses, perqu� la
resta del sistema l'acepti sense haver de canviar-ho.

El secret est� en el fet de que tot el sistema manega identificadors 
de posicions que s�n enters sense signe de 32 bits. Tot el significat
que poden tenir aquests identificadors el manega la topologia 
internament.

Quan es deriva de CTopologia, el principal que caldria redefinir, 
si cal, �s:
\begin{itemize}
\item	Un {\bf constructor} significatiu per a la topologia. Per exemple, en 
	una topologia rectangular �s significatiu indicar l'altura i l'amplada. 
	El constructor de CTopologia simplement reserva espai per N casselles.
	Caldria calcular aquesta N per passar-se-la.
\item	{\tt t\_posicio CTopologia::desplacament (t\_posicio origen, t\_desplacament desplacament)}:
	Una funci� per averiguar la posici� dest� en aplicar-li un vector
	de despla�ament a una posici� origen.
	CTopologia, la defineix de tal forma que el resultat �s una posici� 
	dest� aleatoria.
\item	{\tt bool CTopologia::esValid(t\_posicio id)}:
	Una funci� per saber si un identificador �s v�lid. Nom�s cal
	redefinir-ho si es modifica la correspond�ncia directa entre
	identificador de posici� i index de cassella en l'array de 
	substrats reservada per CTopologia::CTopologia
\item	{\tt t\_posicio CTopologia::posicioAleatoria ()}:
	Una funci� per obtindre aleat�riament una posici� v�lida de la 
	topologia. La funci� general que no caldria redefinir seria
	\begin{verbatim}
		{
			uint32 pos;
			do {pos=rnd.get();} while (!esValid(pos));
			return pos
		}
	\end{verbatim}
	per�, CTopologia no fa servir aquest algorisme donat que
	optimitza agafant un n�mero aleatori entre 0 i N. Aquesta 
	optimitzaci� funciona mentre es mantingui la correspond�ncia 
	entre identificador i index abans comentada.
	Si la subclasse la trenca, es quan cal redefinir la funci�.
\item	{\tt bool CTopologia::unio (t\_posicio origen, t\_posicio desti, t\_desplacament \& desp)}:
	Una funci� per calcular el primer d'un conjunt de desplacaments
	que cal fer per anar de l'origen al dest�.
	Retorna cert si el desplacament �s suficient per arribar a la posici� dest�.
	CTopologia, la defineix de tal forma que el resultat �s un despla�ament 
	aleatori i retorna sempre fals (mai hi arriba).
\end{itemize}

Pot ser molt ilustratiu, de cara a implementar noves topologies, 
fixar-se en les ja existents com CTopologiaToroidal.

%%%%%%%%%%%%%%%%%%%%%%%%%%%%%%%%%%%%%%%%%%%%%%%%%%%%%%%%%%%%%%%%%%%%%%
\section{Programaci� de nous agents}
%%%%%%%%%%%%%%%%%%%%%%%%%%%%%%%%%%%%%%%%%%%%%%%%%%%%%%%%%%%%%%%%%%%%%%

De cara a afegir nous agents al sistema, s'aconsella seguir els
seg�ents passos:

\begin{enumerate}
\item 
	Llegir per sobre el codi dels agents ja implementats per 
	assimilar les solucions que s'han donat a problemes que 
	segurament es tornaran a repetir als nous agents.
	Tamb� conv� mantenir uniforme l'estil de programaci� i
	l'ordre intern dels fitxers per fer-ho m�s mantenible
	a tercers.
	El m�s pr�ctic es partir d'una c�pia d'un agent que tingui, 
	estructuralment, tot o gran part del que interesa implementar.
\item 
	Escollir la classe d'agent de la que volem heretar l'agent. 
	Generalment voldrem que el nou agent pertanyi a un dels quatre 
	grans grups funcionals d'agents: 
	\begin{itemize}
	\item	Subordinadors (CMultiAgent i subclasses) si controla l'accionat d'altres agents
	\item	Posicionadors (CPosicionador i subclasses) si controla una posici� en el bi�top
	\item	Direccionadors (CDireccionador i subclasses) si controla una direcci�
	\item	Actuadors (CActuadors i subclasses) si modifica el substrat a una posici�
	\end{itemize}
	Si no pertany a cap dels quatre grups, caldria plantejar-se
	heretar de CAgent directament. En aquest cas, conv� fer un
	esfor� i fer una classe intermitja que pugui englobar altres
	agents en el futur.
	Anomenarem CAgentNou al nou agent afegit i CAgentVell a l'agent
	del qual heretem.
\item
	Adaptar el constructor de CAgentNou per que proveeixi els 
	par�metres del constructor de la superclasse. 
	Posicionadors i direccionadors, per exemple, necessiten una 
	refer�ncia a un bi�top en el constructor. A les classes
	derivades de CPosicionador est� implementat com fer-ho.
\item
	Afegir dins del constructor, la l�nia.
	\\{\tt m\_tipus+="/ElMeuSubtipus";}\\
	que afegeix la cadena de subtipus a l'identificador de tipus 
	que hereta de la superclasse.
\item
	Afegir els nous atributs (variables membre) dels que en depen 
	l'estat de l'agent i les funcions d'acc�s als mateixos.
\item
	Inicialitzar dins del constructor els nous atributs als 
	valors per defecte.
	Els atributs que siguin depend�ncies amb altres agents,
	o agents subordinats, es recomana que siguin punters, i no
	refer�ncies, per poder-ho deixar sense especificar
	al constructor. S'inicialitzen sempre com a punter
	a NULL. Cal procurar que, si el punter no apunta a un 
	agent v�lid el seu valor sigui NULL i tenir-ho en compte
	quan hi accedim per evitar accesos il�legals a mem�ria.
	Es veu clarament aquesta idea llegint el codi d'alguns
	agents que ho fan.
\item
	Afegir dins del destructor, l'alliberament de mem�ria ocupada
	pels agents subordinat. Les depend�ncies no s'han de alliberar
	pas.
\item
	Redefinir la funci� membre {\tt virtual void CAgentNou::operator() (void)}
	per que faci el que hagi de fer quan l'agent �s accionat.
	Si es tracta d'un actuador, no cal redefinir aquesta sino
	{\tt virtual void CAgentNou::operator() (CSubstrat \& s)} on 
	{\tt s} �s el substrat que hem de modificar. 
	\footnote{Veure l'apartat \ref{TODO} que parla del que cal fer si es redefineix el substrat}
\item
	Redefinir la funci� {\tt virtual void CAgentNou::dump(CMissatger \& msg)}
	per que cridi a la funci� corresponent de la superclasse 
	(CAgentVell a l'exemple) i, despr�s, inserti en el CMissatger 
	les noves l�nies de configuraci� dels par�metres que afegeix l'agent:
	\begin{verbatim}
	void CAgentNou::dump(CMissatger & msg) 
	{
		CAgentVell::dump(msg);
		msg << "- UnParametreNou " << m_valor1 << " " << valor2 << endl;
		msg << "- UnAltreParametreNou " << m_valor3 << endl;
	}
	\end{verbatim}
\item
	Redefinir la funci� {\tt virtual bool CAgentNou::configura(string parametre, istream \& valors, t\_diccionariAgents \& diccionari, CMissatger \& errors)}
	per mirar si el {\tt parametre} �s un dels que ha afegit CAgentNou.
	Si ho �s cal parsejar l'istream {\tt valors} en busca dels valors 
	corresponents, reportar els errors que es produeixin pel CMissatger 
	{\tt errors} i retornar cert per dir que el par�metre era de la classe.
	Si no ho �s, cal cridar a la funci� corresponent de la superclasse
	per que ho pugui interceptar ella.
	El diccionari serveix per, donat un nom d'agent de l'arxiu,
	obtindre un punter a l'agent que s'ha creat que, pot ser, t�
	un nom diferent.
	El diccionari �s un {\tt map<string, CAgent*>}, el seu funcionament 
	s'explica a qualsevol manual sobre les Standard Template 
	Libraries de C++.
	La estructura general de la funci� configura quedar� com aix�:

	\begin{verbatim}
	bool CAgentNou::configura(string parametre, istream & valors, 
	t_diccionariAgents & diccionari, CMissatger & errors)
	{
		if (parametre=="UnParametreNou") {
			// Parsing dels valors...
			return true;
		}
		if (parametre=="UnAltreParametreNou") {
			// Parsing dels valors...
			return true;
		}
		// Li deixem a la superclasse que l'intercepti si vol
		return CAgentVell::configura(parametre, valors, diccionari, errors);
	}
	\end{verbatim}
\item
	Si cap dels atributs (m\_dependencia a l'exemple) �s una 
	depend�ncia amb altre agent, cal redefinir la seg�ent funci� com segueix:
	\begin{verbatim}
	list<CAgent*> CAgentNou::dependencies() {
		list<CAgent*> l=CAgentVell::dependencies();
		if (m_dependencia) l.push_back(m_dependencia); 
		return l;
	}
	\end{verbatim}
\item
	Si cap dels atributs (m\_subordinat a l'exemple) �s un 
	agent subordinat, cal redefinir la seg�ent funci� com segueix:
	\begin{verbatim}
	list<CAgent*> CAgentNou::subordinats() {
		list<CAgent*> l=CAgentVell::subordinats();
		if (m_subordinat) l.push_back(m_subordinat); 
		return l;
	}
	\end{verbatim}
\item
	Afegir a l'arxiu {\tt Agent.cpp} un include a {\tt AgentNou.h}
	i, a la funci� est�tica CAgent::CreaAgent(...) una l�nia com 
	les que ja n'hi ha per cada tipus d'agent, per�, per a CAgentNou.
	Aix� permet que la funci� CAgent::ParsejaArxiu pugui reconeixer
	el nou tipus als arxius de configuraci�.
	
\end{enumerate}

De tots els punts anteriors el que potser �s una mica m�s 
particularitzat s�n els atributs i els m�todes d'acc�s als mateixos, 
i el m�tode d'accionament (o d'actuaci� en el cas dels actuadors).
Per a la resta de coses el m�s pr�ctic es fer un cut\&paste dels agents
ja implementats i retocar el m�nim.

%%%%%%%%%%%%%%%%%%%%%%%%%%%%%%%%%%%%%%%%%%%%%%%%%%%%%%%%%%%%%%%%%%%%%%
\section{Programaci� de nous substrats}
%%%%%%%%%%%%%%%%%%%%%%%%%%%%%%%%%%%%%%%%%%%%%%%%%%%%%%%%%%%%%%%%%%%%%%

%% TODO: Programaci� de nous substrats


%%%%%%%%%%%%%%%%%%%%%%%%%%%%%%%%%%%%%%%%%%%%%%%%%%%%%%%%%%%%%%%%%%%%%%


\addcontentsline{toc}{chapter}{Bibliografia}
%%%%%%%%%%%%%%%%%%%%%%%%%%%%%%%%%%%%%%%%%%%%%%%%%%%%%%%%%%%%%%%%%%%%%
%
%

\nocite{*} % Per que surti tot
\bibliographystyle{alpha}
\bibliography{biologia,alive,tecnologia,documentacio}

%\addcontentsline{toc}{chapter}{Bibliografia}
%\bibliography{biologia}
%\addcontentsline{toc}{section}{�rea de Biologia}
%\bibliography{alive}
%\addcontentsline{toc}{section}{�rea de Artificial-Live}
%\bibliography{programacio}
%\addcontentsline{toc}{section}{�rea de Programaci�}
%\bibliography{documentacio}
%\addcontentsline{toc}{section}{�rea de Sistemes de documentaci�}

\addcontentsline{toc}{chapter}{�ndex Alfab�tic}
\printindex

\end{document}
