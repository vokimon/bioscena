%%%%%%%%%%%%%%%%%%%%%%%%%%%%%%%%%%%%%%%%%%%%%%%%%%%%%%%%%%%%%%%%%%%%%
%
%

%%%%%%%%%%%%%%%%%%%%%%%%%%%%%%%%%%%%%%%%%%%%%%%%%%%%%%%%%%%%%%%%%%%%%
\section{Serveis que donen al biosistema els organismes}
%%%%%%%%%%%%%%%%%%%%%%%%%%%%%%%%%%%%%%%%%%%%%%%%%%%%%%%%%%%%%%%%%%%%%

De cara a poder manegar
\begin{itemize}
\item
L'organisme ha d'anar proveint al biosistema de instruccions que
l'indiquin les seves accions. Com es generen les instruccions �s
q�esti� dels propis individuus.
\item
L'organisme proveeix al biosistema un conjunt de registres que formen
el seu fenotip. El biosistema pot modificar-los i consultar-los.
L'acc�s al fenotip no �s exclusiu del biosistema sin� que el propi
organisme i les seves estructures internes tamb� poden accedir-hi
paral�lelament.
\item
L'organisme ha d'implementar unes funcions vitals que modifiquin
l'estat intern de l'organisme als que no tingui acc�s el biosistema
per altres vies. Aix� no inclou, per exemple, canvis de posici�.
El biosistema ha de proporcionar tots els par�metres
de les funcions vitals que no siguin interns que ell coneixi: el
fenotip del propi organisme o un d'ali�.
\item
Tamb� han d'implementar operadors per tal de crear nous organismes a
partir d'altres i organismes aleatoris.

\end{itemize}

%%%%%%%%%%%%%%%%%%%%%%%%%%%%%%%%%%%%%%%%%%%%%%%%%%%%%%%%%%%%%%%%%%%%%
\section{Estructura interna dels organismes}
%%%%%%%%%%%%%%%%%%%%%%%%%%%%%%%%%%%%%%%%%%%%%%%%%%%%%%%%%%%%%%%%%%%%%


%%%%%%%%%%%%%%%%%%%%%%%%%%%%%%%%%%%%%%%%%%%%%%%%%%%%%%%%%%%%%%%%%%%%%
\subsection{El cariotip}
%%%%%%%%%%%%%%%%%%%%%%%%%%%%%%%%%%%%%%%%%%%%%%%%%%%%%%%%%%%%%%%%%%%%%

Definim el cariotip com el conjunt de cromosomes que cont� un
organisme, doncs en el present cas es consideren organismes
unicel�lulars.

Cada cromosoma est� format per una seq��ncia de bases representada
cadascuna amb un sencer.

Cariotip->Cromosoma->Base->Codo->

%%%%%%%%%%%%%%%%%%%%%%%%%%%%%%%%%%%%%%%%%%%%%%%%%%%%%%%%%%%%%%%%%%%%%
\subsection{El genotip}
%%%%%%%%%%%%%%%%%%%%%%%%%%%%%%%%%%%%%%%%%%%%%%%%%%%%%%%%%%%%%%%%%%%%%

El genotip �s la traducci� del cariotip a elements significatius.
�s el conjunt de gens que s'interpreten

%%%%%%%%%%%%%%%%%%%%%%%%%%%%%%%%%%%%%%%%%%%%%%%%%%%%%%%%%%%%%%%%%%%%%
\subsection{El fenotip}
%%%%%%%%%%%%%%%%%%%%%%%%%%%%%%%%%%%%%%%%%%%%%%%%%%%%%%%%%%%%%%%%%%%%%

El que anomenem propiament fenotip �s un conjunt de 32 registres de
32 bits que t� cada organisme. Representen el cos f�sic de l'organisme.
El fenotip es modifica per acci� directa del genotip, per� tamb�
es veu afectat pel medi mitjan�ant els sensors i, al mateix temps
afecta al medi mitjan�ant els motors. A m�s, �s un dels dos mitjans
que t�nen els organismes per reconeixer-se juntament amb la detecci�
de mol�l�cules excretades.

%%%%%%%%%%%%%%%%%%%%%%%%%%%%%%%%%%%%%%%%%%%%%%%%%%%%%%%%%%%%%%%%%%%%%
\subsection{Sensors i motors}
%%%%%%%%%%%%%%%%%%%%%%%%%%%%%%%%%%%%%%%%%%%%%%%%%%%%%%%%%%%%%%%%%%%%%

%%%%%%%%%%%%%%%%%%%%%%%%%%%%%%%%%%%%%%%%%%%%%%%%%%%%%%%%%%%%%%%%%%%%%
\subsection{Presa de nutrients}
%%%%%%%%%%%%%%%%%%%%%%%%%%%%%%%%%%%%%%%%%%%%%%%%%%%%%%%%%%%%%%%%%%%%%

%%%%%%%%%%%%%%%%%%%%%%%%%%%%%%%%%%%%%%%%%%%%%%%%%%%%%%%%%%%%%%%%%%%%%
\subsection{Metabolisme}
%%%%%%%%%%%%%%%%%%%%%%%%%%%%%%%%%%%%%%%%%%%%%%%%%%%%%%%%%%%%%%%%%%%%%

Un cop els nutrients estan dins de l'organisme, pot metabolitzar-los,
ja sigui per obtindre energia, per obtindre un producte d'excreci� o
totes dues coses.

L'organisme pot fer �s de l'energia que s'obte de les reaccions
durant un temps limitat, si no es consumeix dintre d'aquest temps,
aquesta es disipa.



Es preten que les diferents circumst�ncies permetin que hi hagi un
equilibri

Pot ser, una esp�cie tendeix a acomular masses nutrients.
Si aix� passes, els seus depredadors tenderien a augmentar en
nombre d'organismes i en nombre d'esp�cies.
