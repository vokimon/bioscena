%%%%%%%%%%%%%%%%%%%%%%%%%%%%%%%%%%%%%%%%%%%%%%%%%%%%%%%%%%%%%%%%%%%%%
%
%

\begin{thebibliography}{DARW98}

\section{Biologia}

\bibitem{ElOrigenDeLasEspecies}
	CHARLES DARWIN: {\em The Origin of the Especies} 
	Version electronica, Proyecto Nuremberg (1859)
\bibitem{GeneticaStrickberger}
	M. STRICKBERGER: {\em Gen�tica}, 3a Edici� 
	Ed. Omega, Barcelona.
\bibitem{DelGenAlOrganismo}
	G. MANGIAROTTI: {\em Del gen al organismo. Biologia General}
	Ed. Picin. Padova, Italia.

\section{Vida artificial}

\bibitem{ptGAs}
	HELMUT A. MAYER, {\em ptGA's, Genetic algorithms Using Promoter/Terminator Sequences 
	Evolution of Number, Size and Location of Parameters and Parts of Representation}
\bibitem{RedundanciaRaich}
	RAICH/GHABOUSSI, Autogenesis and Redundancy in GA Representation
	(Abstacts ICGA97)
\bibitem{IntronExonFoster}
	TERENCE SOULE / JAMES A. FOSTER, {\em Comments on the intron/exon 
	distinction as it relates to the genetic programming and biology} 
	(Abstacts ICGA97)
\bibitem{VarLengthKargupta}
	HILLOL KARGUPTA, {\em Relation Learning in Gene Expression: 
	Introns, Variable Legth Representation, And All That}
	(Abstacts ICGA97)
\bibitem{GAvsES}
	Frank Hoffmeister, Thomas Back. 
	{\em Genetic Algortihms and Evolution Strategies: Similarities and Differences},
	Technical Report SyS-91/2,
	Systems Analysis Research Group, LSXI, Dept. of Computer Science,
	University of Dortmund.

\section{Programaci�}

\end{thebibliography}
	
